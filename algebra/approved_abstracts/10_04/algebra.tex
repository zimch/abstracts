\documentclass[a4paper, 12pt]{article}

\usepackage[english, russian]{babel}
\usepackage[T2A]{fontenc}
\usepackage[utf8]{inputenc}
\usepackage{amsthm, amsmath, amsfonts, amssymb, mathtools}
\usepackage{geometry}
\usepackage{indentfirst}
\usepackage{titleps}
\usepackage{soulutf8}
\usepackage{multicol}
\usepackage{tabularx}
\usepackage{pgfplots}
\usepackage{cancel}
\usepackage{import}
\usepackage{xifthen}
\usepackage{pdfpages}
\usepackage{transparent}
\usepackage{setspace}
\usepackage{graphicx}
\usepackage{float}
\usepackage{wrapfig}
\usepackage{contour}
\usepackage{mathrsfs}

\onehalfspacing

\contourlength{1pt}

\pgfplotsset{compat=1.18, width=7cm}

\newpagestyle{main}{
    %\setheadrule{0.4pt}
    \sethead{}{}{}
    %\setfootrule{0.4pt}
    \setfoot{}{\thepage}{}
}

\pagestyle{main}

\theoremstyle{plain}
\newtheorem{theorem}{Теорема}
\newtheorem{corollary}{Следствие}
\newtheorem{lemma}{Лемма}[]
\newtheorem*{lemma*}{Лемма}
\newtheorem*{definition}{Определение}
\newtheorem*{remark}{Замечание}
\newtheorem{example}{Пример}
\newtheorem*{proposition}{Предложение}
\newtheorem*{theorem*}{Теорема}
\newtheorem*{example*}{Пример}
\newtheorem*{corollary*}{Следствие}

\geometry{top=25mm}
\geometry{bottom=30mm}
\geometry{left=20mm}
\geometry{left=20mm}

\newcommand{\incfig}[1]{%
    \def\svgwidth{\columnwidth}
    \import{./figures/}{#1.pdf_tex}
}

\graphicspath{ {./figures/} }

\DeclareMathOperator{\Kerr}{Ker}
\DeclareMathOperator{\Imm}{Im}
\DeclareMathOperator{\Int}{Int}
\DeclareMathOperator{\Mat}{Mat}
\DeclareMathOperator{\End}{End}
\DeclareMathOperator{\sign}{sign}
\DeclareMathOperator{\dist}{dist}
\DeclareMathOperator{\rank}{rank}
\DeclareMathOperator{\diam}{diam}
\DeclareMathOperator{\diag}{diag}
\DeclareMathOperator{\supp}{supp}
\DeclareMathOperator{\grad}{grad}
\DeclareMathOperator{\rot}{rot}
\DeclareMathOperator{\divv}{div}
\DeclareMathOperator{\Ext}{Ext}
\DeclareMathOperator{\Id}{id}
\DeclareMathOperator{\Char}{char}
%\DeclareMathOperator{\dist}{dist}
\DeclareMathOperator*{\id}{id}
\renewcommand{\phi}{\varphi}
\renewcommand{\theta}{\vartheta}
\renewcommand{\epsilon}{\varepsilon}
\newcommand{\R}{\mathbb{R}}
\renewcommand{\C}{\mathbb{C}}
\newcommand{\Q}{\mathbb{Q}}
\newcommand{\N}{\mathbb{N}}
\setcounter{lemma}{11} % вот тут пофиксить
\newcommand{\lrhimani}[1]{\underset{#1}{\underline{\int}}}
\newcommand{\urhimani}[1]{\underset{#1}{\overline{\int}}}
\newcommand{\rhimani}[1]{\underset{#1}{\int}}
\newcommand{\mycontour}[1]{\contour{red}{#1}}
\newcommand{\charf}[1]{\chi_{#1}(x)}
\newcommand{\pfrac}[2]{\frac{\partial #1}{\partial #2}}

\begin{document}
    \title{Алгебра}
    \date{4 октября 2022}
    \maketitle

    \pagebreak

    \begin{theorem}
        $V$ -- в. п. над полем $K$, $\dim V = n < \infty$, $f \in \End(V)$
        \par $f$ диагонализируем $\Leftrightarrow \exists$ базис $V$, состоящий из собст. вект. оператора $f$
    \end{theorem}
    \begin{proof}
        $ $
        \begin{enumerate}
            \item{$\Rightarrow$} $\exists$ базис $v_1, \dots, v_n$
                \[
                    [f]_{\{v_i\}} = \begin{pmatrix}
                        \lambda_1 & & 0 \\
                         & \ddots & \\
                        0 & & \lambda_n
                    \end{pmatrix} \ f(v_i) = \lambda_iv_i \ \quad v_i \text{ -- с. в., отвечающее с. ч. } \lambda_i
                \]
            \item{$\Leftarrow$} $v_1, \dots, v_n$ -- базис из с. в.
                \par $f(v_i) = \lambda_i v_i$
                \[
                    [f]_{\{v_i\}} = \begin{pmatrix}
                        \lambda_1 & & 0 \\
                         & \ddots & \\
                         0 & & \lambda_n
                    \end{pmatrix}    
                \]
        \end{enumerate}
    \end{proof}

    \subsection*{Формулировка теоремы о Жордановой нормальной форме}

    $V$ - в. п. над полем $K$, $\dim V = n < \infty$, $f \in \End(V)$
    \par $\lambda \in K$
    \par
    \par $J_n(\lambda) = \begin{pmatrix}
        \lambda & & & 0 \\
        1 & \ddots & & \\
        & \ddots & & \lambda \\
        0 & & & 1
    \end{pmatrix}$ -- жорданова клетка, размера $n$, отвечающая $\lambda$
    \par
    \par $J_1(\lambda) = \begin{pmatrix}
        \lambda
    \end{pmatrix}$
    \par $J_2(\lambda) = \begin{pmatrix}
        \lambda & 0 \\
        1 & \lambda
    \end{pmatrix}$
    \par $J_3(\lambda) = \begin{pmatrix}
        \lambda & 0 & 0 \\
        1 & \lambda & 0 \\
        0 & 0 & \lambda
    \end{pmatrix}$

    \begin{definition}
        Блочно-диагональная матрица, составленная из жордановых клеток, называется жордановой матрицей 
    \end{definition}

    \[
        \begin{pmatrix}
            \boxed{J_{n_1}(\lambda_1)} & & & 0 \\
            & \boxed{J_{n_2}(\lambda_2)} & & \\
            & & \ddots & & \\
            0 & & & \boxed{J_{n_k}(\lambda_k)}
        \end{pmatrix}    
    \]

    \pagebreak

    \begin{theorem}
        (о жордановой нормальной форме)
        \par $K$ -- алгебраически замкнутое поле
        \par $V$ -- в. п. над $K$, $\dim V < \infty$, $f \in \End(V)$
        \par Тогда $\exists$ базис $V$, такой что матрица $f$ в этом базисе -- жорданова матрица, причем 
        жордановы клетки определены однозначно с точностью до порядка.
    \end{theorem}
    
    \begin{definition}
        Базис в теореме называется жорданвым базисом, а соответствующая жорданова матрица -- канонической
        жордановой формой оператора $f$.
    \end{definition}
    $ $\\
    Жорданова форма однозначна с точностью до порядка следования клеток \\
    Жорданов базис, вообще говоря, неоднозначем (только когда оператор диагонализием с папарно разл. $\lambda$)

\end{document}