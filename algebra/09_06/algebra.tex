\documentclass[a4paper, 12pt]{article}

\usepackage[english, russian]{babel}
\usepackage[T2A]{fontenc}
\usepackage[utf8]{inputenc}
\usepackage{amsthm, amsmath, amsfonts, amssymb, mathtools}
\usepackage{geometry}
\usepackage{indentfirst}
\usepackage{titleps}
\usepackage{soulutf8}
\usepackage{multicol}
\usepackage{tabularx}
\usepackage{pgfplots}
\usepackage{cancel}
\usepackage{import}
\usepackage{xifthen}
\usepackage{pdfpages}
\usepackage{transparent}
\usepackage{setspace}
\usepackage{graphicx}
\usepackage{float}
\usepackage{wrapfig}
\usepackage{contour}
\usepackage{mathrsfs}

\onehalfspacing

\contourlength{1pt}

\pgfplotsset{compat=1.18, width=7cm}

\newpagestyle{main}{
    %\setheadrule{0.4pt}
    \sethead{}{}{}
    %\setfootrule{0.4pt}
    \setfoot{}{\thepage}{}
}

\pagestyle{main}

\theoremstyle{plain}
\newtheorem{theorem}{Теорема}
\newtheorem{corollary}{Следствие}
\newtheorem{lemma}{Лемма}[]
\newtheorem*{lemma*}{Лемма}
\newtheorem*{definition}{Определение}
\newtheorem*{remark}{Замечание}
\newtheorem{example}{Пример}
\newtheorem*{proposition}{Предложение}
\newtheorem*{theorem*}{Теорема}
\newtheorem*{example*}{Пример}
\newtheorem*{corollary*}{Следствие}

\geometry{top=25mm}
\geometry{bottom=30mm}
\geometry{left=20mm}
\geometry{left=20mm}

\newcommand{\incfig}[1]{%
    \def\svgwidth{\columnwidth}
    \import{./figures/}{#1.pdf_tex}
}

\graphicspath{ {./figures/} }

\DeclareMathOperator{\Kerr}{Ker}
\DeclareMathOperator{\Imm}{Im}
\DeclareMathOperator{\Int}{Int}
\DeclareMathOperator{\Mat}{Mat}
\DeclareMathOperator{\End}{End}
\DeclareMathOperator{\sign}{sign}
\DeclareMathOperator{\dist}{dist}
\DeclareMathOperator{\rank}{rank}
\DeclareMathOperator{\diam}{diam}
\DeclareMathOperator{\diag}{diag}
\DeclareMathOperator{\supp}{supp}
\DeclareMathOperator{\grad}{grad}
\DeclareMathOperator{\rot}{rot}
\DeclareMathOperator{\divv}{div}
\DeclareMathOperator{\Ext}{Ext}
\DeclareMathOperator{\Id}{id}
\DeclareMathOperator{\Char}{char}
%\DeclareMathOperator{\dist}{dist}
\DeclareMathOperator*{\id}{id}
\renewcommand{\phi}{\varphi}
\renewcommand{\theta}{\vartheta}
\renewcommand{\epsilon}{\varepsilon}
\newcommand{\R}{\mathbb{R}}
\renewcommand{\C}{\mathbb{C}}
\newcommand{\Q}{\mathbb{Q}}
\newcommand{\N}{\mathbb{N}}
\setcounter{lemma}{11} % вот тут пофиксить
\newcommand{\lrhimani}[1]{\underset{#1}{\underline{\int}}}
\newcommand{\urhimani}[1]{\underset{#1}{\overline{\int}}}
\newcommand{\rhimani}[1]{\underset{#1}{\int}}
\newcommand{\mycontour}[1]{\contour{red}{#1}}
\newcommand{\charf}[1]{\chi_{#1}(x)}
\newcommand{\pfrac}[2]{\frac{\partial #1}{\partial #2}}

\begin{document}
    \title{Алгебра}
    \date{6 сетнября 2022}
    \maketitle

    \pagebreak

    \section*{Делимость в кольце многочленов}

    $K$ -- поле
    \par $K[x]$ -- кольцо многочленов

    \subsection*{\S1 Наибольший общий делитель}

    \begin{definition}
        $f_1, \dots, f_m \in K[x]$
        \par \quad $d \in K[x]$ -- НОД $f_1, \dots, d_m$, если выполняется 2 условия:
        \begin{enumerate}
            \item $d | f_1, \dots, d | f_m$
            \item Если $\tilde d | f_1, \dots \tilde d | f_m$, то $\tilde d | d$
        \end{enumerate}
    \end{definition}

    \begin{illustration}
        $f_1 = \dots = f_m$
        \par \quad Тогда $d = 0$
    \end{illustration}

    \begin{remark}[Обозначение]
        \begin{align*}
            d &= \text{НОД}(f_1, \dots, f_m) \\
            &= \gcd (f_1, \dots, f_m) \\
            &= (f_1, \dots, f_m)
        \end{align*}
    \end{remark}

    \begin{remark}[Вопрос единственности]
        $ $
        \par \quad Пусть $d, \bar d$ -- два НОД$(f_1, \dots, f_m)$
        \par \quad Тогда $\bar d | d$, но и $d | \bar d$
        \par \quad $\Rightarrow \deg d = \deg \bar d$
        \par \quad $\Rightarrow d = c \cdot \bar d, \ c \in K^*$
        \par Тогда будем говорить, что $h$ и $g$ ассоциированы, если $h = cg, \ c \in K^*$
    \end{remark}

    \begin{theorem*}
        $f_1, \dots, f_m \in K[x]$
        \par \quad Тогда $\exists d = $ НОД$(f_1, \dots, f_m)$ и более того $\exists h_1, \dots h_m \in K[x]$:
        \[
            d = h_1 f_1 + \dots + h_m f_m    
        \]
    \end{theorem*}

    \begin{proof}
        $ $
        \begin{enumerate}
            \item $f_1 = \dots = f_m = 0$
                \par Очевидно $0 = 1 \cdot 0 + \dots + 1 \cdot 0$
            \item Среди $f_1, \dots, f_m$ есть хотя бы 1 ненулевой
                \par Расмотрим $I = \{h_1 f_1 + \dots + h_m f_m \ | \ h_i \in K[x]\}$
                \par Очевидно, что $f_1, \dots, f_m \in I$
                \par $I$ содержит ненулевой многочлен наименьшей степени множества $I$
                \par Пусть $d$ -- это ненулевой многочлен (из предположения)
                \par Утверждается, что это и есть НОД$(f_1, \dots, f_m)$
                \par $f_i = q_i d + r_i, \ \deg r_i < \deg d$
                \par $r_i = f_i - q_i d$
                \par $d = h_1 f_1 + \dots + h_m f_m, \ d \in I$
                \par $r_i = -(h_1 q_i f_1) + \dots + (1 - h_i q_i) f_i + \dots + (-h_m q_m) f_m, \ r_i \in I$ % ВХАТАВУААК??
                \par Так как $d$ -- ненулевой многочлен наименьшей степени в $I$
                \par то $r_i = 0, \ f_i = q_i d, \ d | f_i$
                \par $d = h_1 f_1 + \dots + h_m f_m$ \quad (т. к. $d \in I$)
                \par $\tilde d | f_1, \dots, \tilde d | f_m$
                \par $f_i = \tilde d_i \tilde q_i$
                \par $d = \tilde d (h_1 \tilde q_1 + \dots h_m \tilde q_m)$
                \par $\tilde d | d \Rightarrow d = $ НОД$(f_1, \dots, f_m)$
                \par По выбору $d \in I \Rightarrow d$ допускает лин. представление
        \end{enumerate}
    \end{proof}

    \subsection*{\S2 Алгоритм Евклида}

    \begin{lemma}
        $f, g, q \in K[x]$
        \par \quad НОД$(f, g)$ = НОД$(f - qg, g)$
    \end{lemma}

    \begin{proof}
        Пусть $d =$ НОД$(f, g)$
        \par \quad $\tilde d =$ НОД$(f - qg, g)$
        \par \quad $d | f, d | g \Rightarrow d | (f-qg)$
        \par \quad $\Rightarrow d | \tilde d$ (т. к. $\tilde d$ -- НОД$(f - qg, g)$)
        \par \quad $\tilde d | f - qg, \ \tilde d | f$
        \par \quad $f = (f - qg) + qg$
        \par \quad $\tilde d | f$
        \par \quad $\tilde d$ -- общий делитель $f$ и $g$
        \par \quad $\Rightarrow \tilde d | d$
        \par \quad $\tilde d = cd, \ c \in K^*$
    \end{proof}

    \pagebreak
    \textbf{\textit{Рассмотрим алгоритм:}}

    \begin{align*}
        &r_0 = f, \ r_1 = g \\
        \boxed{=} \ &r_0 = q_1 r_1 + r_2, \ \deg r_2 < \deg r_1 \\
        &\dots \\
        &r_{i-1} = q_i r_i + r_{i+1}, \ \deg r_{i+1} < \deg r_i \\
        &r_{n-2} = q_{n-1} r_{n-1} + r_{n}, \ \deg r_n < \deg r_{n-1} \\
        &r_{n-1} = q_n r_n \\
        &r_n \text{ -- последний ненулевой остаток}
    \end{align*}

    \par \quad Из $\boxed{=}$ : $r_{i+1} = r_{i-1} - q_i r_i$. По лемме: НОД$(r_{i-1}, r_i)$ = НОД$(r_{i+1}, r_i)$ = НОД$(r_i, r_{i+1})$

    \begin{align*}
        &\text{НОД}(r_0, r_1) = \\
        &\text{НОД}(r_1, r_2) = \\
        &\dots \\
        &\text{НОД}(r_{n-1}, r_n) = r_n
    \end{align*}

    \subsection*{\S3 Взаимно простые}

    \begin{definition}
        $f_1, \dots, f_m \in K[x]$ взаимно простые, если
        \[
            \text{НОД}(f_1, \dots, f_m) = 1    
        \]
        \par (То есть общими делителями являются только ненулевые константы)
    \end{definition}

    \begin{remark}
        Следует различать простоту и попарную простоту:
        \par $f_1, \dots, f_m$ -- попарно просты, если $\forall i, \ j, \ i \not= j \ \ f_j$ и $f_i$ -- взаимно простые
    \end{remark}

    \begin{theorem}[Критерий взаимной простоты]
        $ $
        \par \quad $f_1, \dots, f_m \in K[x]$ взаимно просты $\Leftrightarrow$
        \par \quad $\exists h_1, \dots, h_m \in K[x]$
        \par \quad т. ч. $h_1 f_1 + \dots + h_m f_m = 1$
    \end{theorem}

    \begin{proof}
        $ $
        \begin{enumerate}
            \item[$\Rightarrow$] 1 -- НОД. По т. \S1 допускает линейное представление
            \item[$\Leftarrow$] $1 | f_1, \dots 1 | f_m$
            \par Пусть $\tilde d$ -- какой-то общий делитель
            \par $\tilde d | f_1, \dots, \tilde d | f_m$
            \par $\Rightarrow \tilde d | 1$
        \end{enumerate}
    \end{proof}

    \begin{theorem}
        $f, g_1, \dots, g_m \in K[x]$
        \par $f, g_i$ взаимно простые $\forall i = 1, \dots, m$
        \par Тогда $f$ взаимно прост с $g_1 \cdot \ldots \cdot g_m$ % ТУТ КОМПОЗИЦИЯ или умножение ??
    \end{theorem}

    \begin{proof}
        $f, g_1$ взаимно просты, тогда
        \par $1 = f u_i + g_i v_i, \quad u_i, v_i \in K[x]$
        \par $1 - f u_i = g_i v_i, \quad i = 1, \dots, m$
        \[
            \prod_{i=1}^m (1 - f u_i) = g_1 \cdot \ldots \cdot g_m \cdot v_1 \cdot \ldots \cdot v_m =
        \]
        \[
            = 1 + f \cdot A % WHAT IS A
        \]
        \[
            1 = f (-A) + g_1 \cdot \ldots \cdot g_m \cdot v_1 \cdot \ldots \cdot v_m
        \]
        \par $\Rightarrow$ по т. 1: $f$ и $g_1 \cdot \ldots \cdot g_m$ -- взаимно просты
    \end{proof}

    \begin{theorem}
        $ $
        \par \quad $f, g, h \in K[x]$
        \par \quad $f | gh$ и $f$ и $g$ взаимно просты
        \par \quad Тогда $f | h$
    \end{theorem}

    \begin{proof}
        $\exists u, v \in K[x]$
        \par \quad $f \cdot u + g \cdot v = 1$
        \par \quad $fhu + ghv = 1$
        \par \quad $\dots$
        \par \quad $f \quad \quad f \Rightarrow f | h$  % Я ХЗ ЧТО У МЕНЯ ТУТ НАПИСАНО
    \end{proof}

    \subsection*{\S4 Неприводимые многочлены ОТА для $K[x]$}

    \begin{definition}
        $f \in K[x] \setminus K$
        \par \quad $f$ -- составной, если
        \par \quad $\exists h, g \not \in K^*$
        \par \quad \ $f = hg$
        \par $ $
        \par \quad Если таких $h$ и $g$ не существует, то $f$ называется неприводимым (над $K$)
    \end{definition}

    \par \quad $f$ неприводим:
    \par \quad \quad $f = hg \Rightarrow h \in K^*$ или $g \in K^*$
    \par \quad \quad векторный сомножитель ассоциированы с $f$
    \par $ $
    \par \quad $f$ неприводим, если его делители -- это в точности константы и ассоциированные с $f$ (полный аналог простых)

    \begin{theorem}[ОТА]
        $ $
        \par \quad $0 \not = f \in K[x]$
        \par \quad Тогда $\exists c \in K^*$ и неприводимые $h_1, \dots, h_m$ со старшим коэффициентом $1$ таким, что
        \[
            f = c \cdot h_1 \cdot \ldots \cdot h_m \quad (m \ge 0)    
        \]
        \par \quad $c$ совпадает со старшим коэффициентом $f$ и тогда разложение единственно с точностью до порядка сомножителя
    \end{theorem}

    \begin{proof}
        $ $
        \begin{enumerate}
            \item Существование:
            \begin{enumerate}
                \item[a)] $f \in K^*$ -- очевидно \quad $c = f, \ m = 0$
                \item[б)] $\deg f > 0$
                \par \quad Если $f$ неприводим, то остановимся
                \par \quad Если $f$ составной, то разложим $f = u \cdot v, \ \deg u < \deg v < \deg f$
                \par \quad Дальше так же поступим с каждым из сомножителей
                \par \quad Процесс оборвется за конечное число шагов
                \par \quad $f = \tilde h_1 \cdot \ldots \cdot \tilde h_m, \ \tilde h_i$ -- неприводимый
                \par \quad $\tilde h_i = c_i h_i, \ h_i$ -- неприводимый, старший коэффициент $1$
                \par \quad $f = (\underbrace{c_1 \cdot \ldots \cdot c_m}_{=m}) \cdot h_1 \cdot \ldots \cdot h_m$
            \end{enumerate}
            \item Единственность:
            \par \quad $f = c h_1 \cdot \ldots \cdot h_m = e \cdot g_1 \cdot \ldots \cdot g_n, \quad h_i, g_i$ -- неприв., ст. коэфф. $1$
            \par \quad $\Rightarrow c = e, \ m = n$ и (после перенумерации) $g_i = h_i, \ c = 1, \dots, m$
            \par \quad НУО: \quad $m \le n$
            \par Индукция по $m$:
            \par \quad База $m = 0 \quad f = c = eg_1 \dots g_m \quad \deg f = 0 \Rightarrow n = 0 \Rightarrow c = e$
            \par \quad Индкционный переход: $m \ge 1$ \quad $h_m | eg_1 \dots g_m \quad \boxed{*}$
            \par \quad Неприводимые либо ассоциированы, либо взаимно простые.
            \par \quad Если $h_m$ не ассоциирован ни с одним из $g_1, \dots, g_m$, то $h_m$ взаимно прост с $g_1, \dots, g_m$,
            а значит $h_m$ взаимно прост с $eg_1 \cdot \ldots \cdot g_m$ (Т. 2)
            \par \quad $\rightarrow$ но многочлен не может делить и быть взаимно простым $\rightarrow$ что противоречит $\boxed{*}$ % ВОТ ТУТ Я ЧЕТА НИПООН
            \par \quad (т. к. из $\boxed{*}$ следует, что $K^* \not \ni h_m = $ НОД$(h_m, eg_1 \cdot \ldots \cdot g_m)$)
            \par \quad $\Rightarrow \exists i : h_m$ ассоциируется с $g_i$. НУО $i = n$
            \par \quad т. к. $h_m, g_n$ со старшем коэффициентом $1$, из ассоциированности следует, что $h_m = g_n$
            \[
                ch_1 \cdot \ldots \cdot h_m = eg_1 \cdot \ldots \cdot g_n    
            \]
            \[
                ch_1, \cdot \ldots \cdot h_{m-1} = eg_1 \cdot \ldots \cdot g_{n-1}
            \]
            \par \quad По индукционному предположению: $m-1 = n-1, \ c = e$ и после перенумерации:
            \[
                g_1 = h_1, \dots, = g_{m-1} = h_{m-1} \Rightarrow m = n \text{ и } g_m = h_m    
            \]
        \end{enumerate}
    \end{proof}

    \subsection*{\S5 Алгебраически замкнутые поля}

    \par $K[x]$
    \par $x - a, \ a \in K$ всегда неприводимые

    \begin{illustration*}
        $x^2 + 1$ неприводим над $\R$
        \par \quad $x^3 + x^2 + 1, \ x^3 + x + 1, \ x^2 + x + 1$ неприводимы над $\mathbb{F}_2$
    \end{illustration*}

    \par Для многочленов степени $2$ и $3$ из отсутствия корней следует неприводимость

\end{document}