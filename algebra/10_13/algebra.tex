\documentclass[a4paper, 12pt]{article}

\usepackage[english, russian]{babel}
\usepackage[T2A]{fontenc}
\usepackage[utf8]{inputenc}
\usepackage{amsthm, amsmath, amsfonts, amssymb, mathtools}
\usepackage{geometry}
\usepackage{indentfirst}
\usepackage{titleps}
\usepackage{soulutf8}
\usepackage{multicol}
\usepackage{tabularx}
\usepackage{pgfplots}
\usepackage{cancel}
\usepackage{import}
\usepackage{xifthen}
\usepackage{pdfpages}
\usepackage{transparent}
\usepackage{setspace}
\usepackage{graphicx}
\usepackage{float}
\usepackage{wrapfig}
\usepackage{contour}
\usepackage{mathrsfs}

\onehalfspacing

\contourlength{1pt}

\pgfplotsset{compat=1.18, width=7cm}

\newpagestyle{main}{
    %\setheadrule{0.4pt}
    \sethead{}{}{}
    %\setfootrule{0.4pt}
    \setfoot{}{\thepage}{}
}

\pagestyle{main}

\theoremstyle{plain}
\newtheorem{theorem}{Теорема}
\newtheorem{corollary}{Следствие}
\newtheorem{lemma}{Лемма}[]
\newtheorem*{lemma*}{Лемма}
\newtheorem*{definition}{Определение}
\newtheorem*{remark}{Замечание}
\newtheorem{example}{Пример}
\newtheorem*{proposition}{Предложение}
\newtheorem*{theorem*}{Теорема}
\newtheorem*{example*}{Пример}
\newtheorem*{corollary*}{Следствие}

\geometry{top=25mm}
\geometry{bottom=30mm}
\geometry{left=20mm}
\geometry{left=20mm}

\newcommand{\incfig}[1]{%
    \def\svgwidth{\columnwidth}
    \import{./figures/}{#1.pdf_tex}
}

\graphicspath{ {./figures/} }

\DeclareMathOperator{\Kerr}{Ker}
\DeclareMathOperator{\Imm}{Im}
\DeclareMathOperator{\Int}{Int}
\DeclareMathOperator{\Mat}{Mat}
\DeclareMathOperator{\End}{End}
\DeclareMathOperator{\sign}{sign}
\DeclareMathOperator{\dist}{dist}
\DeclareMathOperator{\rank}{rank}
\DeclareMathOperator{\diam}{diam}
\DeclareMathOperator{\diag}{diag}
\DeclareMathOperator{\supp}{supp}
\DeclareMathOperator{\grad}{grad}
\DeclareMathOperator{\rot}{rot}
\DeclareMathOperator{\divv}{div}
\DeclareMathOperator{\Ext}{Ext}
\DeclareMathOperator{\Id}{id}
\DeclareMathOperator{\Char}{char}
%\DeclareMathOperator{\dist}{dist}
\DeclareMathOperator*{\id}{id}
\renewcommand{\phi}{\varphi}
\renewcommand{\theta}{\vartheta}
\renewcommand{\epsilon}{\varepsilon}
\newcommand{\R}{\mathbb{R}}
\renewcommand{\C}{\mathbb{C}}
\newcommand{\Q}{\mathbb{Q}}
\newcommand{\N}{\mathbb{N}}
\setcounter{lemma}{11} % вот тут пофиксить
\newcommand{\lrhimani}[1]{\underset{#1}{\underline{\int}}}
\newcommand{\urhimani}[1]{\underset{#1}{\overline{\int}}}
\newcommand{\rhimani}[1]{\underset{#1}{\int}}
\newcommand{\mycontour}[1]{\contour{red}{#1}}
\newcommand{\charf}[1]{\chi_{#1}(x)}
\newcommand{\pfrac}[2]{\frac{\partial #1}{\partial #2}}

\setcounter{lemma}{0}

\begin{document}
    \title{Алгебра}
    \date{13 октября 2022}
    \maketitle

    \pagebreak

    \subsection*{\S Жорданова форма оператора с единственным собственным числом}

    \par $K$ -- алгебраически замкнуто
    \par $V$ -- векторное пространство над $K, \ \dim V \ n < \infty$
    \par $\phi \in \End(K)$ -- линейный оператор
    \par $\chi_\phi(t) = (-1)^n (t - \lambda)^n$
    \par $ $
    \par Надо показать, что $\exists$ базис, в котором $[\phi]$ состоит из жордановых клеток, отвечающих
    собственному числу $\lambda$
    \[
        \{0\} \subsetneqq U_0(\lambda) \subsetneqq U_1(\lambda) \subsetneqq \dots \subsetneqq U_n(\lambda) = V    
    \]
    \par $U_i, \ i = 0, \dots, n$ -- корневые подпространства

    \begin{lemma}
        Если $U_i(\lambda) = U_{i+1}(\lambda)$, то $U_i(\lambda) = U_k(\lambda), \ \forall k \ge i$
    \end{lemma}

    \begin{proof}
        Достаточно доказать, что $U_{i+2}(\lambda) = U_{i+1}(\lambda)$
        \par \quad $v \in U_{i+2}(\lambda)$ \quad $(\phi - \lambda \Id)^{i+2}(v) = 0$ \quad $(\phi - \lambda \Id)(v) \in U_{i+1}(\lambda) = U_i(\lambda)$
        \par \quad $(\phi - \lambda \Id)^{i+1}((\phi - \lambda \Id)(v)) = 0$ \quad $\Rightarrow (\phi - \lambda \Id)^i((\phi - \lambda \Id)(v)) = 0$ 
        \par \quad $(\phi - \lambda\Id)^{i+1}(v) = 0$ \quad $v \in U_{i+1}(\lambda)$
    \end{proof}

    \par Пусть $m$ неизм. инд.:
    \par $U_m(\lambda) = U_{m+1}(\lambda) \Rightarrow V = U_m(\lambda) \Rightarrow U_{m-1} \subsetneqq U_m(\lambda)$

    \begin{definition}
        Если $v \in U_i(\lambda) \ v \not \in U_{i-1}(\lambda)$, то
        \par \quad $v$ -- корневой вектор высоты $i$
        \par \quad $v_1^{(m)}, \dots, v_{k_m}^{(m)}$ -- относительный базис $U_m(\lambda)$ относительно $U_{m-1}(\lambda)$
        \par \quad $v_j^{(m-1)} = (\phi - \lambda\Id)v_j^{(m)}, \ j = 1, \dots, k_m$
        \par \quad $v_j^{(m-1)} \in U_{m-1}(\lambda)$ и отн. л. н. отн. $U_{m-2}(\lambda)$
        \par \quad Дополним до относительного базиса $U_{m-1}(\lambda)$ отн. $U_{m-2}(\lambda)$
    \end{definition}

    \par \incfig{drawing}

    \par $i+1: \ \dots$
    \par $i: \ v_j^{(i)} = (\phi - \lambda\Id)v_j^{(i+1)} \quad i = 1, \dots, k_{i+1}$
    \par \quad и дополним до относительного базиса $U_i(\lambda)$ относительно $U_{i-1}(\lambda)$
    \[
        v_j^{(i)}, \dots \quad j = k_{i+1} + 1, \dots, k_i
    \]

    \par $k_m \le k_{m-1} \le \dots \le k_1$
    \par Обычно можно увидеть следующее: (диагр.)
    \par
    \begin{tabular}{ | c | c | c |}
        \hline
         & &  \\
        \hline
    \end{tabular} \par
    \begin{tabular}{ | c | c | c | c |}
        \hline
         & & & \\
        \hline
    \end{tabular} \par
    \dots \par
    \begin{tabular}{ | c | c | c | c | c | c | c | c | c | c |}
        \hline
         & & & & & & & & &\\
        \hline
    \end{tabular} \par
    \begin{tabular}{ | c | c | c | c | c | c | c | c | c | c | c |}
        \hline
         & & & & & & & & & &\\
        \hline
    \end{tabular}

    \begin{illustration*}
        $\{v_j^{(i)} \ | \ 1 \le i \le m; 1 \le j \le k_j \}$ -- базис $V$
        \par \quad $\{v_j^{(i)}\}$ -- относительный базис $U_1(\lambda)$ относительно $U_0(\lambda)$
        \par \quad то есть $\{v_j^{(i)}\}$ -- базис $U_1(\lambda)$
        \par \quad $\{v_j^{(1)}, \ v_j^{(2)}\}$ -- базис $U_2(\lambda)$
        \par \quad $v_j^{(1)}$ -- базис $U_1(\lambda)$, \quad $v_j^{(2)}$ -- базис $U_2(\lambda)$ относительно $U_1(\lambda)$
        \par По индукции:
        \par \quad $\{v_j^{(1)}\} \bigcup \dots \bigcup \{v_j^{(i)}\}$ -- базис $U_i(\lambda)$
        \par \quad $\{v_j^{(i+1)}\}$ -- относительный базис $U_{i+1}(\lambda)$ относительно $U_i(\lambda)$
    \end{illustration*}

    \par \textbf{Повторение:} Инвариантное подпространство: $V$ -- векторное пространство 
    \par $\exists W \subset E$ $\Rightarrow \exists T : V \rightarrow V \Rightarrow T(W) \subset W \Rightarrow$ инв.

    \par $ $
    \par Покажем, что пространство, порожденное векторами из $1$го столбца , инвариантно

    \par $ $ \par \incfig{drawing-1}
    \[
        k\phi(v_j^{(k)}) = \lambda \phi_j^{(k)} + v_j^{(k-1)} \quad k \ge 2
    \]
    \[
        \phi(v_j^{(1)}) = \lambda v_j^{(1)} \quad k \ge 1
    \]
    \[
        <v_j^{(i), \dots, v_j^{(1)}}>    
    \]
    \par $\phi$ -- инвариантно

    \par $ $
    \par $V = \sum_j <v_j^{(i)}, \dots, v_j^{(1)}> = \bigoplus_j <v_j^{(i)}, \dots, v_j^{(1)}>$
    \par $V$ расскладывается в $\bigoplus$ инвариантных подпространств с выбранным базисом
    \par Клетка в $[\phi]$, отвечающая $j$-му столбцу диаграммы:
    \[
        \begin{pmatrix}
            \lambda & 0 & & \dots & & & 0 \\
            1 & \lambda & 0 & \dots & & & 0 \\
            0 & 1 & \lambda & & & & 0 \\
            \vdots & & & \ddots & & & \vdots\\
            0 & & & & & \lambda & 0 \\
            0 & & & \dots & & 1 & \lambda
        \end{pmatrix} \text{ -- единственная жорданова клетка}
    \]

    \pagebreak
    \begin{remark}
        $ $
        \par Сколько клеток для $\lambda$? -- сколько векторов на нижнем уровне $=$ размерность пространства 
        собственных векторов -- геометрическая кратность $\lambda = \dim U_1(\lambda)$
        \par Размер максимальной клетки = наименьшее $m$ т. ч.:
        \[
            U_m(\lambda) = \bigcup_{i=1}^\infty U_i(\lambda) = U_a(\lambda)    
        \]
        \par \quad $a$ -- алгебраическая кратность $\lambda$
        \par Суммарный размер клеток = $\dim U_m(\lambda) = $ алгебраическая кратность
    \end{remark}

    \begin{remark}
        Лучше снизу вверх, чем наоборот
        \par \quad Мы пытаемся найти $(\phi - \lambda \Id) \boxed{v_j^{(2)}} = v_j^{(1)} \quad \boxed{*}$
        \par \quad и уже знаем: $v_1^{(1)}, \dots , v_{k_1}^{(1)}$ -- базис $U_1(\lambda)$
        \par \quad Но $\boxed{*}$ не всегда разешимо
    \end{remark}

    \begin{theorem}[о попарном разложении]
        $ $
        \par $\chi_\phi(t) = (-1)^n \prod_{i=1}^n (t - \lambda_i)^{a_i}$, $\lambda_i$ -- попарно разл.
        \par $V = \bigoplus_{i=1}^n U_{a_i}(\lambda_i)$ 
    \end{theorem}

    \subsection*{\S Единственность жордановой формы}

    \par $[\phi]$
    \[
        A = \begin{pmatrix}
            \boxed{
            \begin{matrix}
                J_{k_1}(\lambda_1) & & \\
                & \ddots & \\
                & & J_{k_s}(\lambda_1)
            \end{matrix}} & & \\
            & \boxed{
                \begin{matrix}
                    J_{l_1}(\lambda_2) & & \\
                    & \ddots & \\
                    & & J_{l_r}(\lambda_2)
                \end{matrix}} & \\
            & & \ddots
        \end{pmatrix}    
    \]

    \par $A - \lambda I$ : если $\lambda$ не совпадает с $\lambda_i$ в текущем блоке, на диагонали ненулевые элементы
    $\Rightarrow$ обратим (ранг $=$ размеру)
    \par А для совпадающего:
    \[
        n - \rank (A - \lambda I) = \text{ колич. клеток, отвеч. с. ч. } \lambda =
    \]
    \[
        = \text{ размерность пространства решений } (A - \lambda I) x = 0 = \dim U_1(\lambda)    
    \]
    \par (ранг = размер - 1)

    \par $ $
    \par Фиксируем $\lambda$
    \par в жордановой форме:
    \par $d_1$ -- клеток размера $1$
    \par $d_2$ -- клеток размера $2$
    \par \dots
    \par $d_m$ -- клеток размера $m$
    \par $ $
    \par $d_1 + d_2 + \dots + d_m = \dim U_1(\lambda)$
    \par $\dim U_2(\lambda) = \dim \{x : (A - \lambda I)^2 x = 0\} = n - \rank(A - \lambda I)^2$

    \[
        \begin{pmatrix}
            0 & & & & \\
            1 & 0 & & \\
             & & \ddots & & \\
             & & & 0 & \\
             & & & 1 & 0
        \end{pmatrix}^2 = \begin{pmatrix}
            0 & & & & \\
            0 & 0 & & \\
            1 & & \ddots & & \\
             & & & 0 & \\
             & & 1 & 0 & 0
        \end{pmatrix} 
    \]

    \par $d_1 + 2d_2 + \dots + 2d_m = \dim U_2(\lambda)$
    \par $d_1 + 2d_2 + \dots + id_i + \dots + id_m = \dim U_i(\lambda)$
    \par $d_{i+1} + \dots + d_m = \dim U_{i+1}(\lambda) - \dim U_i(\lambda)$
    \par $d_i + d_{i+1} + \dots + d_m = \dim U_i(\lambda) - \dim U_{i-1}(\lambda)$
    \par $d_i = 2 \dim U_i(\lambda) - \dim U_{i-1}(\lambda) - \dim U_{i+1}(\lambda)$
    \par Размерность корневого пространства от выбора базиса никак не зависит
    \par $d_1 = 2 \dim U_1(\lambda) - \dim U_2(\lambda) - 0$
    \par $i = m : \ 2 \dim U_m(\lambda) - \dim U_{m-1}(\lambda) - \underbrace{\dim U_{m+1}(\lambda)}_{= \dim U_m(\lambda)} = \dim U_m(\lambda) - \dim U_{m-1}(\lambda) =$
    \par $= \sum_{k_1, \dots, k_p \in \{1, \dots, n\}, k_i \not= k_j, i\not= j} A_{k_1h_1} \dots A_{k_ph_p} f_{k_1} \wedge \dots f_{k_p} = $
    \par $\sum_{k = (k_1, \dots, k_p), 1 \le k_1 < \dots < k_p \le n} \sum_{\pi} A_{k_1h_1} \dots A_{k_ph_p} \sign \pi f_k$

\end{document}

