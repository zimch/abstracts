\documentclass[a4paper, 12pt]{article}

\usepackage[english, russian]{babel}
\usepackage[T2A]{fontenc}
\usepackage[utf8]{inputenc}
\usepackage{amsthm, amsmath, amsfonts, amssymb, mathtools}
\usepackage{geometry}
\usepackage{indentfirst}
\usepackage{titleps}
\usepackage{soulutf8}
\usepackage{multicol}
\usepackage{tabularx}
\usepackage{pgfplots}
\usepackage{cancel}
\usepackage{import}
\usepackage{xifthen}
\usepackage{pdfpages}
\usepackage{transparent}
\usepackage{setspace}
\usepackage{graphicx}
\usepackage{float}
\usepackage{wrapfig}
\usepackage{contour}
\usepackage{mathrsfs}

\onehalfspacing

\contourlength{1pt}

\pgfplotsset{compat=1.18, width=7cm}

\newpagestyle{main}{
    %\setheadrule{0.4pt}
    \sethead{}{}{}
    %\setfootrule{0.4pt}
    \setfoot{}{\thepage}{}
}

\pagestyle{main}

\theoremstyle{plain}
\newtheorem{theorem}{Теорема}
\newtheorem{corollary}{Следствие}
\newtheorem{lemma}{Лемма}[]
\newtheorem*{lemma*}{Лемма}
\newtheorem*{definition}{Определение}
\newtheorem*{remark}{Замечание}
\newtheorem{example}{Пример}
\newtheorem*{proposition}{Предложение}
\newtheorem*{theorem*}{Теорема}
\newtheorem*{example*}{Пример}
\newtheorem*{corollary*}{Следствие}

\geometry{top=25mm}
\geometry{bottom=30mm}
\geometry{left=20mm}
\geometry{left=20mm}

\newcommand{\incfig}[1]{%
    \def\svgwidth{\columnwidth}
    \import{./figures/}{#1.pdf_tex}
}

\graphicspath{ {./figures/} }

\DeclareMathOperator{\Kerr}{Ker}
\DeclareMathOperator{\Imm}{Im}
\DeclareMathOperator{\Int}{Int}
\DeclareMathOperator{\Mat}{Mat}
\DeclareMathOperator{\End}{End}
\DeclareMathOperator{\sign}{sign}
\DeclareMathOperator{\dist}{dist}
\DeclareMathOperator{\rank}{rank}
\DeclareMathOperator{\diam}{diam}
\DeclareMathOperator{\diag}{diag}
\DeclareMathOperator{\supp}{supp}
\DeclareMathOperator{\grad}{grad}
\DeclareMathOperator{\rot}{rot}
\DeclareMathOperator{\divv}{div}
\DeclareMathOperator{\Ext}{Ext}
\DeclareMathOperator{\Id}{id}
\DeclareMathOperator{\Char}{char}
%\DeclareMathOperator{\dist}{dist}
\DeclareMathOperator*{\id}{id}
\renewcommand{\phi}{\varphi}
\renewcommand{\theta}{\vartheta}
\renewcommand{\epsilon}{\varepsilon}
\newcommand{\R}{\mathbb{R}}
\renewcommand{\C}{\mathbb{C}}
\newcommand{\Q}{\mathbb{Q}}
\newcommand{\N}{\mathbb{N}}
\setcounter{lemma}{11} % вот тут пофиксить
\newcommand{\lrhimani}[1]{\underset{#1}{\underline{\int}}}
\newcommand{\urhimani}[1]{\underset{#1}{\overline{\int}}}
\newcommand{\rhimani}[1]{\underset{#1}{\int}}
\newcommand{\mycontour}[1]{\contour{red}{#1}}
\newcommand{\charf}[1]{\chi_{#1}(x)}
\newcommand{\pfrac}[2]{\frac{\partial #1}{\partial #2}}

\begin{document}
    \title{Алгебра}
    \date{27 сентября 2022}
    \maketitle

    \pagebreak

    \section*{Линейные операторы}
    \subsection*{\S Линейные операторы}

    \begin{definition}
        $V$ -- векторное пространство над $K$
        \par Линейное отображение $f : V \rightarrow V$ называется линейным оператором на $V$
    \end{definition}

    \begin{illustration}
        $ $
        \begin{enumerate}
            \item $V = K^n$
                \par $A \in M_n(K)$
                \par $f(v) = Av$
                \par $f$ -- лин.
            \item $V = K[x]$
                \par $f = \frac{d}{dx}$
                \par $g \in K[x]$
                \par $f(g) = g' = \frac{dg}{dx}$
            \item Интегральный оператор
                \par $V = C([0, 1])$
                \par $\mathcal{K}(x, y) \in C([0, 1]^2)$
                \par $g \mapsto h(x) = \int_0^1 \mathcal{K}(x, y)g(y)dy$
        \end{enumerate}
    \end{illustration}

    $ $

    \par Множество всех линейных операторов на $V$ -- эндоморфизм:
    \[
        \End(V), \quad \mathscr{L}(V) 
    \]
    \par Пусть $f, g \in \End(V), \ c \in K$
    \par \quad Определены:
    \par \quad \ \ $f + g \in \End(V)$, \quad $c \cdot f \in \End(V)$
    \par \quad $\End(V)$ -- векторное пространство над $K$
    \par \quad $f \circ g : V \rightarrow V$, \quad $fg = f \circ g \in \End(V)$, (умножение операторов -- композиция)
    \par $(\End(V), \ +, \ \circ)$ -- кольцо
    \par $1_{\End(V)} = \Id_V$
    \[
        f \circ \Id_V = \Id_V \circ f = f \Rightarrow \text{Кольfцо с } 1    
    \]

    \par \textbf{(*) Аксиома:} $c \in K, \ v \in V$
    \[
        f(cg) = (cf)g = c(fg)   
    \]
    \[
        (f(cg))(v) = f((cg)(v)) = f(cg(v)) =
    \]
    \[
        = c(f(g(v))) = c(fg)(v) = (c(fg)(v)) =    
    \]
    \[
        = ((cf)g)(v) = (cf)(g(v)) = cf(g(v)) = (c(fg)(v))
    \]

    \begin{definition}
        Множество $A$ с тремя операциями:
        \begin{align*}
            &+ : A \times A \rightarrow A \\
            &\circ : A \times A \rightarrow A \\
            &\cdot : A \times A \rightarrow A
        \end{align*}
        \par т. ч.:
        \begin{align*}
            &A, \ +, \ \cdot \text{ -- векторное пространство} \\
            &A, \ +, \ \circ \text{ -- кольцо} \\
            &A, \ +, \ \circ, \ \cdot \text{ -- алгебра над } K
        \end{align*}
    \end{definition}

    \begin{illustration*}[Примеры алгебр]
        $ $
        \begin{enumerate}
            \item Алгебра квадратных матриц над $K$
            \item $K[x]$ -- алгебра многочленов
            \item $\mathbb{H}$ -- алгебра кватернионов над $\R$
        \end{enumerate}
    \end{illustration*}

    \par $ $
    \par Мы проверим, что $\End(V)$ -- алгебра над $K$:
    \par \quad $\dim V < \infty$
    \par \quad Зафиксируем базис $v_1, \dots, v_n$
    \par \quad $f \in \End(V)$
    \par \quad $A = [f]_{\{v_i\}}$ -- матрица линейного оператора в базисе $\{v_i\}$
    \[
        f(v_j) = \sum_{i=1}^n c_{ij} v_i    
    \]
    \par \quad $j$-ый столбец $A = \begin{pmatrix}
        c_{1j} \\
        \vdots \\
        c_{nj}
    \end{pmatrix}$

    \par $\End(V) \rightarrowtail M_n(K)$ % ТУТ ДРУГАЯ СТРЕЛКА Я ХЗ КАК ТАКУЮ НАРИСОВАТЬ ТИПА: >-->->
    \par \quad $f \mapsto [f]_{\{v_i\}}$
    \par \quad $[f+g]_{\{v_i\}} = [f]_{\{v_i\}} + [g]_{\{v_i\}}$
    \par \quad $[cf]_{\{v_i\}} = c[f]_{\{v_i\}}$
    \par \quad $[fg]_{\{v_i\}} = [f]_{\{v_i\}}[g]_{\{v_i\}}$
    \par $\End(V) \cong M_n(K)$

    \subsection*{\S Собственные числа и собственные векторы}

    \par $V$ -- векторное пространство над $K$
    \par $f \in \End(V)$

    \begin{definition}
        $\lambda \in K$ -- собственное число $f$, если $\exists v \not = 0$
        \[
            f(v) = \lambda v    
        \]
        \par Если $\lambda$ -- собственное число $f$, то всякий $v \in V$ такой, что $f(v) = \lambda v$ называется
        собственным вектором $f$, отвечающим собственному числу $\lambda$
    \end{definition}

    \begin{illustration*}
        $ $
        \begin{enumerate}
            \item $V = K[x]$
                \par $\frac{d}{dx} : K[x] \rightarrow K[x]$
                \par $\frac{d}{dx} g = \lambda y$
                \par Если $g \not= 0$, то $\lambda = 0$
                \par Единственное собственное число -- $0$
                \par $g' = 0$
                \par $ $
                \par Если $\Char K = 0$, то $V_k = \{const\}$
                \par Если $\Char K = p > 0$:
                \[
                    V_\lambda = <1, x^p, x^{2p}, \dots> = K[x^p]    
                \]
            \item $\dim V = n < \infty$
                \par $v_1, \dots, v_n$ -- базис
                \par $f \in \End(V)$
                \par $[f(v)]_{\{v_i\}} = [f]_{\{v_i\}} [v]_{\{v_i\}}$
                \par $v$ -- собственный вектор с. ч. $\lambda \Leftrightarrow [v]_{\{v_i\}}$ -- с. в., отвечающий c. ч. $\lambda$
                \par $f(v) = \lambda v \Leftrightarrow A[v]_{\{v_i\}} = \lambda[v]_{\{v_i\}}$
        \end{enumerate}
    \end{illustration*}

    \par $A$ -- матрица оператора $f$ в базисе $v_1, \dots, v_n$
    \par $\chi_A(t) = \det(A - tI) = \chi_f(t)$ -- характеристический многочлен оператора $f$
    \par Если $C$ -- матрица перехода от одного базиса к другому, то матрица $A$ оператора $f$ в другом
    базисе заменяется на сопряженную.
    \[
        A \rightarrow CAC^{-1}    
    \]
    \[
        \det(CAC^{-1} - tI) = \det(CAC^{-1} - tCC^{-1}) = \det(CAC^{-1} - C(tI)C^{-1})    
    \]
    \[
        = \det(C(A - tI)C^{-1}) = \det C \cdot \det (A-tI) \cdot \det C^{-1} =
    \]
    \[
        = \underbrace{\det C \cdot \det C^{-1}}_{1} \cdot \det(A - tI) = \det(A - tI)
    \]
    \[
        \Rightarrow \chi_{CAC^{-1}} = \chi_A \Rightarrow \chi_f \text{ не зависит от выбора базиса}
    \]

    \subsection*{\S Диагонализуемые операторы}

    \par $V$ -- векторное поле над $K$
    \par $\dim V = n < \infty$
    \par $f \in \End(V)$

    \begin{definition}
        $f$ называется диагонализуемым, если $\exists$ базис $V$, такой что $[f]$ в этом базисе -- диагональная
    \end{definition}

    \begin{theorem*}[критерий диагонадизуемости]
        $ $
        \par $f$ -- диагонадизуем $\Leftrightarrow \exists$ базис $V$, состоящий из собственных векторов \\ оператора $f$
    \end{theorem*}

\end{document}