\documentclass[a4paper, 12pt]{article}

\usepackage[english, russian]{babel}
\usepackage[T2A]{fontenc}
\usepackage[utf8]{inputenc}
\usepackage{amsthm, amsmath, amsfonts, amssymb, mathtools}
\usepackage{geometry}
\usepackage{indentfirst}
\usepackage{titleps}
\usepackage{soulutf8}
\usepackage{multicol}
\usepackage{tabularx}
\usepackage{pgfplots}
\usepackage{cancel}
\usepackage{import}
\usepackage{xifthen}
\usepackage{pdfpages}
\usepackage{transparent}
\usepackage{wrapfig}
\usepackage{setspace}

\onehalfspacing

\pgfplotsset{compat=1.18, width=7cm}

\newpagestyle{main}{
    %\setheadrule{0.4pt}
    \sethead{}{}{}
    %\setfootrule{0.4pt}
    \setfoot{}{\thepage}{}
}

\pagestyle{main}

\theoremstyle{plain}
\newtheorem{theorem}{Теорема}
\newtheorem{corollary}{Следствие}
\newtheorem{lemma}{Лемма}[]
\newtheorem*{definition}{Определение}
\newtheorem*{remark}{Замечание}
\newtheorem{illustration}{Пример}
\newtheorem*{proposition}{Предложение}

\geometry{top=25mm}
\geometry{bottom=30mm}
\geometry{left=20mm}
\geometry{left=20mm}

\newcommand{\incfig}[1]{%
    \def\svgwidth{\columnwidth}
    \import{./figures/}{#1.pdf_tex}
}

\DeclareMathOperator{\Kerr}{Ker}
\DeclareMathOperator{\Imm}{Im}
\DeclareMathOperator{\Int}{Int}
\DeclareMathOperator{\Mat}{Mat}
\DeclareMathOperator{\rank}{rank}
\DeclareMathOperator{\diam}{diam}
\DeclareMathOperator*{\id}{id}
\renewcommand{\phi}{\varphi}
\renewcommand{\theta}{\vartheta}
\renewcommand{\epsilon}{\varepsilon}
\newcommand{\R}{\mathbb{R}}
\renewcommand{\C}{\mathbb{C}}
\newcommand{\Q}{\mathbb{Q}}
\newcommand{\N}{\mathbb{N}}
\setcounter{lemma}{9} % вот тут пофиксить
\DeclareMathOperator{\lrhimani}{\underset{\Pi}{\underline{\int}}}
\DeclareMathOperator{\urhimani}{\underset{\Pi}{\overline{\int}}}
\DeclareMathOperator{\rhimani}{\underset{\Pi}{\int}}

\begin{document}
    \title{Алгебра}
    \date{13 сентября 2022}
    \maketitle

    \pagebreak

    \subsection*{\S5 Алгебраически замкнутые поля}

    \par $K[x]$
    \par $x - a, \ a \in K$ всегда неприводимые

    \begin{illustration*}
        $x^2 + 1$ неприводим над $\R$
        \par \quad $x^3 + x^2 + 1, \ x^3 + x + 1, \ x^2 + x + 1$ неприводимы над $\mathbb{F}_2$
    \end{illustration*}

    \par Для многочленов степени $2$ и $3$ из отсутствия корней следует неприводимость

    \begin{theorem*}
        $K$ -- поле
        \par \quad Следующие условия эквивалентны:
        \begin{enumerate}
            \item Любой $f \in K[x], \ \deg f > 0$ имеет в $K$ корень
            \item $\forall f \in K[x]$ с $\deg f > 0$ число его корней в $K$ равно $\deg f$ с учетом кратности
            \item Любой $f \in K[x] \ \deg f > 0$ делится на какой-то линейный
            \item $\forall f \in K[x], \deg f > 0$ полностью раскладывается на произведение линейных сомножителей:
                \[
                    f = b \cdot \prod_{i = 1} (x_i- c_i)^{a_i}    
                \]
            \item Всякий неприводимый над $K$ линеен
        \end{enumerate}
    \end{theorem*}

    \begin{definition}
        Поле $K$, удовлетворяющее любому (а значит всем) из равносильных условий теоремы называется алгебраически замкнутым полем
    \end{definition}

    \begin{proof}
        $ $
        \begin{enumerate} % ЧЕТА Я ХЗ КАК СДЕЛАТЬ ТАК ЧТОБЫ ЭЛЕМЕНТЫ СПИСКА НОРМАЛЬНО ПОДВИНУЛИСЬ ВПРАВО
            \item[$1 \Leftarrow 2$] Если корней $\deg f$, то хотя бы $1$ есть
            \item[$1 \Leftrightarrow 3$] Теорема Безу
            \item[$4 \Rightarrow 3$] Раскладывается $\Rightarrow$ делится
            \item[$4 \Leftarrow 5$]
                \[
                    \begin{cases}
                        f \in K[x] \\
                        \deg f > 1 \\
                        f = \text{ произв. лин. сомножителей } > 1
                    \end{cases}    
                \]
            \item[$4 \Rightarrow 5$]
                \[
                    \begin{cases}
                        f = \text{ произв. непр. (ОТА)} \\
                        \text{всякий непр. -- линейный}
                    \end{cases}    
                \]
            \item[$3 \Rightarrow 4$] $f \in K[x], \ \deg f > 0$
                \par \quad Индукция по $\deg f$
                \par \quad \quad База: $\deg f = 1$ -- доказано
                \par \quad \quad Предположение: $\deg f > 1$
                \par \quad \quad \quad По п. 3: $f = (x - c) g(x)$
                \par \quad \quad \quad $\deg g = \deg f - 1$
                \par \quad \quad \quad По и. п. $g(x) = b \cdot \prod_{i=1} (x - c_i)^{a_i} = f \cdot \text{ пр-ие лин.}$
            \item[$1 \Rightarrow 2$] Индукция по $\deg f$
                \par \quad База: $\deg f = 1$ -- выполнено
                \par \quad Предположение: $\deg f > 1$ (по п. 1 у $f \ \ \exists$ корень $c$)
                \par это значит, что $x - c | f$
                \par $c$ -- корень $f$ кратности $a$
                \par $(x - c)^a | f$ \quad $(x - c)^{a + 1} \not | f$
                \par $f_{(x)} = (x - c)^a g(x)$ \quad $g(c) \not = 0$
                \par Если $g =$ const:
                \par \quad $\deg f = a$
                \par \quad $c$ -- единственный корень $f$ кратности $a$
                \par Если $\deg g > 0$:
                \par \quad то т. к. $\deg g \le \deg f - 1$
                \par \quad по и. п. число корней $g$ с учетом кратности .... (тут нада дописать дальше)
                \par Всякий корень $g$ -- корень $f$ (не меньшей кратности)
                \[
                    (x - d)^e | g \Rightarrow (x - d)^e | f    
                \]
                \par $d$ -- корень $f$, отличный от $c$, тогда $d$ -- корень $g$ не меньшей кратности
                \[
                    (x - d)^e | f \Rightarrow (x - d)^e | g    
                \]
                \par $d \not = c \Rightarrow x-d, \ x-c$ взаимно простые
                \par \quad \quad $(x-d)^e, \ (x-c)$ взаимно простые
                \par \quad \quad $(x-d)^e, \ (x-c)^e$ взаимно простые
                \par По теореме о сопряжении $(x-d)^e | g$
                \par \quad $c$ - не корень $f$
                \par Корни $f$, отличные от $c$ = корни $g$, причем той же кратности
                \par $c$ -- корень $f$ кратности $a$
                \par $\deg f = \deg g$ \quad (ну тут тоже нада дописать чета похоже)
        \end{enumerate}
    \end{proof}

    \begin{theorem}[Без доказательства]
        $ $
        \par Поле $\C$ -- алгебраически замкнутое
    \end{theorem}

    \textit{\textbf{Факты:}} (Без доказательства)
    \par \quad 1) Конечное поле сожержится в счетном алгебраически замкнутом поле
    \par \quad 2) Всякое поле содержится в каком-то алгебраически замкнутом

\end{document}