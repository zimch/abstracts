\documentclass[a4paper, 12pt]{article}

\usepackage[english, russian]{babel}
\usepackage[T2A]{fontenc}
\usepackage[utf8]{inputenc}
\usepackage{amsthm, amsmath, amsfonts, amssymb, mathtools}
\usepackage{geometry}
\usepackage{indentfirst}
\usepackage{titleps}
\usepackage{soulutf8}
\usepackage{multicol}
\usepackage{tabularx}
\usepackage{pgfplots}
\usepackage{cancel}
\usepackage{import}
\usepackage{xifthen}
\usepackage{pdfpages}
\usepackage{transparent}
\usepackage{setspace}
\usepackage{graphicx}
\usepackage{float}
\usepackage{wrapfig}
\usepackage{contour}
\usepackage{mathrsfs}

\onehalfspacing

\contourlength{1pt}

\pgfplotsset{compat=1.18, width=7cm}

\newpagestyle{main}{
    %\setheadrule{0.4pt}
    \sethead{}{}{}
    %\setfootrule{0.4pt}
    \setfoot{}{\thepage}{}
}

\pagestyle{main}

\theoremstyle{plain}
\newtheorem{theorem}{Теорема}
\newtheorem{corollary}{Следствие}
\newtheorem{lemma}{Лемма}[]
\newtheorem*{lemma*}{Лемма}
\newtheorem*{definition}{Определение}
\newtheorem*{remark}{Замечание}
\newtheorem{example}{Пример}
\newtheorem*{proposition}{Предложение}
\newtheorem*{theorem*}{Теорема}
\newtheorem*{example*}{Пример}
\newtheorem*{corollary*}{Следствие}

\geometry{top=25mm}
\geometry{bottom=30mm}
\geometry{left=20mm}
\geometry{left=20mm}

\newcommand{\incfig}[1]{%
    \def\svgwidth{\columnwidth}
    \import{./figures/}{#1.pdf_tex}
}

\graphicspath{ {./figures/} }

\DeclareMathOperator{\Kerr}{Ker}
\DeclareMathOperator{\Imm}{Im}
\DeclareMathOperator{\Int}{Int}
\DeclareMathOperator{\Mat}{Mat}
\DeclareMathOperator{\End}{End}
\DeclareMathOperator{\sign}{sign}
\DeclareMathOperator{\dist}{dist}
\DeclareMathOperator{\rank}{rank}
\DeclareMathOperator{\diam}{diam}
\DeclareMathOperator{\diag}{diag}
\DeclareMathOperator{\supp}{supp}
\DeclareMathOperator{\grad}{grad}
\DeclareMathOperator{\rot}{rot}
\DeclareMathOperator{\divv}{div}
\DeclareMathOperator{\Ext}{Ext}
\DeclareMathOperator{\Id}{id}
\DeclareMathOperator{\Char}{char}
%\DeclareMathOperator{\dist}{dist}
\DeclareMathOperator*{\id}{id}
\renewcommand{\phi}{\varphi}
\renewcommand{\theta}{\vartheta}
\renewcommand{\epsilon}{\varepsilon}
\newcommand{\R}{\mathbb{R}}
\renewcommand{\C}{\mathbb{C}}
\newcommand{\Q}{\mathbb{Q}}
\newcommand{\N}{\mathbb{N}}
\setcounter{lemma}{11} % вот тут пофиксить
\newcommand{\lrhimani}[1]{\underset{#1}{\underline{\int}}}
\newcommand{\urhimani}[1]{\underset{#1}{\overline{\int}}}
\newcommand{\rhimani}[1]{\underset{#1}{\int}}
\newcommand{\mycontour}[1]{\contour{red}{#1}}
\newcommand{\charf}[1]{\chi_{#1}(x)}
\newcommand{\pfrac}[2]{\frac{\partial #1}{\partial #2}}

\begin{document}
    \title{Математический анализ}
    \date{28 ноября 2022}
    \maketitle{}

    \pagebreak

    $L \quad \Lambda^pL \quad \lambda\Lambda\mu \in \bigwedge^{p+1}L \quad \bigoplus_{p=1}^\infty \Lambda^pL$

    \subsection*{Дифференциальные формы}

    \par Пусть $U \subset \R^n$ -- открыто
    \par $L = \R^n$, $L^* = \R^n$ -- сопряженное пространство

    \begin{definition}
        Дифференциальная $p$-форма -- это гладое отображение из $U$ в $\Lambda^pL$
        \[
            F^p(U) = C^\infty(Ul \Lambda^pL^*)    
        \]
        \[
            F^0 = C^\infty(U)    
        \]
        \[
            F^1(U) = \sum_{i=1}^n a_i(x)dx^i, \ a_i \in C^\infty(U)    
        \]
    \end{definition}

    \begin{illustration*}
        $f \in C^\infty(U)$
        \par $df(x) = \sum_{i=1}^n \frac{\partial f}{\partial x^i} dx^i$
        \par $df(x, h) = f'(x) h, \ f'(x) \in L^* $
        \par Производная -- вектор из $\R^n$ (сост. из частных пр-ых)
        \[
            df(x, h) = \sum_{i=1}^n \frac{\partial f}{\partial x^i}(x) h^i    
        \]
        \[
            dx^i(h) = h^i, \quad  dx^i \text{ -- функционал}    
        \]
        \[
            \frac{\partial a_i}{\partial x^j} = \frac{\partial a_j}{\partial x^i}, \quad \forall i, \ j \text{ -- неверно в общ. случ.}
        \]
        \par \quad \textbf{\textit{Пометка:}}\[
            L \ni a = \sum_{i=1}^n a^i e_i
        \]
    \end{illustration*}

    \par $ $
    \par Пусть теперь так:
    \[
        \omega \in F^p(U) \quad \omega = \sum_H a_H(x) dx^H \quad a_H \in C^\infty(U), \ \forall H   
    \]
    
    \begin{illustration*}
        $n = 3$, $\R^3$
        \par $0$-формы: $C^\infty(U)$
        \par $1$-формы: $p(x, y, z)dx + q(x, y, z)dy + r(x, y, z)dz$ \quad $p, q, r \in C^\infty(U)$
        \par $2$-формы: $a(x, y, z)dy \wedge dz + b(x, y, z)dz \wedge dx + c(x, y, z)dx \wedge dy$ \quad $a, b, c \in C^\infty(U)$
        \par $3$-формы: $g(x, y, z) \wedge dy \wedge dz$ \quad $g \in C^\infty(U)$
    \end{illustration*}

    \par Поточечные операции: $\omega + \eta$, \ $\omega \wedge \eta$, \ $f \cdot \omega $
    \par $\omega \in F^p(U) \text{ (тут дописать)}$ %ДОПИСАТЬ С ФОТО !!!%

    \[
        \alpha \wedge \lambda = \alpha \cdot \lambda, \ \alpha \in \Lambda^0P = \R, \ \lambda \in \Lambda^pL    
    \]

    \subsection*{Внешняя производная}
    
    \begin{illustration*}
        Операция $d$ (вн. произв.) действует из $F^p(U)$ в $F^{p+1}(U)$ для $p = 0, \dots, n-1$ по правилу:
        \[ % ВОТ ТУТ СДЕЛАТЬ СПИСОК МАТ ФОРМУЛ (ТИПА ЕНУМЕРЕЙТ)
            1.\ df(x) = \sum_{i=1}^n \frac{\partial f}{\partial x^i}(x) dx^i 
        \]
        \[
            2.\ d(\sum_H a_H(x) dx^H) = \sum_h(da_H(x)) \bigwedge dx^H
        \]
        \[
            d(a(x)dx^{h_1} \wedge \dots \wedge dx^{h_n}) = \sum_{i=1}^n \frac{\partial a}{\partial x^i}(x) dx^i \wedge dx^m \wedge \dots \wedge dx^{np}    
        \]
    \end{illustration*}

    \begin{illustration*}
        $n = 3$, $\R^n$
        \[
            df = \frac{\partial f}{\partial x}(x, y, z) dx^1 + \frac{\partial f}{\partial y}(x, y, z)dy + \frac{\partial f}{\partial z}(x, y, z)dz =    
        \]
        \[ 
            = (\grad f)_1(x, y, z)dx + (\grad f)_2(x, y, z)dy + (\grad f)_3(x, y, z) dz
        \]
        \par Производная $1$-формы:
        \[
            d(p(x, y, z)dx + q(x, y, z) dy + r(x, y, z) dz) = 
        \]
        \[
            = (\frac{\partial q}{\partial x} - \frac{\partial p}{\partial y})dx \wedge dy + (\frac{\partial r}{\partial y} - \frac{\partial q}{\partial z})dy\wedge dz + (\frac{\partial p}{\partial z} - \frac{\partial r}{\partial x})dz \wedge dx =  
        \]
        \[
            = (\rot F)_1 dy \wedge dz + (\rot F)_2 dz \wedge dx + (\rot F)_3 dx \wedge dy
        \]
        \[
            F(x, y, z) = \begin{pmatrix}
                p(x, y, z) \\
                q(x, y, z) \\
                r(x, y, z)
            \end{pmatrix}  
        \]
        \par Производная $2$-формы:
        \[
            d(a(x, y, z) dz \wedge dz + b(x, y, z) dz \wedge dx + c(x, y, z) dx \wedge dy) =    
        \]
        \[
            \left(\underbrace{\frac{\partial a}{\partial x}(x, y, z) + \frac{\partial b}{\partial y} + \frac{\partial c}{\partial z}(x, y, z)}_{= \divv G(x, y, z)}\right) dx \wedge dy \wedge dz =
        \]
        \[ % ЕТА ВИДИМО ДИВЕРГЕНЦИЯ
            G(x, y, z) = \begin{pmatrix}
                a(x, y, z) \\
                b(x, y, z) \\
                c(x, y, z)
            \end{pmatrix}
        \]
    \end{illustration*}

    \subsection*{Свойства внешней производной}

    \begin{enumerate}
        \item Аддитивность: $d(\omega \eta) = d \omega + d \eta$
        \item Действие на $0$-форму: $f \in f^0(U)$
            \[
                df = \sum_{i=1}^n \frac{\partial f}{\partial x^i}(x) dx^i
            \]
        \item $d(\lambda \wedge \mu) = (d\lambda) \wedge \mu + (-1)^{\deg \lambda} \lambda \wedge d \mu$
        \item Свойство Пуанкаре:
            \[
                d(\omega) = 0 \quad \lambda \in F^p(U): \ \deg \lambda = p    
            \]
    \end{enumerate}

    \begin{lemma*}
        Лемма Пуанкаре = свойство Пуанкаре
    \end{lemma*}

    \begin{proof}
        $ $
        \begin{enumerate}
            \item[3.] $\lambda = a(x) dx^H, \ \mu = b(x)dx^K$
            \[
                d(\lambda \wedge \mu) = d(a(x) b(x) \wedge dx^K) = \sum_{i=1}^n \frac{\partial (ab)}{\partial x^i} dx^i \wedge dx^H \wedge dx^K =  
            \]
            \[
                = \sum_{i = 1}^n a(x) \frac{\partial b}{\partial x^i} dx^i \wedge dx^H \wedge dx^K +    
            \]
            \[
                + \sum_{i=1}^n b(x) \frac{\partial a}{\partial x^i}(x) dx^i \wedge dx^H \wedge dx^K = d\lambda \wedge \mu f (-1)^p \lambda \wedge d\mu   
            \]
            \item[4.] $\omega = a(x) dx^H$
                \[
                    d\omega = \sum_{i=1}^n \frac{\partial a}{\partial x^i} dx^i \wedge dx^H   
                \]
                \[
                    d d \omega = \left(\sum_{j=1}^n\sum_{i=1}^n \frac{\partial^2 a}{\partial x^j \partial x^i}(x) dx^j dx^i \right) \wedge dx^H = 0    
                \]
        \end{enumerate}
    \end{proof}

    \begin{lemma*}
        Операция $d$ со свойствами $1-4$ единственна
    \end{lemma*}

    \begin{proof}
        Пусть $\tilde d$ -- такая операция ($1-4$)
        \par $\forall H$ \quad $\tilde d(dx^H) = 0$ ??
        \par Индукция: \quad $p=1$
        \[
            \tilde d (dx^i) = \tilde d(\tilde d x^i); \quad d(x^i) = dx^i = \tilde d (x^i) 
        \]
        \par Переход: \quad $(p-1) \rightarrow p$
        \[
            dx^H = \tilde d (x^{h_1} \cdot d x^{h_2} \wedge \dots \wedge dx^{h_p}) =
        \]
        \[
            = (\tilde d x^{h_1}) \wedge (dx^{h_2} \wedge \dots \wedge dx^{h_p}) + x^{h_1} \underbrace{\tilde d (dx^{h_2} \wedge \dots \wedge dx^{h_p})}_{= 0}    
        \]
        \[
            0 = \tilde d (d x^{h_1} \cdot dx^{h_2}\wedge \dots \wedge dx^{h_p}) = \tilde d (dx^H)
        \]
        \[
            \tilde d(\sum_H a_H(x) dx^H) = \sum_H \tilde d (a_H(x) dx^H) = \sum_H(\tilde d a_H(x) \wedge dx^H + a_H(x) \wedge \tilde d (d x^H))    
        \]
        \[
             = \sum_H \tilde d a_H(x) dx^H = \sum_H \sum_{i=1}^n \frac{\partial a_H(x)}{\partial x^i}(x) dx^i \wedge dx_H =
        \]
        \[
            d\left( \sum_H a_H(x)dx^H \right)    
        \]
        \[
            \Rightarrow \tilde d = d    
        \]
    \end{proof}

    \subsection*{Индуцированное отображение}

    \par $U \subset \R^n, \ V \subset \R^m$, \quad $x \in U$, $y \in V$
    \par $\Phi \in C^\infty(U, V)$
    \par $f \in C^\infty(V) = F^0(V)$
    \par $\Phi^* f = f \circ \Phi$ \quad $\Phi^* : F^*(V) \rightarrow F^0(U)$
    \[
        \Phi^*(dy^i) = \sum_{j=1}^n \frac{\partial Phi^i}{\partial x^j}(x) dx^j = d \Phi^i(x), \quad \Phi(x) = y
    \]
    \[
        H = (h_1, \dots, h_p) \quad \Phi^*(a_H(x) dx^{h_1} \wedge \dots \wedge dx^{h_p}) = (\Phi^*(a_H))(x) \Phi^*(dy^{h_1}) \wedge \dots \wedge \Phi^*(dy^{h_p})
    \]

    \begin{illustration*}
        $U; V \subset \R^2$
        \[
            \Phi^*(dy^1 \wedge dy^2) = \left( \sum_{i=1}^2 \frac{\partial \Phi^1}{\partial x^i}(x) dx^i \right) \wedge \left( \sum_{j=1}^2 \frac{\partial \Phi^2}{\partial x^j}(x) dx^j \right)    
        \]
        \[
            \sum_{i, j = 1}^2 \frac{\partial Phi^1}{\partial x^i} \frac{\partial \Phi^2}{\partial dx^j} dx^i dx^j = \left( \frac{\partial \Phi^1}{\partial x^2} \frac{\partial \Phi^2}{\partial x^1} \right) dx^1 \wedge dx^2 
        \]
        \[
            = \det \Phi'(x) dx^1 dx^2    
        \]
    \end{illustration*}

    \subsection*{Свойства индуцированного отображение}

    \begin{enumerate}
        \item $\Phi^*(\omega \eta) = \Phi^* \omega + \Phi^* \Phi^* \eta$
        \item $\Phi^*(\omega \wedge \eta) = (\Phi^* \omega) \wedge (\Phi^* \eta)$
        \item $\Phi^*(d\omega) = d \Phi^* \omega$
        \item $(\psi \circ \Phi)^* = \Phi^* \circ \psi^*$
    \end{enumerate}

    \begin{proof}
        $ $
        \begin{enumerate}
            \item[3.] По индукции
                \par база: $p=0$
                \[
                    \Phi^* df = d \Phi^* f, \quad f \in C^\infty(V)    
                \]
                \[
                    d(\Phi^* f) = d(f \circ \Phi) = \sum_{i=1}^n \frac{\partial (f \circ \Phi)}{\partial x^i} dx^i = 
                \]
                \[
                    = \sum_{i=1}^b \sum_{j=1}^m \frac{\partial f}{\partial y^j}(\Phi(x)) \frac{\partial \Phi^j}{\partial x^i} dx^i = \sum_{j=1}^m \Phi^* \frac{\partial f}{\partial y^j}\Phi^* dy^j =   
                \]
                \[
                    \Phi^* \left( \sum_{j=1}^m \frac{\partial f}{\partial y^j} dy^j \right) = \Phi^*(df)  
                \]
                \[
                    \text{... тут фото карочи ...}    
                \]
                \[
                    d \omega = d(g, d \eta) = dg \wedge d\eta    
                \]
                \[
                    = \Phi^*(dg) \wedge \Phi^*(d\eta) + \dots \text{ ватафак я тут нипон}    
                \]
            
            \item[4.] Достаточно доказать для $p=0$ и $p=1$
                \par $p=0$: $(\psi \circ \Phi)^* f = f \circ \psi \circ \Phi = (f \circ \psi) \circ \Phi = (\psi^* f) \circ \Phi = \Phi^* \psi^* f$
                \par $p=1$: $(\psi \circ \Phi)^* dz^i = $
                \[
                    = \sum_{i=1}^n \frac{\partial (\psi \circ \Phi)^j}{\partial x^i} dx^i    
                \]
                \par $\Phi^* \psi^* dz^j = $
                \[
                    \text{ блять я ничо не пон нужна фотка или конспект от руки}    
                \]
        \end{enumerate}
    \end{proof}

\end{document}