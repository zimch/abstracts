\documentclass[a4paper, 12pt]{article}

\usepackage[english, russian]{babel}
\usepackage[T2A]{fontenc}
\usepackage[utf8]{inputenc}
\usepackage{amsthm, amsmath, amsfonts, amssymb, mathtools}
\usepackage{geometry}
\usepackage{indentfirst}
\usepackage{titleps}
\usepackage{soulutf8}
\usepackage{multicol}
\usepackage{tabularx}
\usepackage{pgfplots}
\usepackage{cancel}
\usepackage{import}
\usepackage{xifthen}
\usepackage{pdfpages}
\usepackage{transparent}
\usepackage{setspace}
\usepackage{graphicx}
\usepackage{float}
\usepackage{wrapfig}
\usepackage{contour}
\usepackage{mathrsfs}

\onehalfspacing

\contourlength{1pt}

\pgfplotsset{compat=1.18, width=7cm}

\newpagestyle{main}{
    %\setheadrule{0.4pt}
    \sethead{}{}{}
    %\setfootrule{0.4pt}
    \setfoot{}{\thepage}{}
}

\pagestyle{main}

\theoremstyle{plain}
\newtheorem{theorem}{Теорема}
\newtheorem{corollary}{Следствие}
\newtheorem{lemma}{Лемма}[]
\newtheorem*{lemma*}{Лемма}
\newtheorem*{definition}{Определение}
\newtheorem*{remark}{Замечание}
\newtheorem{example}{Пример}
\newtheorem*{proposition}{Предложение}
\newtheorem*{theorem*}{Теорема}
\newtheorem*{example*}{Пример}
\newtheorem*{corollary*}{Следствие}

\geometry{top=25mm}
\geometry{bottom=30mm}
\geometry{left=20mm}
\geometry{left=20mm}

\newcommand{\incfig}[1]{%
    \def\svgwidth{\columnwidth}
    \import{./figures/}{#1.pdf_tex}
}

\graphicspath{ {./figures/} }

\DeclareMathOperator{\Kerr}{Ker}
\DeclareMathOperator{\Imm}{Im}
\DeclareMathOperator{\Int}{Int}
\DeclareMathOperator{\Mat}{Mat}
\DeclareMathOperator{\End}{End}
\DeclareMathOperator{\sign}{sign}
\DeclareMathOperator{\dist}{dist}
\DeclareMathOperator{\rank}{rank}
\DeclareMathOperator{\diam}{diam}
\DeclareMathOperator{\diag}{diag}
\DeclareMathOperator{\supp}{supp}
\DeclareMathOperator{\grad}{grad}
\DeclareMathOperator{\rot}{rot}
\DeclareMathOperator{\divv}{div}
\DeclareMathOperator{\Ext}{Ext}
\DeclareMathOperator{\Id}{id}
\DeclareMathOperator{\Char}{char}
%\DeclareMathOperator{\dist}{dist}
\DeclareMathOperator*{\id}{id}
\renewcommand{\phi}{\varphi}
\renewcommand{\theta}{\vartheta}
\renewcommand{\epsilon}{\varepsilon}
\newcommand{\R}{\mathbb{R}}
\renewcommand{\C}{\mathbb{C}}
\newcommand{\Q}{\mathbb{Q}}
\newcommand{\N}{\mathbb{N}}
\setcounter{lemma}{11} % вот тут пофиксить
\newcommand{\lrhimani}[1]{\underset{#1}{\underline{\int}}}
\newcommand{\urhimani}[1]{\underset{#1}{\overline{\int}}}
\newcommand{\rhimani}[1]{\underset{#1}{\int}}
\newcommand{\mycontour}[1]{\contour{red}{#1}}
\newcommand{\charf}[1]{\chi_{#1}(x)}
\newcommand{\pfrac}[2]{\frac{\partial #1}{\partial #2}}
\setcounter{lemma}{9}

\begin{document}

    % \title{Математический анализ}
    % \date{3 октября 2022}
    % \maketitle

    % \pagebreak

    % \subsection*{Характеристическая функция множества}
    {\noindent\large\textbf{Характеристическая функция множества}} \hfill \boxed{\textbf{03 октября 2022}}

    \begin{definition}
        Характеристическая функция множества 
        \begin{equation*}
            \chi_{E}(x) = 
             \begin{cases}
               1 & x \in E \\
               0 & x \not \in E
             \end{cases}
            \end{equation*}
    \end{definition}


    \begin{lemma}
        $\{\text{Точки разрыва} \ \chi_{E}(x)\} = \delta E$ 
    \end{lemma}

    \begin{proof}
       \par $x \in \Int E \cup \Ext E $
       \par На внутренних точках $\lim \chi_{E}(x) = 1 = \chi_{E}(x)$
       \par На внешних точках $\lim \chi_{E}(x) = 0 = \chi_{E}(x)$
       \bigskip
       \par На границе разрывна - очев

    \end{proof}

    \begin{definition}
        \par $E \in \Pi$, $\Pi$ - п/п, $f: E \rightarrow \R$ - ограничена
        \bigskip
        $$\rhimani E f = \rhimani \Pi f\cdot \chi_{E}$$
        \bigskip
        $$\widetilde f = f \cdot \charf{E}$$

    \end{definition}

    \begin{lemma}
        \par $\mu(\delta E) = 0$, $f: E \rightarrow \R$ - почти везде непрерывна на $E$
        \bigskip
        $$\text{Тогда }\exists \rhimani E f$$
    \end{lemma}

    \begin{proof}
        \par Необходимо доказать, что $\exists \rhimani \Pi \widetilde{f}$
        \par $\{\text{Точки разрыва} \ \widetilde{f} \ \text{на} \ \Pi \} = \{\text{Точки разрыва} \ \widetilde{f} \ \text{на} \ \Int E \} \cup \{\text{Точки разрыва} \ \widetilde{f} \ \text{на} \ \delta E \} \cup \underbrace{\{\text{Точки разрыва} \ \widetilde{f} \ \text{на} \ \Ext E \}}_{\varnothing}$
        \par $\{\text{Точки разрыва} \ \widetilde{f} \ \text{на} \ \Pi \} \subset \{\text{Точки разрыва} \ f \ \text{на} \ \Int E \} \cup \delta E$
        \par $\mu(\{\text{Точки разрыва} \ \widetilde{f} \ \text{на} \ \Pi \}) = 0$ - по критерию Лебега существует интеграл 
    \end{proof}

    \begin{definition}
        \par $E$ - ограничено, $\mu(\delta E) = 0$
        \par $\exists \Pi \in \R^n$ - п/п: $E \in \Pi$
        $$\upsilon(E) = \rhimani E 1 = \rhimani \Pi \charf{E} \textbf{ - Жорданов объём}$$ 

    \end{definition}


    \subsection*{Теорема Фубини}

    \begin{theorem}
        \par Теорема Фубини
        \par Пусть $\Pi = \Pi_1 \times \Pi_2, \Pi_1 \in \R^n, \Pi_2 \in \R^m, f: \Pi \rightarrow \R - \text{ограничена}$
        $$L_1(x) = \lrhimani{\Pi_2} f(x,y)dy$$
        $$U_1(x) = \urhimani{\Pi_2} f(x,y)dy$$
        $$ \text{Если } \ \exists \rhimani \Pi f, \text{то } \ \exists \rhimani{\Pi_1} L_1, \rhimani{\Pi_1} U_1 \ \text{и они равны}$$
        $$\rhimani{\Pi} f(x,y) dx dy = \rhimani{\Pi_1} L_1(x) dx = \rhimani{\Pi_1} U_1(x) dx$$
        $$\rhimani{\Pi_1} dx \Bigg(\lrhimani{\Pi_2} f(x,y) dy\Bigg) = \rhimani{\Pi_1} dx \Bigg(\urhimani{\Pi_2} f(x,y) dy\Bigg)$$
    \end{theorem}

    \begin{proof}
        \par Пусть $P_1$ - разбиение $\Pi_1$, $P_2$ - разбиение $\Pi_2$
        $$P = P_1 \times P_2 \ \text{- разбиение } \Pi$$ 
        $$P = \{\pi_1 \times \pi_2 \mid \pi_1 \in P_1, \pi_2 \in P_2\}$$
        $$L(f, P) = \sum\limits_{\pi \in P} \inf_{\pi} f \cdot v(\pi) = \sum\limits_{\pi_1 \in P_1, \pi_2 \in P_2} \inf_{\pi_1 \times \pi_2} f \cdot v(\pi_1) \cdot v(\pi_2) = \sum\limits_{\pi_1 \in P_1} \left[ \sum\limits_{\pi_2 \in P_2} \inf_{\pi_1 \times \pi_2} f \cdot v(\pi_2) \right] \cdot v(\pi_1)$$
        $$\forall x\in \pi_1: \inf_{\pi_1 \times \pi_2} f \le \inf_{y \in \pi_2} f(x, y)$$
        $$\sum\limits_{\pi_2 \in P_2} \inf_{\pi_1 \times \pi_2} f \cdot v(\pi_2) \le \sum\limits_{\pi_2 \in P_2} \inf_{y \in \pi_2} f(x,y) \cdot v(\pi_2) = L\big(f(x, \cdot), P_2\big) \le L_1 (x) - \text{ $L_1$ - sup таких сумм} $$
        $$\sum\limits_{\pi_2 \in P_2} \inf_{\pi_1 \times \pi_2} f \cdot v(\pi_2) \le \inf_{\pi_1} L_1(x) = m_{\pi_1} L_1$$
        $$L(f, P) \le \sum\limits_{\pi_1 \in P_1} m_{\pi_1} L_1 \cdot v(\pi_1) = L(L_1, P_1)$$
        $$U(f, P) \ge U(U_1, P) - \text{ аналогично}$$
        $$L(f, P) \le L(L_1, P_1) \le U(L_1, P_1) \le U(U_1, P_1) \le U(f, P) \text{ (*)} $$
        $$L(f, P) \le L(L_1, P_1) \le L(U_1, P_1) \le U(U_1, P_1) \le U(f, P) \text{ (**)}$$

        $$\exists \rhimani{\Pi} f \Rightarrow \sup_{P} L(f, P) = \inf_{P} U(f, P)$$
        $$\sup_{P_1} L(L_1, P_1) \le \inf_{P_1} U(L_1, P_1) - \text{Всегда}$$
        
        $$\text{(*): } \inf_{P_1} U(L_1, P_1) \le \inf_{P} U(f, P)$$
        $$\text{(*): } \sup_{P} L(f, P) \le \sup_{P_1} L(L_1, P_1)$$

        $$\sup_{P} L(f, P) \le \sup_{P_1} L(L_1, P_1) \le \inf_{P_1} U(L_1, P_1) \le \inf_{P} U(f, P) - \text{ из равенства 1 и 4 всюду равенство}$$

        $$\sup_{P_1} L(L_1, P_1) = \inf_{P_1} U(L_1, P_1)$$

        $$\exists \rhimani{\Pi_1} L_1 = \rhimani{\Pi} f$$

        $$\sup_{P_1} L(U_1, P_1) \le \inf_{P_1} U(U_1, P_1) - \text{Всегда}$$
        
        $$\text{(**): } \inf_{P_1} U(U_1, P_1) \le \inf_{P} U(f, P)$$
        $$\text{(**): } \sup_{P} L(f, P) \le \sup_{P_1} L(U_1, P_1)$$

        $$\sup_{P} L(f, P) \le \sup_{P_1} L(U_1, P_1) \le \inf_{P_1} U(U_1, P_1) \le \inf_{P} U(f, P) - \text{ из равенства 1 и 4 всюду равенство}$$

        $$\sup_{P_1} L(U_1, P_1) = \inf_{P_1} U(U_1, P_1)$$

        $$\exists \rhimani{\Pi_1} U_1 = \rhimani{\Pi} f$$



    \end{proof}

\end{document}