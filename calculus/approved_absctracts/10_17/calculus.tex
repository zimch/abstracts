\documentclass[a4paper, 12pt]{article}

\usepackage[english, russian]{babel}
\usepackage[T2A]{fontenc}
\usepackage[utf8]{inputenc}
\usepackage{amsthm, amsmath, amsfonts, amssymb, mathtools}
\usepackage{geometry}
\usepackage{indentfirst}
\usepackage{titleps}
\usepackage{soulutf8}
\usepackage{multicol}
\usepackage{tabularx}
\usepackage{pgfplots}
\usepackage{cancel}
\usepackage{import}
\usepackage{xifthen}
\usepackage{pdfpages}
\usepackage{transparent}
\usepackage{wrapfig}
\usepackage{setspace}

\onehalfspacing

\pgfplotsset{compat=1.18, width=7cm}

\newpagestyle{main}{
    %\setheadrule{0.4pt}
    \sethead{}{}{}
    %\setfootrule{0.4pt}
    \setfoot{}{\thepage}{}
}

\pagestyle{main}

\theoremstyle{plain}
\newtheorem{theorem}{Теорема}
\newtheorem{corollary}{Следствие}
\newtheorem{lemma}{Лемма}[]
\newtheorem*{definition}{Определение}
\newtheorem*{remark}{Замечание}
\newtheorem{illustration}{Пример}
\newtheorem*{proposition}{Предложение}

\geometry{top=25mm}
\geometry{bottom=30mm}
\geometry{left=20mm}
\geometry{left=20mm}

\newcommand{\incfig}[1]{%
    \def\svgwidth{\columnwidth}
    \import{./figures/}{#1.pdf_tex}
}

\DeclareMathOperator{\Kerr}{Ker}
\DeclareMathOperator{\Imm}{Im}
\DeclareMathOperator{\Int}{Int}
\DeclareMathOperator{\Mat}{Mat}
\DeclareMathOperator{\rank}{rank}
\DeclareMathOperator{\diam}{diam}
\DeclareMathOperator*{\id}{id}
\renewcommand{\phi}{\varphi}
\renewcommand{\theta}{\vartheta}
\renewcommand{\epsilon}{\varepsilon}
\newcommand{\R}{\mathbb{R}}
\renewcommand{\C}{\mathbb{C}}
\newcommand{\Q}{\mathbb{Q}}
\newcommand{\N}{\mathbb{N}}
\setcounter{lemma}{9} % вот тут пофиксить
\DeclareMathOperator{\lrhimani}{\underset{\Pi}{\underline{\int}}}
\DeclareMathOperator{\urhimani}{\underset{\Pi}{\overline{\int}}}
\DeclareMathOperator{\rhimani}{\underset{\Pi}{\int}}

\setcounter{lemma}{15}

\begin{document}
    % \title{Математический анализ}
    % \date{17 октября 2022}
    % \maketitle

    % \pagebreak
    \null\hfill \boxed{\textbf{17 октября 2022}}

    $G \subset \R^n$ открытое ограниченное, $f : G \rightarrow \R$ огр. и п. в. непрерывно
    \par $\supp f \subset G \Rightarrow \exists \rhimani{G} f$

    \begin{theorem}
        $G \subset \R^n$ открытое ограниченное, $g : G \rightarrow \R^n$ диффеоморфизм,
        \par \quad $g(G)$ огр.,
        \par \quad $f : g(G) \rightarrow \R$ огр., п. в. непрерывно, $\supp f \subset g(G)$
        \par Тогда
        \[
            \exists \rhimani{G} f \circ g |\det g'| = \rhimani{g(G)} f% ДОМНОЖЕННОЕ??    
        \]
    \end{theorem}

    \begin{lemma} % тут нужно \overline сделать длинными%
        Пусть $G \subset \R^n$ открыто, $g : G \rightarrow \R^n$ гомеоморфизм
        \par $E \subset G : \overline E \subset G$. Тогда
        \par $g(\overline E) = \overline {g(E)}$
        \par $g(\Int) = \Int g(E)$
        \par $g(G \setminus \overline E) = g(G) \setminus \overline {g(E)}$
        \par $g(\delta E) = \delta g(E)$
        \par Если $G, \ g(G)$ огр., $g$ -- диффеоморфизм, то
        \[
            \mu(E) = 0 \Leftrightarrow \mu\big(g(E)\big) = 0    
        \]
    \end{lemma}
    \begin{proof}
        \[
            G \ni x_n \xrightarrow[n \rightarrow \infty]{} x \in G \Leftrightarrow g(G) \ni g(x_k) \xrightarrow[n \rightarrow \infty]{} g(x) \in g(G)   
        \]
        \[
            g(x) \in g(\overline E) \Leftrightarrow x \in \overline E \Leftrightarrow g(x) \in \overline{g(E)}   
        \]
        \[
            x \in \Int E \Leftrightarrow \exists \epsilon > 0 \ B_\epsilon(x) \subset E \Leftrightarrow \exists \delta > 0 : B_\delta(g(x)) \subset g(E) \Leftrightarrow   
        \]
        \[
            \Leftrightarrow g(x) \in \Int g(E)    
        \]
        \[
            \begin{rcases}
                x \in \delta E \Leftrightarrow E \ni y_n \xrightarrow[n \rightarrow \infty]{} x \\
                G \setminus \ni x_n \xrightarrow[n \rightarrow \infty]{} x
            \end{rcases} \Leftrightarrow \begin{cases}
                g(E) \ni g(y_n) \xrightarrow[n \rightarrow \infty]{} g(x) \\
                g(G \setminus E) \ni g(z_n) \xrightarrow[n \rightarrow \infty]{} g(x)
            \end{cases} \Leftrightarrow g(x) \in \delta g(E)
        \]
        \[
            x \in G \setminus \overline E = \Int (G \setminus E) \Leftrightarrow g(x) \in \Int \big(g(G) \setminus g(E)\big) = g (G) \setminus \overline{g(E)}    
        \]
        $ $
        \par $\forall \epsilon \ \exists C_l, \ l = 1, \dots, N$ -- открытые кубы
        \[
            E \subset \bigcup_{l = 1}^N C_l, \quad \sum_{l=1}^N v(C_l) < \epsilon  
        \]
        \[
            l(C) \text{ -- длина ребра куба } C \quad v(C) = (l(C))^n \quad \diam C = l(C) \sqrt n    
        \]
        $ $
        \par  Если $\epsilon < {\frac{\delta}{2\sqrt n}}^n$, $\forall l \quad v(C_l) < \epsilon$ $\Rightarrow l(C_1) < \frac{\delta}{2\sqrt n}$
        \par $\diam C_l < \frac{\delta}{2}$ \quad $\dist (\overline E, \delta G) = \delta$

        \[
            \bigcup_{l=1}^N C_l \subset E^{\frac{\delta}{2}} = \{x \in \R^n \ | \ \dist (x, E) < \frac{\delta}{2}\} \subset \overline {E^{\frac{\delta}{2}}} \subset G  
        \]
        \[
            \forall x_1, x_2 \in E^{\frac{\delta}{2}} \ \|g(x_1) - g(x_2)\| \le M \cdot \|x_1 - x_2\| \quad \max_{\overline{E^{\frac{\delta}{2}}}} \|g'\| = M   
        \]
        \[
            g(C_l) \subset B_{\frac{M \cdot \diam C_l}{2}} (g(x_l)) \subset \tilde C_l, \ l(\tilde C_l) = M \cdot \diam C_l    
        \]
        \[
            v(\tilde C_l) = (l(\tilde C_l))^n = M^n \cdot (\diam C_l)^n = M^n \Big(\frac{\diam C_l}{\sqrt{n}}\Big)^n (\sqrt{n})^n = (M\sqrt{n})^n v (C_l)    
        \]
        \[
            g(E) \subset \bigcup_{l=1}^N \tilde C_l, \quad \sum_{l=1}^N \tilde C_l = (M\sqrt{n})^n \cdot \epsilon   
        \]
    \end{proof}

    \begin{remark}
        В условие теоремы $\exists \rhimani{G} f \circ g |\det g'|$
    \end{remark}
    % ТУТ ПЕРЕДЕЛАТЬ КРАСИВО
    \begin{proof}
        $\supp f = \overline{\{y \in g(G) \ | \ f(y) \not = 0\}}$
        \par $\supp (f \circ g |\det g'|) = \supp f \circ g = \overline{\{x \in G \ | \ (f \circ g) (x) \not= 0\}}$
        \par $g(\{x \in G \ | \ f \circ g \not = 0\}) = \{y \in g(G) \ | \ f(y) \not = 0\}$
        \par $\xRightarrow[\text{л. 16}]{}$ замыкание совпалает
        \par $\supp (f \circ g |\det g'|)$ компактен
        \par $\sup_{\supp (f \circ g |\det g'|)} |\det g'| < \infty \Rightarrow f \circ g \cdot |\det g'|$ огр.
        \par $f$ п. в. непрерывно на $g(G) \xRightarrow[\text{л. 16}]{} f \circ g$ п. в. непрерывно на $G$
        \par $|\det g'| \in C(G) \Rightarrow f \circ g |\det g '|$ п. в. непрерывно на $G$
        \par Значит, $\exists \rhimani{G} f \circ g |\det g'|$
    \end{proof}

    \begin{lemma} % ЛЕММА 13
        $G \subset \R^n$ открыто, $g : G \rightarrow \R$ диффеоморфизм
        \par $\forall x \in G \ \exists$ окрестность $U \subset G$
        \[
            g\big|_U = g_1 \circ \dots \circ g_n    
        \]
        \par где $g_k$ -- простейший диффеоморфизм, т. е.
        \[
            ((g_k)(x))_i = x_i, \quad \forall i \not= k, \quad i \text{ -- координата}  
        \]
    \end{lemma}
    \begin{proof}
        Индукция
        \par База $k=1$ : уже простейший
        \par Переход \quad $\big(g(x)\big)_i = x_i , \quad i \ge k+1 \quad x = \big(y = (x_1, \dotsc, x_k), z\big)$ % y, z - vectors (photo)
        \par $ $
        \par Пусть $x_0 \in G \quad g'(x_0) = \begin{pmatrix}
            \frac{\partial g}{\partial y} & \frac{\partial g}{\partial z} \\
            0 & I_{n-k}
        \end{pmatrix}$
        \par $0 \not= \det g'(x_0) = \det \frac{\partial g}{\partial y} (x_O) \Rightarrow$ не все миноры порядка $k-1$ нулевые
        \par \quad перенумерацией компонентов добъемся того, чтобы главный минор был $\not= 0$
        \[
            f : G \rightarrow \R^n \quad (f(x))_i = \begin{cases}
                (g(x))_i, \ i < k \\
                x_i, \ i \ge k
            \end{cases}    
        \]
        \[
            f'(x_0) = \begin{pmatrix}
                \left(\frac{\partial g}{\partial y}\right) & \left(\frac{\partial g}{\partial z}\right)_{k-1} \\
                & I_{n-k+1}
            \end{pmatrix}    
        \]
        \[
            \det f'(x_0) = \left(\frac{\partial g}{\partial y}\right)_{k-1} \not= 0    
        \]

        \par $f \in C^1(G) \quad \exists окрестность U \ni x_0 \quad f\big|_{U} \text{-- диффеоморфизм}$
        \par Положим $h = g \circ (f\big|_{U})^{-1}$ -- диффеоморфизм $f(U) \rightarrow g(U)$
        \[
            g\big|_U = h \circ f\big|_U
        \]
        \par Для $f \ \exists$ окрестность $x_0$, в которой $f$ раскладывается в композицию простейших
        \par $u \in f(U)$
        \[
            i < k \quad (h(u))_i = (g \circ f^{-1}(u))_i = (f \circ f^{-1}(u))_i = u_1
        \]
        \[
            i \ge k \quad (h(u))_i = (g \circ f^{-1}(u))_i = (f^{-1}(u))_i = u_i    
        \]
        \[
            (\underbrace{f(x)}_{=u})_i = x_i = (f^{-1}(u))_i, \quad i > k    
        \]
    \end{proof}

    \begin{lemma}
        Утверждение теоремы верно при $n = 1$
        \begin{lemma}[14']
            В условиях теоремы на $G$ и на $g$ при $n=1$ для $\forall f : g(G) \rightarrow \R$ огр. : $\supp f \subset g(G)$
            \[
                \lrhimani{G} f \circ g |\det g'| = \lrhimani{g(G)} f, \urhimani{G} f \circ g |\det g'| = \urhimani{g(G)} f    
            \]
        \end{lemma}
    \end{lemma}

    \begin{proof}
        % ОБЩЕЕ ДОКАЗАТЕЛЬСТВО
        $\supp f$ компактен
        \[
            \forall x \in \supp f \ \exists \epsilon_x > 0 \quad [  x - \epsilon, x + \epsilon] \subset g(G)    
        \]
        \[
            \supp f \subset \bigcup_{x \in \supp f} (x - \epsilon_x, x + \epsilon_x) \Rightarrow \supp f \subset \bigcup_{i=1}^N (x_i - \epsilon_{x_i}, x_i + \epsilon_{x_i})  
        \]
        \[
            \supp f \subset \bigcup_{i=1}^N [x_i - \epsilon_{x_i}, x_i + \epsilon_{x_i}] \quad \text{отрезки не пересек}   
        \]
        \[
            \forall i \quad (g^{-1})'\big|_{[x_i - \epsilon_{x_i}, x_i + \epsilon_{x_i}]} \ \text{ имеет постоянный знак}    
        \]
        \[
            g^{-1}([x_i - \epsilon_{x_i}, x_i + \epsilon_{x_i}]) \text{ -- отрезок} \Rightarrow \text{ достаточно доказать:}    
        \]
        \[
            \rhimani{g([a, b])} g = \rhimani{[a, b]} f \circ g |\det g'|
        \]
        \[
            g' > 0 \quad \int_{g(a)}^{g(b)} f(y)dy = \int_a^b f\big(g(x)\big)g'(x)dx    
        \]
        \[
            g' < 0 \quad \int_{g(b)}^{g(a)} f(y)dy = \int_a^b f\big(g(x)\big) |g'(x)|dx
        \]
        \textit{(кусок далее надо сделать красиво)}

        % ТУТ Я УСНУЛ
        $P_y$ - разбиение $g([a,b]) \quad P_x = g^{-1}(P_y)'' = \{ g^{-1}(\pi) | \pi \in P_y\}$

        $\min_{g[a,b]} | (g^{-1})'| \cdot d(P_y) \le d(P_x) \le \max_{g([a,b])} | (g^{-1})'| \cdot d(P_y)$
        
        $\sum_{\pi_y \in P_y}\sup_{\pi_y}f \cdot v(\pi_y) = 
        \sum_{\pi_x \in P_x}\sup_{\pi_x} (f \circ g) |g'(\xi(\pi_x))| \cdot v(\pi_x)$

        $\pi_y = g(\pi_x)$   

        $v(\pi_y) = |g(\beta) - g(\alpha)| = |g'(\xi)| \cdot (\beta\alpha)$

        прадалжаем неравенство (емае неправильное все стёр....) $\displaystyle\le \underbrace{\sup_{[a,b]}|g'|}_{<\infty} \sum_{\pi_x \in P_x}\sup_{\Pi_x}f\circ g \cdot v(\pi_x)$

        $P_{x,k}:d(P_{x,k}) \to 0, k \to \infty$

        $P_{y,k}:d(P_{y,k}) \to 0, k \to \infty$

        $\forall \pi_x \exists \xi(\pi_x) < \pi_x \quad v(\pi_y) = |g'(\xi(\pi_x))|\cdot v(\pi_x)$
        
        $U(f, P_{y,k}) \le \sum_{\pi_x \in P_x}\sup_{\pi_x}f\circ g |g'| \cdot v(\pi_x)
        = U(f\circ g |g'|, P_{x,k})
        $

        слева $\overline\int_{g([a,b])}f \le$ справа $\int_{[a,b]}f\circ g |\det g' |$

        $\overline\int_{g^{-1}g([a,b])}f\circ g |g'| \le \overline\int_{g([a,b])}f\circ g
        \circ g^{-1} | g' \circ g^{-1}|\cdot |g^{-1}|$
        
        $\implies \overline\int_{g([a,b])}f = \overline\int_{[a,b]}f\circ g |g'|$
    \end{proof}

    \begin{proof}{(Теоремы)}
        $ $
        \begin{enumerate}
            \item Простейший диффеоморфизм $(g(x))_i = x_i , \ i < n$
                \[
                    g(G) = \Pi_y = \Pi_1 \times \Pi_{y_n} \quad \Pi_1 \subset \R^{n-1}, \Pi_{y_n} \subset \R    
                \]
                \[
                    G = \Pi_x = \Pi_1 \times \Pi_{x_n} \quad \Pi_{x_n} \subset \R    
                \]
                \[
                    \rhimani{g(G)} = \rhimani{\Pi_y} f \chi_{g(G)} \overset{\text{Фубини}}{=} \rhimani{\Pi_1} dy_1 \dots dy_{n-1} \lrhimani{\Pi_{y_n}} dy_n f \chi_{g(G)} = 
                \]
                \[
                    = \rhimani{\Pi_1} dy_1 \dots dy_{n-1} \lrhimani{\pi_n[g(G) \cap (y_1, \dots, y_{n-1}) \times \R]} f(y)dy_n \overset{\text{Лемма 14 (мб 16)}}{=} 
                \]
                \[
                    = \rhimani{\Pi_1}dx_1 \dots dx_{n-1} \lrhimani{\pi_n[G \cap (x_1, \dots, x_{n-1}) \times \R]} (f \circ g)(x) \underbrace{\left|\frac{\partial g_n}{\partial x_n} (x)\right|}_{=|\det g'|}dx_n =    
                \]
                \[
                    = \rhimani{\Pi_1} dx_1 \dots dx_{n-1} \lrhimani{\Pi_{x_n}} f \circ g |\det g'| \chi_G  =   
                \]
                \[
                    = \rhimani{\Pi_x} f \circ g |\det g'| \chi_G  = \rhimani{G} f \circ g |\det g'|   
                \]
        \end{enumerate} % это лишь первый случай.....
    \end{proof}

\end{document}