\documentclass[10pt, a4paper]{article}
\usepackage[utf8]{inputenc}
\usepackage[T2A]{fontenc}
\usepackage[russian]{babel}

\usepackage{amsfonts, amssymb, amsmath}  %% for math symbs
\usepackage{mathrsfs}
\usepackage{float}  %% for table floating
\usepackage{enumerate} %% for lists

\usepackage{fullpage}  %% less margin 

\usepackage{graphicx} %% for pics


\begin{document}
    % \tableofcontents
    % \section*{Диффэрэншл икуэйжнс тэст.}
    \subsection*{1}
    \begin{enumerate}
        \item Сформулируйте достаточное условие асимптотической устойчивости
            нулевого решения системы линейных дифференциальных уравнений
            1-ого порядка с постоянными коэффициентами.
        
        \item Что такое функция Грина краевой задачи для системы линейных 
            дифференциальных уравнений 1-ого порядка? Как её найти?
        \item Напишите формулу частного решения неоднородной линейной системы 
            дифференциальных уравнений для произвольной неоднородности.
        \item Могут ли траектории решений двумерной системы уходить на 
            бесконечность и при $t\to -\infty$, и при $t\to +\infty$? 
            При каком условии?
    \end{enumerate}

    \subsection*{2}
    \begin{enumerate}
        \item Напишите формулу Лиувилля для системы линейных дифференциальных 
            уравнений и для линейного дифференциального уравнения 
            $n$-го порядка.
        \item Оцените $x(100)$, если известно, что $X(t)$ - решение системы
            $$\begin{cases}
                X'_t= AX,\\
                X(0) = B,
            \end{cases}$$
            все собственные числа матрицы $A$ принадлежат отрезку $[-5;3]$, а норма
            вектора $B$ равна единице.
        \item Что такое вронскиан? В каких случаях он обращается в нуль?
        \item Могут ли траектории решений линейной системы дифференциальных уравнений 
        с постоянными коэффициентами не пересекаться никогда?  
    \end{enumerate}

    \subsection*{3}
    \begin{enumerate}
        \item Сформулируйте лемму Грануолла.
        \item Посчитайте $e^A$, где $$A = \begin{pmatrix}
            3 & 1 & 0 \\ 0 & 3 & 1 \\ 0 & 0 & 3
        \end{pmatrix}$$
        \item Напишите формулу частного решения неоднородного линейного 
            дифференциального уравнения $n$-го порядка.
        \item Могут ли траектории решений линейной системы дифференциальных уравнений
            с постоянными коэффициентами пересекаться в одной точке, 
            отвечающей конечному значению переменной?
    \end{enumerate}

    \subsection*{4}
    \begin{enumerate}
        \item Сформулируйте теорему Пикара существования и единственности 
            решения задачи Коши.
        \item Для произвольной матричной нормы $\|\cdot\|$ напишите вид оценки
            нормы матричной экспоненты $\|e^A\|$ с помощью нормы матрицы $\|A\|$.
            Как будет выглядеть эта оценка, если в качестве нормы $\|\cdot\|$ взять
            спектральную норму?
        \item Что такое мультипликатор?
        \item Могут ли траектории решений линейной системы дифференциальных уравнений
            быть ограниченными кривыми (кривыми конечной длины)? 
            Если да, то при каких условиях на систему?
    \end{enumerate}

\end{document}