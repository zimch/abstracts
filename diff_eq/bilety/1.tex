\documentclass[a4paper, 12pt]{article}

\usepackage[english, russian]{babel}
\usepackage[T2A]{fontenc}
\usepackage[utf8]{inputenc}
\usepackage{amsthm, amsmath, amsfonts, amssymb, mathtools}
\usepackage{geometry}
\usepackage{indentfirst}
\usepackage{titleps}
\usepackage{soulutf8}
\usepackage{multicol}
\usepackage{tabularx}
\usepackage{pgfplots}
\usepackage{cancel}
\usepackage{import}
\usepackage{xifthen}
\usepackage{pdfpages}
\usepackage{transparent}
\usepackage{setspace}
\usepackage{graphicx}
\usepackage{float}
\usepackage{wrapfig}
\usepackage{contour}
\usepackage{mathrsfs}

\onehalfspacing

\contourlength{1pt}

\pgfplotsset{compat=1.18, width=7cm}

\newpagestyle{main}{
    %\setheadrule{0.4pt}
    \sethead{}{}{}
    %\setfootrule{0.4pt}
    \setfoot{}{\thepage}{}
}

\pagestyle{main}

\theoremstyle{plain}
\newtheorem{theorem}{Теорема}
\newtheorem{corollary}{Следствие}
\newtheorem{lemma}{Лемма}[]
\newtheorem*{lemma*}{Лемма}
\newtheorem*{definition}{Определение}
\newtheorem*{remark}{Замечание}
\newtheorem{example}{Пример}
\newtheorem*{proposition}{Предложение}
\newtheorem*{theorem*}{Теорема}
\newtheorem*{example*}{Пример}
\newtheorem*{corollary*}{Следствие}

\geometry{top=25mm}
\geometry{bottom=30mm}
\geometry{left=20mm}
\geometry{left=20mm}

\newcommand{\incfig}[1]{%
    \def\svgwidth{\columnwidth}
    \import{./figures/}{#1.pdf_tex}
}

\graphicspath{ {./figures/} }

\DeclareMathOperator{\Kerr}{Ker}
\DeclareMathOperator{\Imm}{Im}
\DeclareMathOperator{\Int}{Int}
\DeclareMathOperator{\Mat}{Mat}
\DeclareMathOperator{\End}{End}
\DeclareMathOperator{\sign}{sign}
\DeclareMathOperator{\dist}{dist}
\DeclareMathOperator{\rank}{rank}
\DeclareMathOperator{\diam}{diam}
\DeclareMathOperator{\diag}{diag}
\DeclareMathOperator{\supp}{supp}
\DeclareMathOperator{\grad}{grad}
\DeclareMathOperator{\rot}{rot}
\DeclareMathOperator{\divv}{div}
\DeclareMathOperator{\Ext}{Ext}
\DeclareMathOperator{\Id}{id}
\DeclareMathOperator{\Char}{char}
%\DeclareMathOperator{\dist}{dist}
\DeclareMathOperator*{\id}{id}
\renewcommand{\phi}{\varphi}
\renewcommand{\theta}{\vartheta}
\renewcommand{\epsilon}{\varepsilon}
\newcommand{\R}{\mathbb{R}}
\renewcommand{\C}{\mathbb{C}}
\newcommand{\Q}{\mathbb{Q}}
\newcommand{\N}{\mathbb{N}}
\setcounter{lemma}{11} % вот тут пофиксить
\newcommand{\lrhimani}[1]{\underset{#1}{\underline{\int}}}
\newcommand{\urhimani}[1]{\underset{#1}{\overline{\int}}}
\newcommand{\rhimani}[1]{\underset{#1}{\int}}
\newcommand{\mycontour}[1]{\contour{red}{#1}}
\newcommand{\charf}[1]{\chi_{#1}(x)}
\newcommand{\pfrac}[2]{\frac{\partial #1}{\partial #2}}
\parindent 0px

\begin{document}
    \textbf{1.} Каноническая форма дифференциального уравнения $n$-го порядка 
    (системы дифференциальных уравнений первого порядка). 

    Постановка задачи Коши для дифференциального уравнения n-го 
    порядка и для системы дифференциальных уравнений
    первого порядка, дуализм этих задач.
    
    \rule{\textwidth}{1px}
    
    \par $ $
    $$F(x, y(x), y'(x), \dotsc, y^{(n)}(x)) = 0 \quad\textbf{ -- общий вид}$$
    $\displaystyle\frac{\partial F}{\partial y^{(n)}} \neq 0$, 
    $n$ -- порядок дифференциального уравнения. 
    
    \,\\
    По теореме о неявной функции: $y^{(n)} = F(x, y, \dotsc, y^{(n-1)})$
    \textbf{-- канонический вид}

    % $y^{(n)} = f(x, y, \dotsc, y^{(n-1)})$

    \,\\
    $
    \begin{cases}
        y(x_0) = y_0 \\
        y'(x_0) = y_1 \\
        \dotsc \\
        y^{(n-1)(x_0) = y_{n-1}}
    \end{cases}
    \iff
    \begin{cases}
        u'_1=u_2 \\
        u_2'=u_3 \\
        \dotsc \\
        u'_{n-1} = u_n \\
        u'_n = F(x, \vec{u})
    \end{cases}
    \overset{\overset{\text{замена переменных}}{\downarrow}}{u(x)}= 
    \begin{cases}
        y(x) \\
        y'(x) \\
        \vdots \\
        y^{(n-1)}(x)
    \end{cases}
    $

    \,\\
    Каждое решение уравнения 1 системы переходит с помощью замены 
    в решение второй системы.

    Задача Коши.
    $\sqsupset y^{(n)} = F(x, y, \dotsc, y^{(n-1)}) \qquad 
    \begin{matrix*}[l]
        t \in D \subset \mathbb{R} \\ y_0, \dotsc, y_{n-1} \in \mathbb{R}
    \end{matrix*}$
    
    Тогда 
    $\begin{cases}
        y(x_0) = y_0 \\
        \dot{y}(x_0) = y_1 \\
        \dotsc \\
        y^{(n-1)}(x_0) = y_{n-1}
    \end{cases}$ имеет решение в $\varepsilon$-окрестности точки $x_0$
\end{document}