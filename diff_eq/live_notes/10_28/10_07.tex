\documentclass[12pt, a4paper]{article}
\usepackage[utf8]{inputenc}
\usepackage[T2A]{fontenc}
\usepackage[russian]{babel}

\usepackage{amsfonts, amssymb, amsmath}  %% for math symbs
\usepackage{mathrsfs}
\usepackage{float}  %% for table floating
\usepackage{enumerate} %% for lists

\usepackage{fullpage}  %% less margin 

\usepackage{graphicx} %% for pics
\usepackage{amsthm, amsmath, amsfonts, amssymb, mathtools}

\parindent 0px  % no white space in new lines
% \baselinestretch{1.5}
\renewcommand{\baselinestretch}{1.3} 

%% titling
\title{Hello world} 
\author{Lindy2076}
\date{22.22.2} %%\today

\begin{document}
    я опоздал на минуту, не знаю о чем речь

    $\dot{\vec{x}} = P(t)\vec{x} \mapsto \dot{\vec{y}} = R\vec{y} \quad \exists G(t) : \det G(t) \neq 0, G(t+w) = G(t)$

    $\vec{x}(t) = G(t)\vec{y}(t)$

    А когда у нас появляются периодические решения, когда у нас неоднородное уравнение?

    $\dot{\vec{x}} = P(t)\vec{x} + \vec{q}(t), \quad P(t+w) = P(t), \vec{q}(t+w) = q(t)$

    $? \varphi(t)$ - периодическое решение
    
    \par $ $

    \textbf{T1.} $\dot{\vec{\varphi}}(t) = P(t)\vec{\varphi}(t) + q(t),\quad \varphi(t+w) = \varphi(t) \iff \vec\varphi(0) = \vec{\varphi}(w)$

    D.$\Rightarrow$ очевидно

    $\Leftarrow$ Построим задачу Коши.

    $\begin{cases}
        \dot{\vec{\varphi}}(t) = P(t) \vec{\varphi}(t) + q(t) &t \mapsto t+w\\
        \vec{\varphi}(0) = \varphi_0
    \end{cases}$

    $\begin{cases}
        \dot{\vec{\varphi}}(t+w) = P(t+w)\vec{\varphi}(t+w) + \vec{q}(t+w) = P(t)\vec{\varphi}(t+w) = \vec{q}(t)\\
        \vec{\varphi}(w) = \vec{\varphi}(0) = \varphi_0
    \end{cases}$

    $\implies \exists!$ решение задачи Коши $\implies \vec{\varphi}(t+w) = \vec{\varphi}(t)$

    \par $ $
    
    \textbf{Т.2 Достаточное условие разрешимости}

    Если $\mu_1\neq 0, \dotsc, \mu_n \neq 1$ - мультипликаторы $\implies \exists! 
    \text{ решение } \varphi(t) = \varphi(t+w)=\varphi(t)$

    D. Выберем фундаметальную матрицу однородной системы, с помощью которой мы выделяли матрицу МонодромиИ

    $\widetilde{\Phi}(t): \begin{cases}
        \dot{\widetilde{\Phi}} = P(t)\widetilde{\Phi} \\
        \widetilde{\Phi}(0) = I
    \end{cases}$

    $U(t, \tau) = \widetilde{\Phi}(t) = \widetilde{\Phi}^{-1}(\tau)$

    $\begin{cases}
        \dot{\vec{\varphi}} = P(t)\vec{\varphi} + q(t) \\
        \vec{\varphi}(0) = \vec{\varphi}(w) = \vec{\varphi}_0
    \end{cases} \iff \vec{\varphi}(t) |\underset{ t=w}{ } = 
    \widetilde{\Phi}(t)\vec{\varphi}_0 + \int^t_0\widetilde{\Phi}(t)\widetilde{\Phi}^{-1}(\tau)\vec{q}(\tau)dtd\tau$

    $\vec\varphi_0 = \vec{\varphi}(w) = \underbrace{\widetilde{\Phi}(w)}_{=B}\vec{\varphi_0} +
    \int^w_0\widetilde{\Phi}(w)\widetilde{\Phi}^{-1}(\tau)\vec{q}(\tau)dtd\tau$

    $(I - \widetilde{\Phi}(w))\vec{\varphi}_0 = \int^w_0 \widetilde{\Phi}(w)\widetilde{\Phi}^{-1}(\tau)\vec{q}(\tau)d\tau$

    \textit{$\exists(I - \widetilde{\Phi}(w))^{-1} \iff \det(I-B)\neq 0$} % hz

    \par $ $

    \textbf{Сл.} $P(t) = P \equiv const, \vec{q}(t+w) = q(t) \implies \exists!\vec{\varphi}$ периодич.

    Если $\forall \lambda_j$ - собственные числа $P: \lambda_j \neq \frac{i2\pi k}{w}$ - отсутствие резонанса

    D. $\widetilde{\Phi}(t) = e^{tp}, \quad \vec{\varphi}_0 = (I - e^{wp})^{-1}$
    $\int^w_0 e^{(w-\tau)p}q(\tau)d\tau$ - дост. усл.

    $e^{w\lambda_j}\neq 1$ - тоже самое что и в условии

    $\vec{\varphi}(t) = e^{tp}(I - e^{wp})^{-1}\int^w_0 e^{(w-\tau)p}\vec{q}\tau d\tau + \int^t_0 e^{(t-\tau)p}q(t)d\tau$
    $\\= e^{tp}( (I - e^wp)^{-1} \int^w_0 e^{(w-\tau)p}\vec{q}(\tau)d\tau + \int^t_0 e^{-\tau p}q(\tau)d\tau )$

    $(I - e^{wp})\vec{\varphi}(t) = e^{tp}(\int^w_0 e^{(w-\tau)p}q(\tau)d\tau + \int^t_0 e^{-\tau p}q(\tau) - \int^t_0 e^{(w-\tau)p}q(\tau)d\tau) =$

    $= e^{tp}(\int_0^t e^{-\tau p}\vec{q}(\tau)d\tau - \int^t_w e^{(w-\tau)p}\vec{q}(\tau)d\tau ) = e^{tp} \int^t_{t-w} e^{-\tau p}q(\tau)d\tau$% замена tau = w + \widetilde tau \int_0^{t-w} e^{-\tau p} q(\tau)d\tau

    $\boxed{\vec\varphi(t) = (1 - e^{wp})^{-1}\int^t_{t-w} e^{(t-\tau)p}q(\tau)d\tau}$

    \par $ $

    \par $ $
    
    {\large\textbf{Краевая задача}}
    
    Есть отрезок $[a,b]$
    
    $\begin{cases}
        \dot{\vec{x}} = P(t)\vec{x}  & P(t):[a,b]\to \mathbb{C}^{n\times n}\\
        A\vec{x}(\alpha) + B\vec{x}(\beta) = 0 & \alpha, \beta \in [a,b]
    \end{cases}$

    $?\exists \vec{x}(t) \not\equiv 0$

    $\begin{cases}
        \dot\Phi(t) = P(t)\Phi(t) \\
        \Phi(\alpha) = I
    \end{cases}\vec{x} = \Phi(t)\vec{C} \implies (A + B\Phi(\beta))\vec{C} = 0
    \\\exists \vec{c}\not\equiv 0 \iff \det(A+B\Phi(\beta)) = 0$ 

    Иначе  пусть $\det(A + B\Phi(\beta)) \neq 0$

    \textbf{Функция Грина}

    Нарисуем картинку, двумерная плоскость, альфа и бета симметрично расположены на обеих осях, образуют квадрат,
    у него есть диагональ. Фунцкия грина:

    $G(t, \tau): t\in [\alpha, \beta], \tau \in [\alpha, \beta], t\neq \tau$

    1. $\frac{d}{dt} G(t, \tau) = P(t) G(t, \tau)$

    2. $AG(\alpha, \tau) + BG(\beta, \tau) = 0, \forall \tau \in (\alpha, \beta)$

    3. $G(\tau + 0, \tau) - G(\tau - 0, \tau) = I$

    \par $ $

    1. $G(t, \tau) = \begin{cases}
        \Phi(t)S(\tau) & t \in [\alpha, \tau)\\
        \Phi(t)T(\tau) & t \in (\tau, \beta]
    \end{cases}$

    2. $AS(\tau) + B\Phi(\beta)T(\tau) = 0$

    3. $\Phi(\tau)S(\tau) - \Phi(\tau)T(\tau) = - I$

    слау 2 3 

    $\Phi^{-1}(\tau) \cdot \mid \implies
    \begin{cases}
        AS(\tau) + B\Phi(\beta)T(\tau) = 0 \implies A(\Phi^{-1}(\tau) + T(\tau)) + B\Phi(\beta)T(\tau) = 0\\
        S(\tau) - T(\tau) = \Phi^{-1}(\tau) \implies S(\tau) = T(\tau) + \Phi^{-1}(\tau)
    \end{cases}$

    $\implies (A+ B\Phi(\beta))T(\tau) = -A\Phi^{-1}(\tau)$ домножим на обратное к левой части без Т
    
    $\implies T(\tau) = -(A + B\Phi(\beta))^{-1}A\Phi^{-1}(\tau)$

    $= -(A + B\Phi(\beta))^{-1}(A \pm B\Phi(\beta))\Phi^{-1}(\tau) = 
    -(A + B\Phi(\beta))^{-1}((A + B\Phi(\beta)) - B\Phi(\beta))\Phi^{-1}(\tau)$

    $= -(I - (A + B\Phi(\beta))^{-1}B\Phi(\beta))\Phi^{-1}(\tau)$

    Ответ: $G(t, \tau) = \begin{cases}
        -\Phi(t)(A + B\Phi(\beta))^{-1}B\Phi(\beta)\Phi^{-1}(\tau) & t \in [\alpha, \tau)\\
        \Phi(t)(I - (A + B\Phi(\beta))^{-1}B\Phi(\beta))\Phi^{-1}(\tau) & t \in (\tau, \beta]
    \end{cases}$

    \par $ $

    {\large\textbf{Неоднородная краевая задача}}

    $\begin{cases}
        \dot{\vec{x}} = P(t)\vec{x} + q(t)\\
        A\vec{x}(\alpha) + B\vec{x}(\beta) = 0
    \end{cases}$

    \textbf{Т.} Пусть $\Phi(\alpha) = I \land \det(A + B\Phi(\beta))\neq 0$

    $\implies \exists ! $ решение краевой задачи $\vec{x}(t) = \int^\beta_\alpha G(t, \tau)\vec{q}(\tau)d\tau$

    D. Разобъем наше решение в сумму двух инт

    $\vec{x}(t) = \int^t_\alpha G(t, \tau)q(\tau)d\tau + \int^\beta_t G(t, \tau)q(\tau)d\tau$

    дифф по иксу

    $\dot{\vec{x}}(t) = (G(t, \tau - 0) - G(t, t+0))q(t) + P(t)\int_\alpha^t G(t, \tau)q(\tau)d\tau + P(t)\int\int_t^\beta G(t, \tau)q(\tau)d\tau= 
    Iq(t)  + P(T)\int^\beta_\alpha G(t, \tau)q(\tau) d\tau = P(t)\vec{x} + q(t)$

    $A\vec{x}(\alpha) + Bx(\beta) = \int^\beta_\alpha (AG(\alpha, \tau) + BG(\beta, \tau))\vec{q}(\tau)d\tau = 0$

    Существенность

    Пусть есть два реш

    $x_1, x_2$

    $\vec{\varphi} = \vec{x_1} - \vec{x_2}$

    $\implies 
    \begin{cases}
        \dot{\vec\varphi} = P(t)\vec{\varphi}\\
        A\vec{\varphi}(\alpha) + B\vec{\varphi}(\beta) = 0
        
    \end{cases}\implies\vec{\varphi} \equiv 0$

    
    
\end{document}