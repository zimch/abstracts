\documentclass[12pt, a4paper]{article}
\usepackage[utf8]{inputenc}
\usepackage[T2A]{fontenc}
\usepackage[russian]{babel}

\usepackage{amsfonts, amssymb, amsmath}  %% for math symbs
\usepackage{mathrsfs}
\usepackage{float}  %% for table floating
\usepackage{enumerate} %% for lists

\usepackage{fullpage}  %% less margin 

\usepackage{graphicx} %% for pics
\usepackage{hyperref} %% for links

\parindent 10px  % no white space in new lines

\begin{document}

Ж. формы нужны чтобы написать ответ в дифф уравнении для фунд. матрицы

Ф с точкой = АФ, А - константа. Ф = е в степени Аt

а как посчитать эту экспоненту? идея в жордановых формах

Ликбез: жордановы формы:\\

Есть матрица $n\times n$ из $\mathbb{C}$. У матрицы есть собственные числа, чтобы их найти надо построить характеристический полином, которые есть ничто иное как $\det (A - \lambda I) = 0 \implies $ корни. 

существует преобразование $A = S^{-1}JS$, где $J$ - набор жордановых клеток.

$J = \left(
\begin{matrix}
J_{p} 
\end{matrix}
\right)
$

$J_1 = \lambda, J_2 = \left(\begin{matrix}
\lambda & 1 \\ 0 & \lambda
\end{matrix}\right), J_k = \left(\begin{matrix}
\lambda & 1 & 0 & \cdots & 0 \\ 
0 & \lambda & 1 & 0 & 0\\
\vdots & 0 & \lambda & 1 & 0\\
0 & \dotsc & 0 & \lambda & 1 \\
0 & \cdots & \cdots & 0 & \lambda
\end{matrix}\right)$ (на побочной диагонали 1, на основной собственное число, в других местах 0)\\

$\lambda_1, \dotsc, \lambda_m$

рассмотрим $\lambda$ -- собственное число $A$,\\
 $rank(A-\lambda I) < n$. ядро непустно, следовательно ранк не может равняться $n$.
 
утв. если матрица $A=A^+$ самосопряженная, то все $\lambda_j \in \mathbb{R}$, $\lambda_j \neq \lambda_k$. короче лямбды ортогональны.

почему? рассм $
\lambda_j(\vec{x_{\lambda_j}}, \vec{x_{\lambda_k}}) = (A\vec{x_{\lambda_j}}, \vec{x_{\lambda_k}}) = (A \vec{x^{(j)}}, \vec{x^{(k)}}) = (\vec{x^{(j)}}, A^+ \vec{x^{(k)}}) = \lambda_k (\vec{x_{\lambda_j}}, \vec{x_{\lambda_k}})
$ 
извините там короче какая-то муть.

короче всегда если есть самосопряженный оператор и все собственные числа разные то они ортогональны. работает только для самосопряженных операторов! там нет блоков, тупа собственные числа стоят

сл. $A = A^+, \quad \lambda_j, j = 1\dotsc n$ собственные числа кратности 1 $\implies \{ \vec{x^{(j)}}\}, j=\overline{1,n}$ образует базис в $\mathbb{C}^n$.

$\exists S $ -- унитарное, а в вещественном ортагональное, т.ч. $S^+AS = \Lambda = \left(\begin{matrix}
\lambda_1 & 0 \\ 0 & \lambda_n
\end{matrix}\right) =\implies A = S\Lambda S^+
$

утв2. $\lambda_j, j=\overline{1, k}$ - подл собств знач $A$. 

индукция по $k$, при $k=1$ утверждение очевидно. 

индуктивный переход. от противно. пусть они могут быть л.н.. 

$\alpha_1 \vec{x^{(1)}} + \dotsc \alpha_k \vec{x^{(k)}} = 0$ домножим на $\lambda_k, \alpha_1 \neq 0$

$\alpha_1 \lambda_k \vec{x^{(1)}} + \dotsc \alpha_k \lambda_k \vec{x^{(k)}} = 0$. 

чета под воздействием $A$ лямбды превращаются в л1, л2...лн. и там короче какая-то муть. ну короче такое невозможно.

СЛ. собственные вектора матрицы образуют базис. следовательно матрица имеет диагональную форму. $S^{-1}AS = \Lambda \implies A = S\Lambda S^{-1}$. такое называется нормальной матрицей. собственные числа комплексные.\\

что делать когда у матрицы ровно $k<n$ различных собственных значений? поскольку их меньше n то может возникнуть ситуация когда отвечающих им векторов не хватает для построения базиса. а где остальные брать? маленькое замечание: если матрица самосопряженная, то собственных вектором хватит.

ето тот случай когда алгебраическая кратность совпадает с геометрической кратностью. 

пусть у произвольной матрцы имеется $k< n$ л.н. собственных векторов, а больше сосбственных векторов нет, отвечающих собственным значениям лямбда1...лямбдаК геометрической кратности р1...рК, р1+...+рК=н. тогда найдётся набор собственных векторов $\{ \vec{e^{(1)}}_1, \dotsc, \vec{e^{(1)}}_{r_1}, \vec{e^{(2)}}_1, \dotsc, \vec{e^{(2)}}_{r_2}, \dotsc, \vec{e^{(k)}}_1, \vec{e^{(k)}}_{r_k} \}$\\

$A \vec{e_1^{(j)}} = \lambda_j \vec{e^{(j)}}_1, \quad A \vec{e^{(j)}}_2 = \vec{e^{(j)}}_1 + \lambda_1 \vec{e^{(j)}}_2, \dotsc, A \vec{e^{(j)}}_{r_j} = \vec{e^{(j)}}_{r-1} + \lambda_j \vec{e^{(j)}}$\\

чета присоеденённые вектора

$\{e_1^{(j)}, \dotsc, e_{r_j}^{(j)} \}$ образует базис в пространстве однородного линейного уравнения. $(A-\lambda I)\vec{x} = 0$, $r_j$ -- геом кратность, т.е. количество векторов в базисе пространства решения уравнения.\\

утв3. если собственные значения разные то разумеется сосбственные подпространства решений $(A - \lambda I )\vec{x} = 0$ и $(A - \lambda_k I)\vec{x} = 0$ пересекаются только в 0.

от противного. пусть не так. пусть существует вектор икс, такой что он одновременно лежит в первом и втором пространстве. но тогда можно просто взять и вычесть два уравнения. получится, что матрицы А сократятся и останутся $(\lambda_k - \lambda_j)\vec{x} = 0$, что не возможно, т.к. собственные различны.

в обратную сторону: введем следующее определение. назовём е ётый второй присоедененным вектором первого порядка, е ётый третий присоедененным вектором второго порядка и так далее вплоть до е ётого р катого присоедененным вектором р к минус первого морядка.\\
опр. икс - присоедененный вектор первого порядка, отвечающий собственному значению лямбда, тогда и только тогда, когда если на него подействовать оператором А минус лямбда И то получится игрек - сосбвенный вектор. $(A- \lambda I)\vec{x} = \vec{y} \neq 0$,

$(A - \lambda I) \vec{y} = 0$

рассмотрим пространство решений уравнения (а минус лябмда И) в квадрате вектор икс равно нулю - эль два. вообще говоря имеет место включение эль 1 в эль 2. 

в эль 2 есть векторы двух типов. те, кто лежит в эль 1 и те кто не лежит в  эль 1, но лежат в эль 2. вот они нас и интересуют.

если у нас есть два соотношения $(A - \lambda I)\vec{x} \neq 0$,\\$(A - \lambda I)^2\vec{x} = 0$ то вектор икс - присоедененный вектор первого порядка. 

Для катого порядка $(A- \lambda I)^{k-1}\vec{x} \neq 0$, \\
$(A - \lambda I)^k \vec{x} = 0$ то вектор икс присоедененный вектор к минус первого порядка.

эль 1 лежит в эль 2 лежит в ... лежит в эль к (без равенства), а дальше эль к лежит (а на самом деле равно) эль к+1 и т.д.

так как размерность цэ эн равно конечному н то существует к когда эль к равно эль к + 1. в крайнем случае к станет равно н. а иначе вообще говоря к меньше н.

отсюда вывод:\\
утв5.
если лямбда -- собственное значение, то существует максимальная эль к -- макс, которое является ядром оператора $(A - \lambda I)^k$. при этом выполняется свойство: $(A - \lambda I)L^{(k-1)} \subset L^{(k)}$\\

лямба - собств знач. цэ эн равно эль 1 прямая сумма Д. если икс лежит в эль 1 то А икс тоже лежит в эль. но если икс  лежит в Д то и А икс лежит в Д. при этом можно взять сужение А на эль, которое будет иметь лишь одно собственное значение лямбда. А сужение А на Д не имеет собственного значения лямбда.

лябмда равна нулю. шаг 1 пусть нет присоедененных векторов. значит не существует ненулевого решения. есть только собственные векторы. не существует вектора икс такого что А икс равно нулю и А квадрат икс равно 0. пусть эль равно ядру А, а Д - такие вектор-игреки, которые получаются так: игрек = А вектор икс, где икс из цэ эн.

поскольку размерность ядра А равно к то размерность Д равно н минус к по теореме о рангах.

от противного. пусть не так. пусть есть ненулевой игрек, такой что игрек в Д и игрек в эль. 
если игрек в эль, то А игрек равно нулю. если игрек в Д, то это значит что игрек равно А вектор икс. А квадрат икс равно 0 следовательно икс - присоедененный вектор первого порядка, а таких векторов не существует по первому шагу.\\

шаг 2. пусть есть присоедененные вектора. на каком-то этапе рост подпространства остановится. пусть эль пэ максимально нерасширяемое подпростнатсво. эль пэ равно ядру А в степени пэ, Д пэ равно игреки, которые равны А в степени пэ на вектор икс, при иксах их цэ эн. отсюда всё цэ эн - прямая сумма эль пэ и дэ пэ. это верно тогда и только тогда когда эль пэ в пересечении с дэ пэ равно нулю. от противного игрек не равно нулю, игрек из дэ пэ и из эль пэ. тогда из дэ игрек равняется А в степени пэ вектор икс, из эль игрек равен А игрек = 0, А в степени пэ минус 1 вектор икс равен 0. противоречие. 

сл. если лябмда 1 собственное значение цэ эн равен эль пэ один один плюс Д пэ один один, лябмда 1, ... лямбда к - соственные значения алг кратности пэ1...пэК, отсюда цэ эн = эль пэ один один плюс эль пэ два два плюс ... плюс эль пэ к к.

сл2. для любого йод существует пэ йод такое что (А минус лямбда И) в степени п йод икс равно 0 для любого икс из эль пэ 1. 

опр. базисом пространства эль относительно его подпространства эль с волной мы назовём набор векторов е1...е эль, которые лежат в эль и которые после пополнения базисом эль с волной, которые есть е1 с волной...е эль с волной, получим объединение двух наборов, который будет базисом эль. 

в эль пэ базис относительно эль пэ минус один е1...е ку -- присоединенные вектора пэ минус первого порядка


\end{document}