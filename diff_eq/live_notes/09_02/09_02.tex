\documentclass[12pt, a4paper]{article}
\usepackage[utf8]{inputenc}
\usepackage[T2A]{fontenc}
\usepackage[russian]{babel}

\usepackage{amsfonts, amssymb, amsmath}  %% for math symbs
\usepackage{mathrsfs}
\usepackage{float}  %% for table floating
\usepackage{enumerate} %% for lists

\usepackage{fullpage}  %% less margin 

\usepackage{graphicx} %% for pics

\parindent 0px  % no white space in new lines

%% titling
\title{Hello world} 
\author{Lindy2076}
\date{22.22.2} %%\today

\begin{document}
\section{начало}

	Диффура (ОДУ) - обыкновенные диффиренциальные уравнения
	
	$$F(x, y(x), y'(x), y''(x), \dotsc, y^{(n)}(x)) = 0$$ - обычное дифф уравнение $n$-ного порядка
	
	$$\frac{\partial F}{\partial y^{(n)}} \neq 0$$

Это был общий вид.\\
Канонический вид:
$$y^{(n)} = f(x,y,y',\dotsc,y^{(n-1)})$$

второй закон ньютона: 

$\begin{cases}
m\vec{a} &= \vec{F} \\
m\vec{X}''(t) &= \vec{F}(t, \vec{x}(t), \vec{x}'(t))
\end{cases}$
это типичное дифф. уравнение

Задача Коши:



\textbf{Теорема Пикара:} $\exists!$ решения задачи Коши
$$\begin{cases}
	y' = f(x,y) \text{ чета пусто непон}\\
	y(x_0) = y_0
\end{cases}$$
$ y = \varphi(x)$ - ищем такое решение 

$(x_0, y_0) \in Int(X, Y)$\\
1) $f \in C(\overline{X}, \overline{Y}), X, Y$ - области \\
2) Функция Липшицева по y, равномерна по $x \in \overline{X}$, если

$\exists L>0: | f(x,y_1) - f(x,y_2)| \le L |y_1-y_2|$\\

\textit{Доказательство:}
Возьмём интрегал от задачи Коши:
$$\int^{x}_{x_0} y'(x) dt = \int^{x}_{x_0} f(x, y(x))dt \implies \\ y(x)-y(x_0) = \int^x_{x_0} f(t, y(t))dt \implies y(x) = y_0 + \int^x_{x_0}f(t,y(t))dt$$

Теперь решим после стрелочки вправо:

$$\{\varphi_k\}^\infty_{k=0}$$
$$
\begin{align}
\varphi_0 =& y_0 \\
\varphi_1 =& y_0 + \int^x_{x_0}f(t, \varphi_0(t))dt \\
\dotsc \\
\varphi_m =& y_0 + \int^x_{x_0}f(t, \varphi_{m-1}(t))dt \\
\dotsc \\
\end{align}$$
$$ \varphi_k \in C({\overline X \subset \overline K}), K - \text{компакт, } \varphi_k \overset{k}{\rightrightarrows } \varphi, k\to\infty
$$

$$\varphi_0(x) + \big(\varphi_1(x) - \varphi_0(x)\big) + \big(\varphi_2(x) - \varphi_1(x)\big) + \dotsc = ? = \varphi_0 + \sum^\infty_{k=1}(\varphi_k - \varphi_{k-1}) \le \text{ суммируемая мажоранта}
$$
чета про веерштраса и мажоранты
посчитаем модуль разницы

$$
\big|\varphi_m(x) - \varphi_{m-1}(x)\big| \le \int^x_{x_0} \big|f\big(t, \varphi_{m-1}(t)\big) - f\big(t, \varphi_{m-2}(t)\big)\big|dt \le^{\text{липшецевость} }
$$

$$
x, x_0 \in K - \text{ компакт}
$$
$$
\le L \int^x_{x_0} \big|\varphi_{m-1}(t) - \varphi_{m-2}(t) \big|dt \le 
$$

по индукции предполгаем, что этот модуль не превосходит ...

$$
M = \max_{x,y \in(\overline{X}, \overline{Y})} |f(x,y)|
$$
$$
	\max_{x\in K} \big| \varphi_m(x) - \varphi_{m-1}(x)\big| \le \frac{ML^{m-1}|x-x_0|^m}{m!}
$$
продолжение дела...
$$
\le L \int^x_{x_0}\big| \varphi_m(x) - \varphi_{m-1}(x)\big|dt \le \frac{ML^{m-2}L}{(m-1)!}\int^x_{x_0}|t-x_0|^{m-1}dt \le \frac{ML^{m-1}|x-x_0|^m}{m!}
$$

База $|\varphi_1 - \varphi_0| \le \int^x_{x_0} |f(t, y_0)|dt$

$$
\varphi_0 + \sum^\infty_{k=1}(\varphi_k - \varphi_{k-1}) \le \frac{M}{L}\sum^\infty_{m=1} \frac{L^m |x-x_0|^m}{m!} = |\varphi_0| + \frac{M}{L} (e^{L|x-x_0|} - 1)
$$

$$
\varphi_k(x) = y_0 + \int^x_{x_0} f\big(t, \varphi_{k-1}(t)\big)dt \le M|x-x_0|, k \to \infty
$$
равномерно перделим $\varphi_k(x)$ к $\varphi(x)$:
$$
\varphi(x) = y_0 + \int^x_{x_0} f\big(t, \varphi(t)\big)dt
$$
чета по теореме Бэроу про переменный верхний пердел
показали существование короче. 

$y(x) = y_0 + \int^x_{x_0}f\big(t, y(t)\big)dt \mid \cdot \frac{d}{dx}$

$\begin{cases}
	y' = f\big(x, y(t)\big) \\ y(x_0) = y_0
\end{cases}$

осталась единственность.
Положим, что есть два решения. $\varphi$  и $\psi$.
$$
\psi(x) = y_0 + \int^x_{x_0} f\big(t, \psi(t)\big)dt
\varphi(x) = y_0 + \int^x_{x_0} f\big(t, \varphi(t)\big)dt
$$

$\displaystyle 
|\psi(x) - \varphi(x)| \\
\quad\qquad \upuparrows\\
|\psi(x) - \varphi_m(x)| \le \int^x_{x_0} \Big|f\big(t, \psi(t)\big) - f\big(t, \varphi_{m-1}(t)\big)\Big|dt 
\\\le L \int^x_{x_0}|\psi - \varphi_{m-1}|dt = M\frac{L^{m-1}|x-x_0|^m}{m!} \to 0 \\
\text{база: } |\psi(x) - \varphi_0| \le \int^x_{x_0} \big| f\big(t, \psi(t)\big)dt \big| \le M|x-x_0| \to 0, m \to \infty
 $ \\

ип: $| \psi - \varphi_m| \le \frac{L^m M |x-x_0|^{m+1}}{(m+1)!}$

$\implies \psi = \varphi$

докзали на компакте $K \subset U_\varepsilon(x_0)$. Он входит в некую окрестность точки $x_0$.\\
// $M$ связно, если для любых двух открытых множеств $G_1, G_2$ и $M \in G_1 \cup G_2$ и $G_1 \cap G_2 \neq \varnothing$ 

\textbf{Следствие 1:}\\ 
Пусть $\overline{x} = [x_0-a, x_0+a]$, $\overline{y} = [y_0-b, y_0+b]$ - отрезки.\\
$\exists$ пикаровский интервал $[x_0 - h, x_0 + h]$, $\varphi $ - решение задачи Коши: \\
$\forall x \in [x_0 +h, x_0-h], h = \min\{a, 
\frac{b}{M}\}$ \\
$\varphi(x) = y_0 + \int^x_{x_0}f\big(t, \varphi(t)\big)dt$\\

$|\varphi(x) - y_0| \le M|x-x_0| \le b$\\\\

\textbf{Следствие 2:}

$
\begin{cases}
\vec{y'}(x) = \vec{f}\big(x, \vec{y}(x)\big)\\
\vec{y}(x_0) = \vec{y}^{(0)}

\end{cases}$ \\

$\vec{y}, \vec{f} \in \mathbb{R}^n$ \\

$1.) \vec{f} \in C(\overline{X}, \overline{Y})\\
 2.) || \vec{f}(x, \vec{y}^{(1)} - \vec{f}(x, \vec{y}^{(2)}) || \le L|| \vec{y}^{(1)} - \vec{y}^{(2)} ||$\\\\

\textbf{Следствие 3:}\\
$$\frac{\partial f}{\partial y}(x, y) \in C(\overline{Y}) \implies L = \max_{K \subset Y} \Big|\frac{\delta f}{\delta y}\Big|$$\\\\
\textbf{Следствие 4:}\\
пусть условие липшецевости выполнено независимо\\
% вообще ниже три палочки мб описка хз

$\forall a >0, \|\vec{y}\| < \infty \ \exists L_a = L$\\

$||f(x, \vec{y}^{(1)}) - f(x, \vec{y}^{(2)})|| \le L \| \vec{y}^{(1)} - \vec{y}^{(2)} \| \implies \exists! \vec{\varphi}(x)$, определенная на $[x_0 -a, x_0 + a]$\\

Если же $L_a = L(a) \quad \forall \vec{y} \in \mathbb{R}$, то вообще говоря не всякое решение продолжимо, даже на отрезке $[x_0-a, x_0+a]$ \\


Непрерывная зависимость решения задачи Коши
% че бл y_m нихкиx m \implies y (x) m m 

$
\begin{cases}
y' = f(x, y_1) \\
y(x_0) = m

\end{cases}
$\\
$ z = y - m \implies
\begin{cases}
z' = f(x, z+m) \\ z(x_0) = 0
\end{cases}
 = f(x,z,m)
$ 

\textbf{Теорема:}\\
$
\begin{cases}
y' = f(x, y, m)\\ y(x_0) = y_0
\end{cases}
$\\

1. $f(x,y,m) \in C(\overline{D}), \overline{D} = \{ |x-x_0| \le y, |y-y_0| \le b, |m - m_0 \le c| \} $\\
%липшевость чета
2. $|f(x,y,m) - f(x,y_2,m)| \le L|y_1 - y_2|$\\
$\implies [x_0 - h, x_0 + h), h_1 = \min\{a, \frac{b}{m}\}$\\
$\exists!\varphi_m(x) \in C_m[m_0-c, m_0+c]$\\

\begin{list}{•}{Литература:}
\item Курс обыкновенных диффуров. Петровский
\item Чето тоже про дифуры. Эльцгольц
\end{list}
 
 
\end{document}