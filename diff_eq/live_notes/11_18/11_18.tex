\documentclass[12pt, a4paper]{article}
\usepackage[utf8]{inputenc}
\usepackage[T2A]{fontenc}
\usepackage[russian]{babel}

\usepackage{amsfonts, amssymb, amsmath}  %% for math symbs
\usepackage{mathrsfs}
\usepackage{float}  %% for table floating
\usepackage{enumerate} %% for lists

\usepackage{fullpage}  %% less margin 

\usepackage{graphicx} %% for pics
\usepackage{amsthm, amsmath, amsfonts, amssymb, mathtools}

\parindent 5px  % no white space in new lines
% \baselinestretch{1.5}
\renewcommand{\baselinestretch}{1.3} 

%% titling
\title{Hello world} 
\author{Lindy2076}
\date{22.22.2} %%\today

\begin{document}
  \textbf{Теорема} Есть задача Коши(перестали песать векторы, потому что надоело):

  $
  \begin{cases}
    \dot x = f(t, x, \mu)\\ x(t_0) = x_0\\
    (t_0, x_0) \in G_\mu
  \end{cases},$
  $\exists \frac{\partial f}{\partial x}, \frac{\partial F}{\partial \mu} \in C(G_\mu) \implies X(t, t_0, x_0, \mu) $ - решение

  1. $\exists \frac{\partial x}{\partial t}, \frac{\partial x}{\partial T_0}, \frac{\partial x}{\partial x_0}, \frac{\partial x}{\partial \mu} \in C(D_\mu)$, при этом:
  
  2. $\frac{\partial x }{\partial x_0} =: y $ -- решение однородной $\dot y = \frac{\partial f}{\partial x}\big(t, x(t, t_0, x_0, \mu), \mu\big)y $

  3. $\frac{\partial x}{\partial \mu} = y$ -- решение неоднородной $\dot y = \frac{\partial f}{\partial x}\big(t, x(t,t_0,x_0,\mu), \mu\big)y + \frac{\partial f}{\partial \mu}$

  D. Достаточно доказать существование частных производных по начальным данным и параметрам
  % извините, подвиньтесь (с)
  ну там крч непрерывное поэтому решение тоже гладкое
  Почему существует частная производная $\frac{\partial x}{\partial x_0{\text{чёта}}}$? Рассмотрим решение $x(t,t_0, x_0, \mu)$ на $[\tau, t]\times V\subset G_\mu$.

  Пусть $0 < r$ даст нам $\forall h: |h| < r $ и даже $t_0, x_0$
  
  $x(t, n) = x(t, t_0, x_0 + \underset{\text{произв по напр}}{h\cdot e_i}, \mu), \quad\Delta x = \underbrace{x(t,h)}_{\begin{cases}
    \dot x = f(t, x, \mu) \\ x(t_0) = x_0 + he_i
  \end{cases}} - \underbrace{x(t, 0)}_{\begin{cases}
    \dot x = f(t, x, \mu)\\x(t_0) = x_0
  \end{cases}}$

  Продифференцируем по $t$:

  $\frac{d}{dt}(\Delta x) = f\big(t, x(t,h), \mu\big) - f\big(t,x(t,0), \mu\big) = 
  % \int^1_0 \frac{\partial f}{\partial x} \big(t, x(t,0) + s\Delta x, \mu\big)ds = 
  \int^1_0 \frac{\partial f}{\partial x} \big(t, x(t, 0) + s\Delta x, \mu\big)\Delta x ds =\\
  \int^1_0 \frac{\partial f}{\partial x}\big(t, x(t,0) + s\Delta x, \mu\big)ds \frac{\Delta x}{n}
  $

  Можно ли $(h \to 0 ?)$
  Интеграл непрерывен, зависит от $t,h$, обозначим его за $A$:

  $A(t, n) \longrightarrow A(t, 0) = \frac{\partial f}{\partial x} \big( t, x(t, t_0, x_0, \mu), \mu\big)$

  $y = \lim_{h \to 0} \frac{\Delta x}{h}$  -- решение $\begin{cases}
    \dot y = \int^1_0 A(t, 0)ds\cdot y = A(t, 0)y \quad \frac{\Delta x}{h} = e_i\\ y(t_0) = e_i
  \end{cases} $

  Очевидно, что наш $y = \frac{\partial x}{\partial x_{0, i}}$

  Ну мы кроме непрерывности и следующего не пользовались:

  $\frac{\partial x(t_0)}{\partial h}= \frac{x_0 + he_i - x_0}{h} = \frac{he_i}{h} = e_i$ 

  Предел существует, потому что это решение задачи, для которой существует единственное решение.

  Почему существует $\frac{\partial x}{\partial \mu_j}$?

  Аналогично! Возьмём $x(t,q) = x(t, t_0, x_0)\mu + qe_j$

  $\Delta x = x\underbrace{(t,q)}_{\begin{cases}
    \dot x = f(t, x, \mu + qe_j)\\x(t_0) = x_0
  \end{cases}
} - \underbrace{x(t,0)}_{\begin{cases}
    \dot x = f(t, x, \mu) \\ x(t_0) = x_0
  \end{cases}}$

  etc...

  Почему существует $\frac{\partial x}{\partial t_0}$?

  $x\big(t, t_0, x_0(t_0, t, x)\big) \equiv x$ в силу единственности решения задачи Коши на одной и той же траектории.

  Возьмём некий $t_0$ на некотором $I(t, x)$. И по скольку мы уже доказали существование производной $\frac{\partial x}{\partial x_0}, \frac{\partial x_0}{\partial t_0}$ и непрерывная, посчитаем производную икса 

  Далее зависимость от $\mu$ опускаем\dots

  Что такое $\frac{\partial x}{\partial x_0}$??? Это: $$\lim_{\Delta t_0 \to 0} \frac{x\big(t, t_0 + \Delta t, x_0(t_0, t, x)\big) - x\big(t, t_0, x_0(t_0, t, x)\big) \pm x\big(t, t_0 + \Delta t, x_0(t_0 + \Delta t, t,x)\big) }{\Delta t_0} =$$

  $$= -\frac{\partial x}{\partial x_0} \frac{\partial x_0}{\partial t_0} = - \frac{\partial x}{\partial x_0} f(t_0, x_0)$$ они существуют и непрерывны.

  \par $ $

  \textbf{Следствие (Формула Лиувилля)}

  $$\det \frac{\partial x}{\partial x_0}(t, t_0, x_0) = e^{\int^t_{t_0} \text{Tr} \frac{\partial f}{\partial x}(s, x(s, t_0, x_0))ds}$$

  $\begin{cases}
    \dot y = \frac{\partial f}{\partial x_0}y\\ y(t_0) = I
    
  \end{cases}$

  $x(t, t_0, x_0) =: x_t(x_0) = T_t$ -- отображение нмерного фазового пространства в себя

  Элемент объема $\det \frac{\partial x_t(x_0)}{\partial x_0} dx_0$ 

  И если $\text{Tr} \frac{\partial f}{\partial x} \equiv 0 \implies $ отображение $T_t$ сохраняет фазовый объём.

  Т. Лиувилля верна в этом случае

  \textbf{Пример:} есть гамельтониан, две группы переменных ($y$ - не координата)
  
  $\dot x_k = \frac{\partial E}{\partial y_k} = F_k$

  $\dot y_k = - \frac{\Delta E}{\partial dx_k}$

  $\sum \frac{\partial f_k}{\partial x_k} + \frac{\partial f_{k + k}}{\partial y_k} = 0$

  \par $ $
  
  \textbf{Теорема.} Пусть существуют и непрерывны в $G_\mu$ всевозможные 
  производные $f$ по $x, \mu$, вплоть до порядка $l \ge 1$.

  Тогда $x(t, t_0, x_0, \mu)$ -- решение $\begin{cases}
    \dot x = f(t, x, \mu) \\ x(t_0) = x_0
  \end{cases}$ 
  имеет всевозможные непрерывные производные по $x_0, \mu$ вплоть до $l$-того порядка. 

  Д. Индукция по $l$.

  \par $ $

  {\large\textbf{Устойчивость решений диффуров}}

  \textbf{Устойчивость в малом (или по первому приближению)}

  Рассмотрим 
  $\dot x = f(t, x, \mu), \quad f:G_\mu \to \mathbb{R}^n, \quad f\in \text{Lip}_x(G_\mu)$ локально, $x \in \mathbb{R}^n, \mu \in \mathbb{R}^m$

  Пусть при $\mu = \overline{\mu}, x= \overline{x}(t)$ решения, определено при $t \in  [t_0, \infty)$

  Решение $\overline{x}: [t_0, \infty) \to \mathbb{R}^n$ при $\mu = \overline{\mu}$ называется \textbf{устойчивым в малом} тогда и только тогда, когда 
  
  существует окрестность начального условия $V^* = V^*(\overline{x}(t_0, \mu), \mu) \in \mathbb{R}^{n+m}$
%это не наш метод(с)

  $S = [t_0, \infty) \times V^* \subset D_\mu$ - область определения $x = x(t, x_0, \mu)$ и $x$ непрерывно по $(x_0, \mu)$ в точке $\big(\overline{x}(t_0), \overline{\mu}\big)$ равномерно по $t \in [t_0, + \infty)$
  
  Иначе решение называется неустойчивым

  Пример. 

  $\dot x = \mu x, \quad \mu, x \in \mathbb{R}$

  $x(t, x_0, \mu) = x_0 e^{\mu (t-t_0)}$

  Если $\mu = \overline{\mu} < 0 \implies $ устойчивость в малом есть

  $| x(t, x_0, \mu) | \le |x_0| e^{\mu(t-t_0)} \underset{t\to\infty}{\longrightarrow} 0$

  $\mu \in U_\delta(\overline{\mu}) < 0$

  Иначе, если $\overline{\mu} \ge 0 \implies \mu > 0 \implies e^{\mu(t-t_0)} \to \infty$. $x$ неустойчив.

  \par $ $

  Как понять в общем случае устойчива ли система?

  Положим $y = x- \overline{x}(t), \alpha = \mu - \overline{\mu} \implies $

  $\begin{cases}
    \dot y = F(t, y, \alpha), & F(t, y, \alpha) = f(t, y + \overline{x}(t), \alpha + \overline{\mu}) - f(t, \overline{x}(t), \overline{ \mu}) \\ y = 0
  \end{cases}$

  При $\mu = \overline \mu \implies \alpha = 0 \implies y \equiv 0$ при $t \in [t_0, \infty)$

  Решение $y \equiv 0, t \in [t_0, \infty)$ при $\alpha = 0$ \textbf{устойчива в малом} $\iff \\
  \forall \varepsilon > 0 \ \exists \delta_1 > 0, \delta_2 > 0 \ \forall \|y_0\| < \delta_1, \forall \| \alpha \| < \delta_2 :$

  1. решение $y(t, y_0, \alpha) \ \begin{cases}
    \dot y = F(t, y ,\alpha) \\ y(t_0) = y_0
  \end{cases}$ продолжимо на $[t_0, \infty)$

  2. $\forall t \ge t_0 : \| y(t, y_0, \alpha) \| < \varepsilon $

  Пример. $\exists \delta^* > 0 : U = \{(t, x, \mu) : t \in [t_o, \infty), \| x - \bar x(t) \| < \delta^*, \| \mu - \mu^* \| < \delta^*\} \subset G_\mu$  - область липшецевости

  $\forall (t_0, x_0, \mu) \in U \| y(t, y_0, \alpha) \| < \varepsilon \implies $ продолжимо в силу липшицевости на $[t_0, \infty)$
\end{document}