\documentclass[12pt, a4paper]{article}
\usepackage[utf8]{inputenc}
\usepackage[T2A]{fontenc}
\usepackage[russian]{babel}

\usepackage{amsfonts, amssymb, amsmath}  %% for math symbs
\usepackage{mathrsfs}
\usepackage{float}  %% for table floating
\usepackage{enumerate} %% for lists

\usepackage{fullpage}  %% less margin 

\usepackage{graphicx} %% for pics

\parindent 0px  % no white space in new lines

%% titling
\title{Hello world} 
\author{Lindy2076}
\date{22.22.2} %%\today

\begin{document}
\section{начало}

	Диффура (ОДУ) - обыкновенные диффиренциальные уравнения
	
	$$F(x, y(x), y'(x), y''(x), \dotsc, y^{(n)}(x)) = 0$$ - обычное дифф уравнение $n$-ного порядка
	
	$$\frac{\delta F}{\delta y^{(n)}} \neq 0$$

Это был общий вид.\\
Канонический вид:
$$y^{(n)} = f(x,y,y',\dotsc,y^{(n-1)})$$

второй закон ньютона: 
\begin{align}
m\vec{a} &= \vec{F} \\
m\vec{X}''(t) &= \vec{F}(t, \vec{x}(t), \vec{x}'(t))
\end{align}
это типичное дифф. уравнение

Задача Коши:



Теорема Пуанкаре: $\exists!$ решения задачи коши
$$\begin{matrix}
	y' = F(x,y) \text{ чета пусто непон}\\
	y(x_0) = y_0
\end{matrix}$$
$ y = \phi(x) - \text{чзх}$\\
$(x_0, y_0) \in Int(X, Y)$\\
1) $f \in C(\overline{X}, \overline{Y}), X, Y$ - области \\
2) Функция Липшицева по y, равномерна по $x \in \overline{X}$
$\exists L>0: | f(x,y_1) - f(x,y_2)| \le L |y_1-y_2|$\\

$$\int^{x}_{x_0} \cdot dt | y'(x) = f(x, y(x)) \implies \\ y(x)-y(x_0) = \int^x_{x_0} f(t, y(t))dt \implies y(x) = y_0 + \int^x_{x_0}f(t,y(t))dt$$

как решить последнюю 
gbrfhjdtybt
$$\{\phi_k\}^\infty_{k=0}$$
$$
\begin{matrix}
\phi_0 =& y_0 \\
\phi_1 =& y_0 + \int^x_{x_0}f(t, \phi_0(t))dt \\
\dotsc \\
\phi_m =& y_0 + \int^x_{x_0}f(t, \phi_{m-1}(t))dt \\
\dotsc \\
\end{matrix}$$
$$ \phi_k \in C(\overline{X \sup K}), \phi_k \rightarrow^k \phi, k\to\infty
$$

$$\phi_0(x) + (\phi_1(x) - \phi_o(x)) + (\phi_2(x) - \phi_1(x)) + \dotsc = ? = \phi_0 + \sum^\infty_{k=1}(\phi_k - \phi_{k-1}) \le \text{ суммируемая мажоранта}
$$
чета про веерштраса и мажоранты
посчитаем модуль разницы

$$
|\phi_m(x) - \phi_{m-1}(x)| \le \inf^x_{x_0} |f(t, \phi_{m-1}(t)) - f(t, \phi_{m-2}(t))|dt \le^{\text{липцевость} }
$$

$$
x, x_o \in K - \text{ компакт}
$$
$$
\le L \int^x_{x_0} |\phi{m-1}(t) - \phi_{m-2}(t) |dt \le 
$$

по индукции предполгаем, что этот модуль не превосходит ...

$$
M = \max_{x,y \in(\overline{X}, \overline{Y})} |f(x,y)|
$$
$$
	\max_{x\in K} | \varphi_m(x) - \varphi_{m-1}(x)| \le \frac{ML^{m-1}(x-x_0)^m}{m!}
$$
продолжение дела...
$$
\le L \int^x_{x_0}||dt \le \frac{ML^{m-2}L}{(m-1)!}\int^x_{x_0}|t-x_0|^{m-1}dt \le \frac{ML^{m-1}(x-x_0)^m}{m!}
$$

База $|\varphi_1 - \varphi_0| \le \int^x_{x_0} |f(t, y_0)|dt$

$$
\varphi_0 + \sum^\infty_{k=1}(\varphi_k - \varphi{k-1}) \le \frac{M}{L}\sum^\infty_{m=1} \frac{L^m (x-x_0)^m}{m!} = |\varphi_0| + \frac{M}{L} (e^{L|x-x_0| - 1)}
$$

$$
\varphi_k(x) = y_0 + \int^x_{x_0} f(t, \varphi_{k-1}(t))dt \le M(x-x_0), k \to \infty
$$
перделим $\varphi_k(x)$ к $\varphi(x)$
$$
\varphi(x) = y_0 + \int^x_{x_0} f(t, \varphi(t))dt
$$
чета по теореме Бэроу про переменный верхний пердел
показали существование короче. осталась единственность.
Положим, что есть два решения. $\varphi$  и $\psi$.
$$
\psi(x) = y_0 + \int^x_{x_0} f(t, \psi(t))dt
\varphi(x) = y_0 + \int^x_{x_0} f(t, \varphi(t))dt
$$

$\displaystyle |\psi(x) - \varphi(x)| \\ |\psi(x) - \varphi_m(x)| \le \int^x_{x_0} |f(t, \psi(t)) - f(t, \varphi(t))|dt \le \frac{L^{}m-1(x-x_0)^m}{m!} \\
база |\psi(x) - y_0| \le \int^x_{x_0}F(t, \psi(t))dt \le M|x-x_0| \to 0, m \to \infty
 $ \\

докзали на комакте $K$. Он входит в некую окрестность точки $x_0$.\\
// $M$ связно, если для любых двух открытых множеств $G_1, G_2$ и $M \in G_1 \cup G_2$ и $G_1 \cap G_2 \neq \emptyset$ 
Следствие 1:\\ 
Пусть $\overline{x} = [x_0-a, x_0+a]$, $\overline{y} = [y_0-b, y_0+b]$ - отрезки.\\
$\exists$ пикаровский интервал $[x_0 - h, x_0 + h]$, $\varphi $ - решение задачи коши \\
$\forall x \in [x_0 +h, x_0-h], h = \min\{a, 
frac{b}{M}\}$ \\
$\varphi(x) = y_0 + \int^x_{x_0}f(t, \varphi(t))dt$\\

$|\varphi(x) - y_0| \le M|x-x_0| \le b$\\\\

Следствие 2:
$\left\{
\begin{matrix}
\vec{y}'(x) = \vec{f}(x, \vec{y}(x))\\
\vec{y}(x_0) = \vec{y}^{(0)}

\end{matrix} \right.$ \\

$\vec{y}, \vec{f} \in \mathbb{R}^n$ \\

$1.) \vec{f} \in C(\overline{X}, \overline{Y})\\
 2.) || \vec{f}(x, \vec{y}^{(1)} - \vec{f}(x, \vec{y}^{(2)}) || \le L|| \vec{y}^{(1)} - \vec{y}^{(2)} ||$\\\\

Следствие 3:\\
$\frac{\delta f}{\delta y}(x, y) \in C(\overline{Y}) \implies L = \max_{K \subset Y} |\frac{\delta f}{\delta y}|$\\\\

Следствие 4:\\

пусть условие липшевости выполнено независимо\\
% вообще ниже три палочки мб описка хз

$\forall a >0, ||\vec{y}|| < \infty \exists L_a = L$\\

$||f(x, \vec{y}^{(1)}) - f(x, \vec{y}^{(2)})|| \le L || \vec{y}^{(1)} - \vec{y}^{(2)} \implies \exists! \vec{\varphi}(x)||$, определенная на $[x_0 -a, x_0 + a]$\\

Если же $L_a = L(a) \quad \forall \vec{y} \in \mathbb{R}$, то вообще говоря не всякое решение продолжимо, даже на отрезке $[x_0-a, x_0+a]$ \\


Непрерывная зависимость решения задачи Коши
% че бл y_m нихкиx m \implies y (x) m m 
$
\left\{
\begin{matrix}
y' = f(x, y_1) \\
y(x_0) = m
 
\end{matrix}
\right.
$\\
$ z = y - m \implies \left\{
\begin{matrix}
z' = f(x, z+m) \\ z(x_0) = 0
\end{matrix} \right.
 = f(x,z,m)
$ 

Теорема:\\

$\left\{
\begin{matrix}
y' = f(x, y, m)\\ y(x_0) = y_0
\end{matrix}\right.
$\\

1. $f(x,y,m) \in C(\overline{D}), \overline{D} = \{ |x-x_0| \le y, |y-y_0| \le b, |m - m_0 \le c| \} $\\
%липшевость чета
2. $|f(x,y,m) - f(x,y_2,m)| \le L|y_1 - y_2|$\\
$\implies [x_0 - h, x_0 + h), h_1 = \min\{a, \frac{b}{m}\}$\\
$\exists!\varphi_m(x) \in C_m[m_0-c, m_0+c]$\\

\begin{list}{•}{Литература:}
\item Курс обыкновенных диффуров. Петровский
\item Чето тоже про дифуры. Эльцгольц
\end{list}
 
 
\end{document}