\documentclass[a4paper, 12pt]{article}

\usepackage[english, russian]{babel}
\usepackage[T2A]{fontenc}
\usepackage[utf8]{inputenc}
\usepackage{amsthm, amsmath, amsfonts, amssymb, mathtools}
\usepackage{geometry}
\usepackage{indentfirst}
\usepackage{titleps}
\usepackage{soulutf8}
\usepackage{multicol}
\usepackage{tabularx}
\usepackage{pgfplots}
\usepackage{cancel}
\usepackage{import}
\usepackage{xifthen}
\usepackage{pdfpages}
\usepackage{transparent}
\usepackage{setspace}
\usepackage{graphicx}
\usepackage{float}
\usepackage{wrapfig}
\usepackage{contour}
\usepackage{mathrsfs}

\onehalfspacing

\contourlength{1pt}

\pgfplotsset{compat=1.18, width=7cm}

\newpagestyle{main}{
    %\setheadrule{0.4pt}
    \sethead{}{}{}
    %\setfootrule{0.4pt}
    \setfoot{}{\thepage}{}
}

\pagestyle{main}

\theoremstyle{plain}
\newtheorem{theorem}{Теорема}
\newtheorem{corollary}{Следствие}
\newtheorem{lemma}{Лемма}[]
\newtheorem*{lemma*}{Лемма}
\newtheorem*{definition}{Определение}
\newtheorem*{remark}{Замечание}
\newtheorem{example}{Пример}
\newtheorem*{proposition}{Предложение}
\newtheorem*{theorem*}{Теорема}
\newtheorem*{example*}{Пример}
\newtheorem*{corollary*}{Следствие}

\geometry{top=25mm}
\geometry{bottom=30mm}
\geometry{left=20mm}
\geometry{left=20mm}

\newcommand{\incfig}[1]{%
    \def\svgwidth{\columnwidth}
    \import{./figures/}{#1.pdf_tex}
}

\graphicspath{ {./figures/} }

\DeclareMathOperator{\Kerr}{Ker}
\DeclareMathOperator{\Imm}{Im}
\DeclareMathOperator{\Int}{Int}
\DeclareMathOperator{\Mat}{Mat}
\DeclareMathOperator{\End}{End}
\DeclareMathOperator{\sign}{sign}
\DeclareMathOperator{\dist}{dist}
\DeclareMathOperator{\rank}{rank}
\DeclareMathOperator{\diam}{diam}
\DeclareMathOperator{\diag}{diag}
\DeclareMathOperator{\supp}{supp}
\DeclareMathOperator{\grad}{grad}
\DeclareMathOperator{\rot}{rot}
\DeclareMathOperator{\divv}{div}
\DeclareMathOperator{\Ext}{Ext}
\DeclareMathOperator{\Id}{id}
\DeclareMathOperator{\Char}{char}
%\DeclareMathOperator{\dist}{dist}
\DeclareMathOperator*{\id}{id}
\renewcommand{\phi}{\varphi}
\renewcommand{\theta}{\vartheta}
\renewcommand{\epsilon}{\varepsilon}
\newcommand{\R}{\mathbb{R}}
\renewcommand{\C}{\mathbb{C}}
\newcommand{\Q}{\mathbb{Q}}
\newcommand{\N}{\mathbb{N}}
\setcounter{lemma}{11} % вот тут пофиксить
\newcommand{\lrhimani}[1]{\underset{#1}{\underline{\int}}}
\newcommand{\urhimani}[1]{\underset{#1}{\overline{\int}}}
\newcommand{\rhimani}[1]{\underset{#1}{\int}}
\newcommand{\mycontour}[1]{\contour{red}{#1}}
\newcommand{\charf}[1]{\chi_{#1}(x)}
\newcommand{\pfrac}[2]{\frac{\partial #1}{\partial #2}}

\begin{document}
    
    \title{Математический анализ}
    \date{5 сентября 2022}
    \maketitle{}

    \pagebreak
    \section*{Повторение}
    \subsection*{Линейные и нормированные пространства}
    
    $\mathit{L}$ -- линейное пространство
    \newline
    \par
    \begin{tabularx}{\textwidth}{X l}
        $\| \cdot \|$ -- норма & $\| \alpha x \| = |\alpha| \| x \|, \forall \alpha \in \R (\C), x \in \mathit L$
    \end{tabularx}
    \newline
    \par
    \begin{tabularx}{\textwidth}{X l}
        Нормированное пространство & $\| x + y \| \le \| x \| + \| y \|$ \\
        & $\| x \| = 0 \Rightarrow x = 0$ \\
    \end{tabularx}

    \begin{remark}
        Норма всегда порождает метрику (нормированное $\Rightarrow$ метрическое).
    \end{remark}

    \begin{remark}
        $\forall$ конечномерное пространство полное.
    \end{remark}

    \begin{definition}
        Полное нормированное пространство $\leftrightharpoons$ банохово.
    \end{definition}

    \begin{illustration}
        Неполное нормированное пространство:
        \[
            C([0; 1]), \quad \| f \|_L = \int_0^1 |f(x)| dx
        \]
        \[
            \int_0^\frac12 f_n(x) dx \rightarrow 0 \quad \exists N: n > N < \frac \epsilon 2
        \]
        \[
            \int_\frac12^1 (1 - f_n(x)) dx \rightarrow 0 \quad \exists N: n > N < \frac \epsilon 2
        \]
        \[
            \forall \epsilon \ \exists N \ \forall n, m > \int_0^1 |f_n(x) - f_m(x)| dx < \epsilon   
        \]
        Можно дополнить его до полного, получится:
        \[
            \mathit{L}_1(0, 1) = \{ f : [0, 1] \rightarrow \R : \int_0^1 |f_n(x)| dx < \infty \}    
        \]
        $C([0,1])$ будет полным по другой норме: $\| f \| = \max_{x\in[0,1]} |f(x)|$
    \end{illustration}

    \subsection*{Линейные операторы}

    \begin{definition}
        Линейный оператор
        \[
            A(\alpha x + \beta y) = \alpha Ax + \beta Ay \quad A : L \rightarrow M, \quad \text{ где } M = \R / \C,\quad A \text{ - функционал(?)}    
        \]
    \end{definition}

    \begin{remark}
        Операторы из $\R^m$ в $\R^n$ $\leftrightarrow$ матрицы $\Mat^{m, n}$ 
    \end{remark}

    \subsection*{Норма оператора}

    \[
        \| A \| = \sup_{x \not= 0} \frac{\| Ax \|_\mathit{M}}{\| x \|_\mathit{L}}, \quad M, L \text{ - нормированные пространства}
    \]

    \begin{illustration}
        Неограниченный оператор:
        \[
            L = C^1([0, 1]), \quad M = C([0, 1]) \quad \| f \| = \sup_{[0, 1]} |f|    
        \]
        \[
            \| f \| = \max_{[0,1)} |f|    
        \]
        \[
            (Af)(x) = f'(x)     
        \]
        \[
            f_n(x) = x^n \quad \| f_n \| = 1 \ \forall n \quad Af_n = f_n' = nx^{n-1} \quad \| Af \| = n    
        \]

    \[\raisebox{-15mm}{\begin{tikzpicture}
        \begin{axis}[ymin=0, ymax=1, xmin=0, xmax=1]
            \addplot[color=red, dashed, mark=x, samples=100]{x^2};
        \end{axis}
    \end{tikzpicture}}
    \qquad
    \frac{\| Af_n \|}{\| f_n \|} \xrightarrow[n \rightarrow \infty]{} \infty
    \]
    \end{illustration}

    \begin{proposition}
        \[
            \| A \| = \sup_{x \in B_1(x) \setminus \{ 0 \}} \|Ax \| = 
            \sup_{a \in S_1(0)} \| Ax \| = 
            \sup_{x \in B_1(0) \setminus \{ x \}} \frac{\|Ax\|}{\|x\|} = 
            \inf \{c : \| Ax \| \le c \| x \|\} \quad \forall x \in L
        \]
        \par$B_r(x) = \{ y \in L  : \| y - x \| < r\}$ -- открытый шар радиуса $r$
        \par$B_r[x] = \{ y \in L  : \| y - x \| \le r\}$ -- замкнутый шар радиуса $r$
        \par$\bar B_r(x) \not= B_r[x]$, где $\bar B_r(x)$ -- замыкание
        \par$S_r(x) = \{ y \in L  : \| y - x \| = r\}$ -- сфера
    \end{proposition}

    \begin{proposition}
        $A \in B(L) \text{ ( -- множество ограниченных линейных операторов)} \Leftrightarrow A$ непр. в точке $0 \Leftrightarrow A$ непр. в $\forall x \in L \Leftrightarrow A$ равн. непр. на $L$.
    \end{proposition}

    \begin{remark}
        \[
            \| A_1 \cdot A_2 \| \le \| A_1 \| \cdot \| A_2 \|   
        \]
        \[
            A : \R^n \rightarrow \R^n, \ A \in \Mat^{n, n} \quad \| A \| \leq \sqrt{\sum_{i, k = 1}^n |a_{i, k}^2|}
        \]
    \end{remark}

    \begin{definition}
        Матрицы $\|\cdot\|$ и $|\cdot|$ эквивалентны, если $\exists c_1, c_2 > 0$ т. ч.
        \[
            \forall x \in L \ c_1 \|x\| \le |x| \le c_2 \|x_2\|
        \]
        \par Тогда $\| A \| \thicksim \sum_{i, k = 1}^n |a_{ik}| \thicksim \max_{i, k \in \{1, \dots, n\}} |a_{ik}| \thicksim \sqrt{\sum_{i, k = 1}^n |a_{ik}|^2}$
    \end{definition}

    \begin{remark}
        \[
            A : \R^n  \rightarrow \R \ \exists a \in \R^n \ \forall x \in \R^n : Ax = (a, x) \quad \| A \|_{B(\R^n, \R)} = \| a \|_{\R^n}  %\underset{B(\R^n, \R)}{=} \| a \|_{\R^n}   
        \]
        \[
            A : \R \rightarrow \R^n \ \exists a \in \R^n \ \exists x \in \R : Ax = a \cdot x \quad \| A \|_{B(\R, \R^n)} = \| a \|_{\R^n}  
        \]
    \end{remark}

    \subsection*{Обратный оператор}
    
    $A : L \rightarrow M$ -- линейный оператор

    \begin{enumerate}
        \item $\exists B : M \rightarrow L \ : \ AB = I_M$ -- ед. оператор в пространстве $M$
        \par $B \leftrightharpoons$ правый обратный
        \item $\exists C : M \rightarrow L \ : \ CA = I_L$ -- ед. оператор в пространстве $L$
        \par $C \leftrightharpoons$ левый обратный
        \item $\exists$ оба и равны, $A^{-1} \leftrightharpoons$ обратный оператор
    \end{enumerate}

    $A \in \Mat^{n} : \exists A^{-1} \Leftrightarrow \det A \not= 0 \Leftrightarrow \Kerr A = \{0\} \Leftrightarrow \dots \Leftrightarrow \rank A = n$

    \begin{theorem}
        $A \in B(\R^n), \exists A^{-1}, B \in B(\R^n), \| B - A \| < \frac{1}{\| A^{-1} \|}$
        \par Тогда $B$ обратим,
        \[
            \| B^{-1} \| \le \frac{1}{\frac{1}{\| A^{-1} \|}  - \| B - A \|}, \| B^{-1} - A^{-1} \| \le \frac{\| A^{-1} \| \cdot \| B - A \|}{\frac{1}{\| A^{-1} \|} - \| B - A \|}  
        \]
    \end{theorem}

    \begin{proof}
        $x \in \R^n$
        \[
            \| Bx \| = \| Ax - (A-B)x \| \ge \| Ax \| - \| (B-A)x \| \ge \frac{\| x \|}{\| A^{-1} \|} - \| B-A \| \cdot \| x \| = \left(\frac{1}{\|A^{-1}\|} - \|B-A\|\right) \|x\|
        \]
        \par Так как:
        \[
            \|Ax\| \ge \frac{\|x\|}{\|A^{-1}\|} \quad x = (A^{-1})(Ax) \quad \|x\| \le \|A^{-1}\| \cdot \|Ax\|  
        \]
        
            $Bx = 0 \Rightarrow \|x\| = 0$, так как  $(\|Bx\| \ge \|x\| \cdot \underbrace{(..)}_{>0}) \Rightarrow x = 0 \quad \Kerr B = \{0\} \Rightarrow \exists B^{-1}$  
        
        \[
            \begin{matrix} y = Bx \\ x = B^{-1}y \end{matrix} \quad \|y\| \ge \left(\frac{1}{\|A^{-1}\|} - \|B-A\|\right)\|B^{-1}y\|, \ \forall y \in \R^n  
        \]
        \[
            \Rightarrow \|B^{-1}\| \le \frac{1}{\frac{1}{\|A^{-1}\|} - \|B-A\|}    
        \]
        \[
            B^{-1} - A^{-1} = B^{-1}(I-BA^{-1}) = B^{-1}(A-B)A^{-1}
        \]
        \[
            \|B^{-1}-A^{-1}\| \le \|B^{-1}\| \cdot \|A - B\| \cdot \|A^{-1}\| \le \frac{\|A^{-1}\| \cdot \|B-A\|}{\frac{1}{\|A^{-1}\|} - \|B-A\|}
        \]
    \end{proof}

    \begin{remark}\leavevmode
        \begin{enumerate}
            \item Множество обратимых операторов открыто
            \item Отображение $A \mapsto A^{-1}$ непрерывно
        \end{enumerate}
    \end{remark}
    
    \section*{Дифференцирование обратной функции}

    $D \subset \R^n \quad f : D \rightarrow \R^n \quad x \in \Int D \quad \exists A \in B(\R^n. \R)$
    
    \begin{definition}
        Если $f(x + h) = f(x) + Ah + o(\|h\|), \quad h \rightarrow 0$, тогда говорят $A$ -- производная $f$ в точке $x$.
    \end{definition}

    \begin{figure}[ht]
        \centering
        \incfig{figurka}
        \label{fig:figurka}
    \end{figure}

    % \[
    %     y_0 = f(x_0) \in \Int f(D)  
    % \]
    % \[
    %     \exists (f^{-1})(y_0) = (f'(x_0))^{-1}    
    % \]

    Рассмотрим $f^{-1} \circ f = \id_D$

    Продифференцируем : $(f^{-1})'(\underbrace{f(x_0)}_{= y_0}) \cdot f'(x_0) = I$

    Пусть теперь есть функция на открытом множестве:

    $D$ открыто $\quad f \in C'(D, \R^n) \quad x _0 \in D \ f'(x_0)$ обратима

    \begin{illustration}\leavevmode
        \begin{enumerate}
            \item $f(x, y) = \begin{pmatrix}e^x \cos y \\ e^x \sin y \end{pmatrix}$
            \par $f : \R^2 \rightarrow \R^2$
            \[
                f'(x, y) = \begin{pmatrix} e^x\cos y & - e^x\sin y \\ e^x\sin y & e^x\cos y \end{pmatrix}    
            \]
            \item $n = 1 \quad f \in C^1(D, \R)$
            \par $f'(x_0) \not= 0$
            \par $f\big|_U$ -- биекция между $U$ и $V$
            \[
                \exists (f^{-1})'(y) = \frac{1}{f(f^{-1}(y))} \quad \begin{aligned} &f \in C^1(U, V) \\ &f^{-1} \in C^1(V, U) \end{aligned}    
            \]
        \end{enumerate}
    \end{illustration}

    \begin{theorem}
        $D \subset \R^n$ открыто, $f \in C^1(D, \R^n)$
        \par $x_0 \in D, \ f^{-1}(x_0)$ -- обратимая матрица
        \par Тогда $\exists$ окрестность $x_0, \> U \subset D$
        \par $V := f(U)$ открыто и $f\big|_U$ -- биекция между $U$ и $V$, $f^{-1} \in C^1(V, U)$
        \[
            (f^{-1})'(f(x_0)) = (f'(x_0))^{-1}, \ \forall x \in U \quad f'(x_0) \in I    
        \]
    \end{theorem}

    \begin{proof}
        $ $ \par
        \begin{enumerate}
            \item Пусть $f'(x_0) = I$. При $x$, близких к $x_0$, $f(x) \not= f(x_0)$
                \[
                    f(x) = f(x_0) + (x-x_0) + o(\|x-x_0\|)    
                \]
                \[
                    \frac{\|f(x)-f(x_0)\|}{\| x-x_0 \|} = \frac{\|x-x_0\|}{\|x-x_0\|} + o(1), \ x \rightarrow x_0  
                \]
                $\exists$ окрестность, в которой $\|o(1)\| < \frac12 \Rightarrow f(x)-f(x_0) \not= 0$
            \item $f \in C^1(D, \R^n) \Rightarrow f'(x) \xrightarrow[x \rightarrow x_0]{} I$
                \par $\exists r > 0, \ x_0 : \|f'(x) - I\| < \frac12$ для $\forall x \in B_r(x_0)$
                \par $K = B_{\frac{r}{2}}[x_0] \subset B_r(x_0)$
            \item $g(x) = f(x) - x$
                \par $g'(x) = f'(x) - I \quad \|g'(x)\| < \frac12, \ x \in K$
                \[
                    \|g(x_1) - g(x_2)\| \le \frac12 \|x_1-x_2\| \quad \text{(т. Лагранжа)}
                \]
                \[
                    \|f(x_1) - f(x_2) - (x_1 - x_2)\| 
                    \le \frac12 \|x_1 - x_2\| \Rightarrow 
                \]
                \[
                    \frac12 \|x_1 - x_2\| \le \|f(x_1) - f(x_2)\| 
                    \le \frac32 \|x_1 - x_2\| \quad \text{(нер-во треуг., из этого следует инъективность)}   
                \]
                \[
                    0 \le \|f(x_1) - f(x_2)\| \le C \cdot \|x_1 - x_2\| - \text{из этого следует непрерывность} 
                \]
                $f$ -- биекция из $K$ в $f(K)$
                \par $f^{-1}$ -- из $f(K)$ в $K$ непр.
            \item $\delta(K)$ компактно, поскольку это сфера, сфера замкнута, сферу можно вписать в куб, куб - компакт, сфера - его замкнутое подмножество, значит тоже компакт.
                \[
                    \|f(\cdot) - f(x_0)\|_{\ge 0} \in C(\delta(K), \R), \ x_0 \not\in \delta(K) \quad\text{(вместо точки подставляем x)}
                \]
                Если $\inf_{x \in \delta(K)} \|f(x) - f(x_0)\| = 0$, то $\exists x' \in \delta(K) : f(x') = f(x_0)$.
                \par Это означало бы, что $x' = x_0 \in \delta(K)$ (т. к. $f\big|_K$ биекция)
                \par Значит, $\inf_{x \in \delta(K)} \|f(x) - f(x_0)\| > 0. \ \exists d > 0 \ B_d(f(x_0)) \cap f(\delta(K)) = \varnothing$
            \item $V = B_{\frac{d}{2}}(f(x_0))$
                \par $x \in \delta(K), \ y \in V$
                \[
                    \|f(x) - y\| = \|\underbrace{f(x) - f(x_0)}_{\|\cdot\| > d} - \underbrace{(y - f(x_0))}_{\|\cdot\| < \frac{d}{2}}\| > d - \frac{d}{2} = \frac{d}{2}
                \]
                $h_y(x) = \|f(x) - y\|^2 \in C^1(K, \R_+)$ - производная скалярного произведения
                \par $x \in \delta(K) \quad h_y(x) > \frac{d^2}{4}$
                \par $x = x_0 \quad h_y(x_0) = \|f(x_0)-y\|^2 < \frac{d^2}{4}$
                \par $\Rightarrow x_y \not\in \delta(K), \ x_y \in \Int K$
                \par $h_y'(x_y) = (f(x) - y, f(x) - y)'_{\big|_{x=x_y}} = 2 (f(x) - y)^T \cdot f'(x)\big|_{x=x_y} = 0$
                \par $V \subset f(K) \Rightarrow f(x_y) = y \Rightarrow y \in f(K)$
            \item $x \in U$
                \[
                    \underbrace{\underbrace{f(x + h)}_{y+k} - \underbrace{f(h)}_y}_k = f'(x)h + \phi(x, h) \quad \frac{\|\phi(x, h)\|}{\|h\|} \xrightarrow[\|h\| \rightarrow 0]{} 0
                \]
                \[
                    \rotatebox[]{90}{\text{\scriptsize В силу биекции}} \begin{cases}
                        y = f(x) \\
                        x = f^{-1}(y) \\
                        x + h = f^{-1}(y + k)
                    \end{cases}
                \]
                \[
                    k = f^{-1}(x) \cdot (f^{-1}(y + k) - f^{-1}(y)) + \phi(f^{-1}(y, h))    
                \]
                \[
                    \left(\text{требовали в п. } 2 \text{:} \|f'(x) - I\| < \frac12 \Rightarrow \exists (f'(x))^{-1}\right)    
                \]
                \[
                    f^{-1}(y + k) = f^{-1}(y) + (f'(x))^{-1} \cdot k - \underbrace{(f'(x))^{-1} \cdot \phi(f^{-1}(y), h)} _{= o(\|k\|), \ \|k\| \rightarrow 0}   
                \]
                \[
                    ^{(*)}\left(\frac{\|(f(x))^{-1} \cdot \phi(f^{-1}(y), h)\|}{\|k\|}\right) \le \frac{\|\left(f(x)\right)^{-1}\| \cdot \|\phi(f^{-1}(y), f^{-1}(y + k) - f^{-1}(y))\|}{\|k\|}    
                \]
                \[
                    \frac12\underbrace{\|x_1 - x_2\|}_{h} \le \underbrace{\|f(x_1) - f(x_2)\|}_{k} \le \frac32\underbrace{\|x_1 - x_2\|}_{h}   
                \]
                \[
                    ^{(*)} \le \|(f'(x))^{-1}\| \cdot \underbrace{\frac{\|\phi(f^{-1}(y), f^{-1}(y + k) - f^{-1}(y))\|}{\|f^{-1}(y + k) - f^{-1}(y)\|}}_{\xrightarrow[h \rightarrow 0 \newline (k \rightarrow 0)]{} 0} \cdot \frac{\|f^{-1}(y+k) - f^{-1}(y)\|}{\|k\|}
                \]
                \[
                    \Rightarrow \exists (f^{-1})'(f(x)) = (f'(x))^{-1}    
                \]
            \item $\left(f^{-1}\right)'(y) = \left(f'(f^{-1}(y))\right)^{-1}$
                \par $f^{-1} : y \mapsto f^{-1}(y) \quad f^{-1} \in C(V, U)$
                \par $f' : f^{-1}(y) \mapsto f'(f^{-1}(y)) \quad f' \in C(D, \R^n)$
                \par $A \mapsto A^{-1} \quad$ непрерывное отображение
                \par $\quad \Rightarrow f^{-1} \in C^1(V, U)$
            \item Общий случай $\quad f'(x_0) = A, \ \det A \not= 0$
                \par $\tilde f(x) := A^{-1}f(x)$
                \par $\tilde f'(x) = A^{-1}f'(x)$
                \par $\tilde f'(x_0) = A^{-1}A = I$
                \par $\exists U, \tilde V, \tilde f\big|_U$ -- биекция, $\tilde f^{-1} \in C^1(\tilde V, U)$
                \par $\quad (\tilde f^{-1})' (\tilde V) = (\tilde f')^{-1}(U)$
                \par $f(x) = A \tilde f(x) \quad f\big|_U$ биекция м. $U$ и $A \tilde V := V$
                \par $y = A \tilde f(x) \quad x = \tilde f^{-1}(A^{-1} y) \Rightarrow f^{-1} \in C^1(V, U)$
                \par $A^{-1}y = \tilde f(x) \quad f^{-1} = \tilde f^{-1} \cdot A^{-1}$
                
                \[
                    (f^{-1})'(f(x)) = (\tilde f^{-1})(\overbrace{A^{-1}f(x)}^{=\tilde f(x)}) A^{-1} = (\tilde f'(x))^{-1} A^{-1} =
                \]
                \[
                    = (A^{-1} f'(x))^{-1}A^{-1} = (f'(x))^{-1} AA^{-1} = (f'(x))^{-1}    
                \]
        \end{enumerate}
    \end{proof}

    \begin{remark}
        \[
            \begin{cases}
                f \in C^r(D, \R^n) \\
                \dots \\
                \dots
            \end{cases} = \begin{cases}
                f^{-1} \in C^r(D, \R^n) \\
                \dots \\
                \dots
            \end{cases}
        \]
        $\quad (f^{-1})' = (f')_{ik}^{-1} = \frac{\Delta_{ik}}{\Delta} \quad$ выражается через $\frac{\partial f_l}{\partial x_m}, \quad l, m = 1, 2, \dots, n$
    \end{remark}

\end{document}