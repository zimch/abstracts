\documentclass[a4paper, 12pt]{article}

\usepackage[english, russian]{babel}
\usepackage[T2A]{fontenc}
\usepackage[utf8]{inputenc}
\usepackage{amsthm, amsmath, amsfonts, amssymb, mathtools}
\usepackage{geometry}
\usepackage{indentfirst}
\usepackage{titleps}
\usepackage{soulutf8}
\usepackage{multicol}
\usepackage{tabularx}
\usepackage{pgfplots}
\usepackage{cancel}
\usepackage{import}
\usepackage{xifthen}
\usepackage{pdfpages}
\usepackage{transparent}
\usepackage{wrapfig}
\usepackage{setspace}

\onehalfspacing

\pgfplotsset{compat=1.18, width=7cm}

\newpagestyle{main}{
    %\setheadrule{0.4pt}
    \sethead{}{}{}
    %\setfootrule{0.4pt}
    \setfoot{}{\thepage}{}
}

\pagestyle{main}

\theoremstyle{plain}
\newtheorem{theorem}{Теорема}
\newtheorem{corollary}{Следствие}
\newtheorem{lemma}{Лемма}[]
\newtheorem*{definition}{Определение}
\newtheorem*{remark}{Замечание}
\newtheorem{illustration}{Пример}
\newtheorem*{proposition}{Предложение}

\geometry{top=25mm}
\geometry{bottom=30mm}
\geometry{left=20mm}
\geometry{left=20mm}

\newcommand{\incfig}[1]{%
    \def\svgwidth{\columnwidth}
    \import{./figures/}{#1.pdf_tex}
}

\DeclareMathOperator{\Kerr}{Ker}
\DeclareMathOperator{\Imm}{Im}
\DeclareMathOperator{\Int}{Int}
\DeclareMathOperator{\Mat}{Mat}
\DeclareMathOperator{\rank}{rank}
\DeclareMathOperator{\diam}{diam}
\DeclareMathOperator*{\id}{id}
\renewcommand{\phi}{\varphi}
\renewcommand{\theta}{\vartheta}
\renewcommand{\epsilon}{\varepsilon}
\newcommand{\R}{\mathbb{R}}
\renewcommand{\C}{\mathbb{C}}
\newcommand{\Q}{\mathbb{Q}}
\newcommand{\N}{\mathbb{N}}
\setcounter{lemma}{9} % вот тут пофиксить
\DeclareMathOperator{\lrhimani}{\underset{\Pi}{\underline{\int}}}
\DeclareMathOperator{\urhimani}{\underset{\Pi}{\overline{\int}}}
\DeclareMathOperator{\rhimani}{\underset{\Pi}{\int}}

\DeclareMathOperator{\Kerr}{Ker}
\DeclareMathOperator{\Imm}{Im}
\DeclareMathOperator{\Int}{Int}
\DeclareMathOperator{\Mat}{Mat}
\DeclareMathOperator{\rank}{rank}
\DeclareMathOperator*{\id}{id}
\renewcommand{\phi}{\varphi}
\renewcommand{\theta}{\vartheta}
\renewcommand{\epsilon}{\varepsilon}
\newcommand{\R}{\mathbb{R}}
\renewcommand{\C}{\mathbb{C}}

\begin{document}
    \title{Математический анализ}
    \date{19 сентября 2022}
    \maketitle

    \pagebreak

    \subsection*{Условный экстремум функции}

    $D \subset \R^{n+m}, \ f : D \rightarrow \R, \quad z \in D$
    \par $\quad \quad \Phi : D \rightarrow \R^m$
    \quad \quad \quad условие $\Phi(z) = 0$

    \begin{definition}
        точка $z_0 \in D$ называется условным экстремумом функции $f$ при условии $\Phi = 0$, если $\Phi(z_0) = 0$ и $\exists U$ -- окрестность $z_0, \ U \subset \R^{n+m} \ \forall z \in D \cap U : \Phi(x) = 0$
        выполняется условие $f(z) \ge f(z_0)$ ( -- условный $\min$); $f(z) \le f(z_0)$ ( -- условный $\max$)
    \end{definition}

    Пусть $D \subset \R^{n+m}$, открытое, $f \in C^1(D, \R), \ \Phi \in C^1(D, \R^m)$
    \par \quad $z_0$ -- точка условного экстремума
    \par \quad $\rank \Phi'(z_0) = m$
    \par \quad Перенумеруем координаты $\begin{tabular}{|c|c|}\hline
        $\Phi'_x$ & $\Phi'_y$ \\
        \hline
    \end{tabular} = \Phi'$
    \par \quad $z = (x, y)$ так, чтобы было $\det \Phi'_y(z_0) \not= 0 \quad z_0 = (x_0, y_0)$
    \par \quad По \textit{Th. о неявном отображении} $\exists U$ -- окрестность $x_0$:
    \[
        y = \phi(x), \ x \in U, \ \Phi(x, \phi(x)) = 0
    \]
    \par Тогда $f(x, \phi(x))$ имеет безусловный экстремум в точке $x_0$ (условие выполняется)
    \par $\Rightarrow ^{(1)} \ \underbrace{f'_x(z)}_{1 \times n} + \underbrace{f'_y(z)}_{1 \times m} \cdot \underbrace{\phi'(x)}_{m \times n} = 0$ ( -- это и есть необх. усл. экстремума)

    \par $\phi'(x_0) = -\left(\Phi'_y(z_0)\right)^{-1} \cdot \Phi'_x(z_0)$
    \par $^{(1)} $ НУО: $f'_x(z) - \underbrace{f'_y(z)\left(\Phi'_y\right)}^{-1}_{= \lambda \text{ (см. след. \S)}} \Phi'_x(z) = 0$

\end{document}