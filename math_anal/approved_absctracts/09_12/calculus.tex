\documentclass[a4paper, 12pt]{article}

\usepackage[english, russian]{babel}
\usepackage[T2A]{fontenc}
\usepackage[utf8]{inputenc}
\usepackage{amsthm, amsmath, amsfonts, amssymb, mathtools}
\usepackage{geometry}
\usepackage{indentfirst}
\usepackage{titleps}
\usepackage{soulutf8}
\usepackage{multicol}
\usepackage{tabularx}
\usepackage{pgfplots}
\usepackage{cancel}
\usepackage{import}
\usepackage{xifthen}
\usepackage{pdfpages}
\usepackage{transparent}
\usepackage{setspace}
\usepackage{graphicx}
\usepackage{float}
\usepackage{wrapfig}
\usepackage{contour}
\usepackage{mathrsfs}

\onehalfspacing

\contourlength{1pt}

\pgfplotsset{compat=1.18, width=7cm}

\newpagestyle{main}{
    %\setheadrule{0.4pt}
    \sethead{}{}{}
    %\setfootrule{0.4pt}
    \setfoot{}{\thepage}{}
}

\pagestyle{main}

\theoremstyle{plain}
\newtheorem{theorem}{Теорема}
\newtheorem{corollary}{Следствие}
\newtheorem{lemma}{Лемма}[]
\newtheorem*{lemma*}{Лемма}
\newtheorem*{definition}{Определение}
\newtheorem*{remark}{Замечание}
\newtheorem{example}{Пример}
\newtheorem*{proposition}{Предложение}
\newtheorem*{theorem*}{Теорема}
\newtheorem*{example*}{Пример}
\newtheorem*{corollary*}{Следствие}

\geometry{top=25mm}
\geometry{bottom=30mm}
\geometry{left=20mm}
\geometry{left=20mm}

\newcommand{\incfig}[1]{%
    \def\svgwidth{\columnwidth}
    \import{./figures/}{#1.pdf_tex}
}

\graphicspath{ {./figures/} }

\DeclareMathOperator{\Kerr}{Ker}
\DeclareMathOperator{\Imm}{Im}
\DeclareMathOperator{\Int}{Int}
\DeclareMathOperator{\Mat}{Mat}
\DeclareMathOperator{\End}{End}
\DeclareMathOperator{\sign}{sign}
\DeclareMathOperator{\dist}{dist}
\DeclareMathOperator{\rank}{rank}
\DeclareMathOperator{\diam}{diam}
\DeclareMathOperator{\diag}{diag}
\DeclareMathOperator{\supp}{supp}
\DeclareMathOperator{\grad}{grad}
\DeclareMathOperator{\rot}{rot}
\DeclareMathOperator{\divv}{div}
\DeclareMathOperator{\Ext}{Ext}
\DeclareMathOperator{\Id}{id}
\DeclareMathOperator{\Char}{char}
%\DeclareMathOperator{\dist}{dist}
\DeclareMathOperator*{\id}{id}
\renewcommand{\phi}{\varphi}
\renewcommand{\theta}{\vartheta}
\renewcommand{\epsilon}{\varepsilon}
\newcommand{\R}{\mathbb{R}}
\renewcommand{\C}{\mathbb{C}}
\newcommand{\Q}{\mathbb{Q}}
\newcommand{\N}{\mathbb{N}}
\setcounter{lemma}{11} % вот тут пофиксить
\newcommand{\lrhimani}[1]{\underset{#1}{\underline{\int}}}
\newcommand{\urhimani}[1]{\underset{#1}{\overline{\int}}}
\newcommand{\rhimani}[1]{\underset{#1}{\int}}
\newcommand{\mycontour}[1]{\contour{red}{#1}}
\newcommand{\charf}[1]{\chi_{#1}(x)}
\newcommand{\pfrac}[2]{\frac{\partial #1}{\partial #2}}

\begin{document}
    % \title{Математический анализ}
    % \date{12 сентября 2022}
    % \maketitle{}

    % \pagebreak
    
    \begin{corollary}
        (th. о производной обратной функции) \hfill \boxed{\textup{\textbf{12 сентября 2022}}}
        \par $D \subset \R^n$,  открыто, $f \in C^1(D, \R^n), \ f'(x)$ обратима при $\forall x \in D$
        \par Тогда для $\forall$ открытого $G \subset D \quad f(G)$ открыто
    \end{corollary}
    \begin{proof}
        Докажем сначала для $G = D$
        \[ \forall y \in f(D) \quad f^{-1}(y) := x \quad f(x) \text{ обр.,} \]
        \[ \exists U \text{ крестность } x : f(U) \text{ открыто} \]
        \[ y \in f(U) \subset f(D) \]
        \[ \Rightarrow f(U) \text{ - окр-ть } y \]
        \[\text{т. о. } f(D) \text{ открыто}\]
        \par Пусть $G \subset D$, открыто. Рассмотрим $f\big|_G \Rightarrow$
        \par $\quad \Rightarrow$ принимая доказанное $\Rightarrow$
        \par $\quad f\big|_G(G) = f(G)$ -- открыто
    \end{proof}

    $ $
    \linebreak
    \par $f$ -- биекция
    \par образ $\forall$ открытого множества открыт $\quad f$ -- окрытое отображение
    \par прообраз $\forall$ открытого множества открыт, $\quad f$ -- непрерывное отображение

    \begin{definition}
        Если и то, и другое, то $f$ -- гомеоформизм
    \end{definition}

    $ $
    \par $f : U \rightarrow V$
    \par $f^{-1} \in C(V, U)$
    \par $f \in C(U, V)$

    \begin{definition}
        Если $f : U \rightarrow V$ -- биекция, $f \in C^r(U, V)$,
        \par $\quad f^{-1} \in C^r(V, U)$, то $f$ -- диффеоморфизм гладкости $r \in [0, \infty]$
    \end{definition}

    \subsection*{Неявно заданные отображения}

    $D \subset \R^{n+m}$, открытое $\quad \Phi \in C^1(D, \R^m)$
    \par $D \ni a = (x, y)$, задана $\quad \Phi(x, y) = 0$
    \[
        \Phi'(a) = \left(\begin{array}{c c c | c c c}
            \dfrac{\partial \Phi_1}{\partial x_1} & \dots & \dfrac{\partial \Phi_1}{\partial x_n} & \dfrac{\partial \Phi_1}{\partial y_1} & \dots & \dfrac{\partial \Phi_1}{\partial y_m} \\
            \vdots & & \vdots & \vdots & & \vdots \\
            \dfrac{\partial \Phi_m}{\partial x_1} & \dots & \dfrac{\partial \Phi_m}{\partial x_n} & \dfrac{\partial \Phi_m}{\partial y_1} & \dots & \dfrac{\partial \Phi_m}{\partial y_m} \\
        \end{array}\right)
    \]

    \par $\Phi'(a)(h, k) = \Phi'_x(a)h + \Phi'_y(a)k$
    \par Ищем такую $y = \phi(k)$, что $\Phi(x, \phi(x)) = 0$

    \begin{illustration}
        $x^2 + y^2 = 1 \quad y = \sqrt{1 - x^2}$
        \par $\quad \Phi(x, y) = x^2 + y^2 + 1$
        \par $\quad \Phi'(x, y) = (2x \ 2y)$
        \par $\quad \Phi'_x = 2x$
        \par $\quad \Phi'_y = 2y$
    \end{illustration}

    \begin{illustration}
        $\Phi(x, y) = Ax + By \quad B$ -- квадратная
        \par \quad $B$ обратима $\Leftrightarrow$ уравнение разрешимо
        \[
            \Phi'_x = A \quad \Phi'_y = B    
        \]
        \[
            \underbrace{\Phi(x, y)}_{=0} = \underbrace{\Phi(x_0, y_0)}_{=0} + \Phi'(x_0, y_0)(x - x_0) +    
        \]
        \[
            + \Phi'_y(x_0, y_0)(y - y_0) + o(x-x_0, y-y_x), \quad x \rightarrow 0 \ y \rightarrow 0    
        \]
        \par $\Phi_y'(x_0, y_0)$ -- обратима
    \end{illustration}

    \begin{theorem}
        $D \subset \R^{n+m}$, открыто, $\Phi \in C^1(D, \R^m)$
        \par $\Phi(x_0, y_0) = 0 \ (x_0, y_0) \in D, \ \Phi'_y(x_0, y_0)$ -- обратимо. Тогда
        \par $\exists U$ -- окр. $x_0$, $V$ -- окр. $y_0$, $U \times V  \subset D$
        \begin{enumerate}
            \item $\forall x \in U \ \exists! y \in V : \Phi(x, y) = 0$
            \par (это задает отображение $\phi : U \rightarrow V, \ y = \phi(x)$)
            \item $\phi \in C^1(U, V), \ \forall x \in U, y \in V \quad \Phi'_y(x, \phi(x))$ обратимо
            \item $\phi'(x) = -(\Phi'_y(x, \phi(x)))^{-1} \Phi'_x(x, \phi(x))$
        \end{enumerate}
    \end{theorem}

    \begin{proof}
        $ $
        \begin{enumerate}
            \item $F(x, y) = (x, \Phi(x, y)) : D \rightarrow \R^{n + m}$
                \[
                    F'(x, y) = \left(\begin{array}{@{}c|c@{}}
                        I & 0 \\
                        \hline
                        \Phi'_x(x, y) & \Phi'_y(x, y)
                      \end{array}\right)  
                \]
                \[
                      F \in C^1(D, \R^{n+m})
                \]
                \[
                      \det F'(x, y) = \det \Phi'_y(x, y)
                \]
                \[
                      \Phi'_y(x_0, y_0) \text{ обратимо} \Rightarrow F'(x_0, y_0) \text{ обратимо}
                \]
                Th. о локальном обращении
                \par $\exists \hat U$ -- окрестность $(x_0, y_0) : F(\hat U) = \hat V$ открытое,
                \par $F\big|_{\hat U}$ -- биекция, $F^{-1} \in C^1(\hat V, \hat U), \ F(x_0, y_0) = (x_0, 0)$
                \par $\exists \delta > 0 \ \hat{\hat U} = B_\delta^n(x_0) \times B_\delta^m(y_0) \subset \hat U$
                \par $\hat{\hat U} \subset B_{\sqrt{2}\delta}^{n + m}(x_0, y_0) \subset \hat{U}, \quad$ $n, \ m$ -- размерности
                \par Тогда $F(\hat{\hat U}) \subset F(\hat U) = \hat V$
                \par $(x_0, 0) \in F(\hat{\hat U}) \subset \hat V$
                \par $\exists \epsilon > 0 : B_\epsilon^{n+m}(x_0, 0) \subset F(\hat {\hat U})$ - так как $\hat {\hat U} $ открыто
                \par $\underbrace{B^n_\epsilon(x_0)}_{= \hat{\hat V}} \times \{0\} \subset F(\hat{\hat U})$
                \par $\forall x \in \hat{\hat V} \quad F^{-1}(x, 0) =: (x_1, y)$
                \par Это означает
                \[
                      (x_1, \Phi(x, y)) = F(x_1, y) = (x, 0)
                \]
                \par $\Rightarrow x_1 = x, \ \Phi(x, y) = 0 \quad$ (отсюда следует $\epsilon < \delta$)
                \[
                      \forall x \in \underbrace{B_\epsilon^n(x_0)}_{= U} \ \exists y \in \underbrace{B_\delta^n(y_0)}_{= V} : \Phi(x, y) = 0
                \]
                Если $\exists x \in U, y_1, y_2 \in V : \Phi(x, y_1) = \Phi(x, y_2) = 0$
                \par \quad то $F(x, y_1) = (x, 0) = F(x, y_2)$
                \par \quad \quad $F$ биект. и $\hat{\hat U} \Rightarrow y_1 = y_2$ % и???
            \item $\phi = \pi_2(x, y) = \pi_2 \circ F^{-1}(x, 0) = \pi_2 \circ F^{-1} \circ E(x)$
                \par $\pi_2 : (x, y) \mapsto y \in C^1(\R^{m+n}, \R^m)$
                \par $F^{-1} \in C^1(\hat V, \hat U)$
                \par $E  : x \mapsto (x, 0) \in C^1(\R^n, \R^{n+m}) \quad E(U) = U \times \{0\} \subset \hat V$
            \item $\Phi(x, \phi(x)) = 0$
                \par $0 = \Phi'(x, \phi(x))$ % дичь какая-то хз...
                \par $\Phi'(x, \phi(x)) = \Phi'(x, \phi(x)) + \underbrace{\Phi'_y(x, \phi(x))}_{\text{обр.}} \cdot \phi'(x)$ 
        \end{enumerate}
    \end{proof}

\end{document}