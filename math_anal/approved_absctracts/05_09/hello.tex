\documentclass[a4paper, 12pt]{article}

\usepackage[english, russian]{babel}
\usepackage[T2A]{fontenc}
\usepackage[utf8]{inputenc}
\usepackage{amsthm, amsmath, amsfonts, amssymb, mathtools}
\usepackage{geometry}
\usepackage{indentfirst}
\usepackage{titleps}
\usepackage{soulutf8}
\usepackage{multicol}
\usepackage{tabularx}
\usepackage{pgfplots}
\usepackage{cancel}
\usepackage{import}
\usepackage{xifthen}
\usepackage{pdfpages}
\usepackage{transparent}
\usepackage{setspace}
\usepackage{graphicx}
\usepackage{float}
\usepackage{wrapfig}
\usepackage{contour}
\usepackage{mathrsfs}

\onehalfspacing

\contourlength{1pt}

\pgfplotsset{compat=1.18, width=7cm}

\newpagestyle{main}{
    %\setheadrule{0.4pt}
    \sethead{}{}{}
    %\setfootrule{0.4pt}
    \setfoot{}{\thepage}{}
}

\pagestyle{main}

\theoremstyle{plain}
\newtheorem{theorem}{Теорема}
\newtheorem{corollary}{Следствие}
\newtheorem{lemma}{Лемма}[]
\newtheorem*{lemma*}{Лемма}
\newtheorem*{definition}{Определение}
\newtheorem*{remark}{Замечание}
\newtheorem{example}{Пример}
\newtheorem*{proposition}{Предложение}
\newtheorem*{theorem*}{Теорема}
\newtheorem*{example*}{Пример}
\newtheorem*{corollary*}{Следствие}

\geometry{top=25mm}
\geometry{bottom=30mm}
\geometry{left=20mm}
\geometry{left=20mm}

\newcommand{\incfig}[1]{%
    \def\svgwidth{\columnwidth}
    \import{./figures/}{#1.pdf_tex}
}

\graphicspath{ {./figures/} }

\DeclareMathOperator{\Kerr}{Ker}
\DeclareMathOperator{\Imm}{Im}
\DeclareMathOperator{\Int}{Int}
\DeclareMathOperator{\Mat}{Mat}
\DeclareMathOperator{\End}{End}
\DeclareMathOperator{\sign}{sign}
\DeclareMathOperator{\dist}{dist}
\DeclareMathOperator{\rank}{rank}
\DeclareMathOperator{\diam}{diam}
\DeclareMathOperator{\diag}{diag}
\DeclareMathOperator{\supp}{supp}
\DeclareMathOperator{\grad}{grad}
\DeclareMathOperator{\rot}{rot}
\DeclareMathOperator{\divv}{div}
\DeclareMathOperator{\Ext}{Ext}
\DeclareMathOperator{\Id}{id}
\DeclareMathOperator{\Char}{char}
%\DeclareMathOperator{\dist}{dist}
\DeclareMathOperator*{\id}{id}
\renewcommand{\phi}{\varphi}
\renewcommand{\theta}{\vartheta}
\renewcommand{\epsilon}{\varepsilon}
\newcommand{\R}{\mathbb{R}}
\renewcommand{\C}{\mathbb{C}}
\newcommand{\Q}{\mathbb{Q}}
\newcommand{\N}{\mathbb{N}}
\setcounter{lemma}{11} % вот тут пофиксить
\newcommand{\lrhimani}[1]{\underset{#1}{\underline{\int}}}
\newcommand{\urhimani}[1]{\underset{#1}{\overline{\int}}}
\newcommand{\rhimani}[1]{\underset{#1}{\int}}
\newcommand{\mycontour}[1]{\contour{red}{#1}}
\newcommand{\charf}[1]{\chi_{#1}(x)}
\newcommand{\pfrac}[2]{\frac{\partial #1}{\partial #2}}

\DeclareMathOperator{\Kerr}{Ker}
\DeclareMathOperator{\Imm}{Im}
\DeclareMathOperator{\Int}{Int}
\DeclareMathOperator{\Mat}{Mat}
\DeclareMathOperator{\rank}{rank}
\DeclareMathOperator*{\id}{id}
\renewcommand{\phi}{\varphi}
\renewcommand{\theta}{\vartheta}
\renewcommand{\epsilon}{\varepsilon}
\newcommand{\R}{\mathbb{R}}
\renewcommand{\C}{\mathbb{C}}

\begin{document}
    
    \author{zimch}
    \title{Математический анализ}
    \maketitle{}

    \pagebreak
    \section*{Повторение}
    \subsection*{Линейные и нормированные пространства}
    
    $\mathit{L}$ -- линейное пространство
    \newline
    \par
    \begin{tabularx}{\textwidth}{X l}
        $\| \cdot \|$ -- норма & $\| \alpha x \| = |\alpha| \| x \|, \forall \alpha \in \R (\C), x \in \mathit L$
    \end{tabularx}
    \newline
    \par
    \begin{tabularx}{\textwidth}{X l}
        Нормированное пространство & $\| x + y \| \le \| x \| + \| y \|$ \\
        & $\| x \| = 0 \Rightarrow x = 0$ \\
    \end{tabularx}

    \begin{remark}
        Норма всегда порождает метрику (нормированное $\Rightarrow$ метрическое).
    \end{remark}

    \begin{remark}
        $\forall$ конечномерное пространство полное.
    \end{remark}

    \begin{definition}
        Полное нормированное пространство $\leftrightharpoons$ банохово.
    \end{definition}

    \begin{illustration}
        Неполное нормированное пространство:
        \[
            C([0; 1]), \quad \| f \|_< = \int_0^1 |f(x)| dx
        \]
        \[
            \int_0^\frac12 f_n(x) dx \rightarrow 0 \quad \exists N: n > N < \frac \epsilon 2
        \]
        \[
            \int_\frac12^1 (1 - f_n(x)) dx \rightarrow 0 \quad \exists N: n > N < \frac \epsilon 2
        \]
        \[
            \forall \epsilon \ \exists N \ \forall n, m > \int_0^1 |f_n(x) - f_m(x)| dx < \epsilon   
        \]
        \[
            \mathit{L}(0, 1) = \{ f : [0, 1] \rightarrow \R : \int_0^1 |f_n(x)| dx < \infty \}    
        \]
    \end{illustration}

    \subsection*{Линейные операторы}

    \begin{definition}
        Линейный оператор
        \[
            A(\alpha x + \beta y) = \alpha Ax + \beta Ay \quad A : L \rightarrow M \quad \text{, где } M = \R \setminus \C, A \text{ - функционал(?)}    
        \]
    \end{definition}

    \begin{remark}
        Операторы из $\R^m$ в $\R^n$ $\leftrightarrow$ матрицы $\Mat^{n, m}$ 
    \end{remark}

    \[
        \| A \| = \sup_{x \not= 0} \frac{\| Ax \|_\mathit{M}}{\| x \|_\mathit{L}}, \quad M, L \text{ - нормированные пространства}
    \]

    \begin{illustration}
        Неограниченный оператор:
        \[
            L = C'([0, 1]), \quad M = C([0, 1]) \quad \| f \| = \sup_{[0, 1]} |f|    
        \]
        \[
            \| f \| = \max_{[0,1]} |f|    
        \]
        \[
            (Af)(x) = f'(x)     
        \]
        \[
            f_n(x) = x^n \quad \| f_n \| = 1 \ \forall n \quad Af_n = f_n' = nx^{n-1} \quad \| Af \| = n    
        \]

    \[\raisebox{-15mm}{\begin{tikzpicture}
        \begin{axis}[ymin=0, ymax=1, xmin=0, xmax=1]
            \addplot[color=red, dashed, mark=x, samples=100]{x^2};
        \end{axis}
    \end{tikzpicture}}
    \qquad
    \frac{\| Af_n \|}{\| f_n \|} \xrightarrow[n \rightarrow \infty]{} \infty
    \]
    \end{illustration}

    \begin{proposition}
        \[
            \| A \| = \sup_{x \in B_1(x) \setminus \{ 0 \}} \|Ax \| = \sup_{a \in S_1(0)} \| Ax \| = \sup_{x \in B_1(0) \setminus \{ x \}} = \inf \{c : \| Ax \| \le c \| x \|\} \quad \forall x \in L
        \]
        \par$B_r(x) = \{ y \in L  : \| y - x \| < r\}$ -- открытый шар радиуса $r$
        \par$B_r[x] = \{ y \in L  : \| y - x \| \le r\}$ -- замкнутый шар радиуса $r$
        \par$\bar B_r(x) \not= B_r[x]$, где $\bar B_r(x)$ -- замыкание
        \par$S_r(x) = \{ y \in L  : \| y - x \| = r\}$ -- сфера
    \end{proposition}

    \begin{proposition}
        $A \in B(L) \Leftrightarrow A$ непр. в точке $0 \Leftrightarrow A$ непр. в $\forall x \in L \Leftrightarrow A$ равн. непр. на $L$.
    \end{proposition}

    \begin{remark}
        \[
            \| A_1 \cdot A_2 \| \le \| A_1 \| \cdot \| A_2 \|   
        \]
        \[
            A : \R^n \rightarrow \R^n, \ A \in \Mat^{n, n} \quad \| A \| \leq \sqrt{\sum_{i, k = 1}^n |a_{i, k}^2|}
        \]
    \end{remark}

    \begin{definition}
        Матрицы $\|\cdot\|$ и $|\cdot|$ эквивалентны, если $\exists c_1, c_2 > 0$ т. ч.
        \[
            \forall x \in L \ c_1 \|x\| \le |x| \le c_2 \|x_2\|
        \]
        \par Тогда $\| A \| \thicksim \sum_{i, k = 1}^n |a_{ik}| \thicksim \max_{i, k \in \{1, \dots, n\}} |a_{ik}| \thicksim \sqrt{\sum_{i, k = 1}^n |a_{ik}|^2}$
    \end{definition}

    \begin{remark}
        \[
            A : \R^n  \rightarrow \R \ \exists a \in \R^n \ \forall x \in \R^n : Ax = (a; x) \quad \| A \| \underset{B(\R^n, \R)}{=} \| a \|_{\R^n}   
        \]
        \[
            A : \R \rightarrow \R^n \ \exists a \in \R^n \ \exists x \in \R : Ax = a \cdot x \quad \| A \| \underset{B(\R, \R^n)}{=} \| a \|_{\R^n}  
        \]
    \end{remark}

    \pagebreak

    \subsection*{Обратный оператор}
    
    $A : L \rightarrow M$ -- линейный оператор

    \begin{enumerate}
        \item $\exists A : M \rightarrow L \ : \ AB = I_M$ -- ед. оператор в пространстве $M$
        \par $B \leftrightharpoons$ правый обратный
        \item $\exists C : M \rightarrow L \ : \ CA = I_l$ -- ед. оператор в пространстве $L$
        \par $C \leftrightharpoons$ левый обратный
        \item $\exists$ оба и равны, ьл $A^{-1} \leftrightharpoons$ обратный оператор
    \end{enumerate}

    $A \in \Mat^{n} : \exists A^{-1} \Leftrightarrow \det A \not= 0 \Leftrightarrow \Kerr A = \{0\} \Leftrightarrow \dots \Leftrightarrow \rank A = n$

    \begin{theorem}
        $A \in B(\R^n), \exists A^{-1}, B \in B(\R^n), \| B - A \| < \frac{1}{\| A^{-1} \|}$
        \par Тогда $B$ обратим,
        \[
            \| B^{-1} \| \le \frac{1}{\left\| \frac{1}{A^{-1}} \right\| - \| B - A \|}, \| B^{-1} - A^{-1} \| \le \frac{\| A^{-1} \| \cdot \| B - A \|}{\left\|\frac{1}{A^{-1}}\right\| - \| B - A \|}  
        \]
    \end{theorem}

    \begin{proof}
        $x \in \R^n$
        \[
            \| Bx \| = \| Ax - (A-B)x \| \ge \| Ax \| - \| (B-A)x \| \ge \frac{\| x \|}{\| A^{-1} \|} - \| B-A \| \cdot \| x \| = (\frac{1}{\|A^{-1}\|} - \|B-A\|\|x\|)
        \]
        \par Так как:
        \[
            \|Ax\| \ge \frac{\|x\|}{\|A^{-1}\|} \quad x = (A^{-1})(Ax) \quad \|x\| \le \|A^{-1}\| \cdot \|Ax\|  
        \]
        \[
            Bx = 0 \Rightarrow \|x\| = 0 \Rightarrow x = 0 \quad \Kerr B = \{0\} \Rightarrow \exists B^{-1}  
        \]
        \[
            \begin{matrix} y = Bx \\ x = B^{-1}y \end{matrix} \quad \|y\| \ge (\frac{1}{\|A^{-1}\|} - \|B-A\|)\|B^{-1}y\|, \ \forall y \in \R^n  
        \]
        \[
            \Rightarrow \|B^{-1}\| \le \frac{1}{\frac{1}{\|A^{-1}\|} - \|B-A\|}    
        \]
        \[
            B^{-1}A^{-1} = B^{-1}(I-BA^{-1}) = B^{-1}(A-B)A^{-1}
        \]
        \[
            \|B^{-1}-A^{-1}\| \le \frac{\|A^{-1}\| \cdot \|B-A\|}{\frac{1}{\|A^{-1}\|} - \|B-A\|}
        \]
    \end{proof}

    \begin{remark}\leavevmode
        \begin{enumerate}
            \item Множество операторов открыто
            \item Отображение $A \mapsto A^{-1}$ непрерывно
        \end{enumerate}
    \end{remark}

    \pagebreak
    
    \section*{Дифференцирование обратной функции}

    $D \subset \R^n \quad f : D \rightarrow \R^n \quad x \in \Int D \quad \exists A \in B(\R^n. \R)$
    
    \begin{definition}
        Если $f(x + h) = f(x) + Ah + o(\|h\|), \quad h \rightarrow 0$, тогда говорят $A$ -- производная $f$ в точке $x$.
    \end{definition}

    \begin{figure}[ht]
        \centering
        \incfig{figurka}
        \label{fig:figurka}
    \end{figure}

    % \[
    %     y_0 = f(x_0) \in \Int f(D)  
    % \]
    % \[
    %     \exists (f^{-1})(y_0) = (f'(x_0))^{-1}    
    % \]

    Рассмотрим $f^{-1} \circ f = \id_D$

    Продифференцируем : $(f^{-1})'(\underbrace{f(x_0)}_{= y_0}) \cdot f'(x_0) = I$

    Пусть теперь есть функция на открытом множестве:

    $D$ открыто $\quad f \in C'(D, \R^n) \quad x _0 \in D \ f'(D_0)$ обратима

    \begin{illustration}\leavevmode
        \begin{enumerate}
            \item $f(x, y) = \begin{pmatrix}e^x \cos y \\ e^x \sin y \end{pmatrix}$
            \par $f : \R^2 \rightarrow \R^2$
            \[
                f'(x, y) = \begin{pmatrix} e^x\cos y - e^x\sin y \\ e^x\sin y + e^x\cos y \end{pmatrix}    
            \]
            \item $n = 1 \quad f \in C^1(D, \R)$
        \end{enumerate}
    \end{illustration}



\end{document}