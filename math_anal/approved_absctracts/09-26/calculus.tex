\documentclass[a4paper, 12pt]{article}

\usepackage[english, russian]{babel}
\usepackage[T2A]{fontenc}
\usepackage[utf8]{inputenc}
\usepackage{amsthm, amsmath, amsfonts, amssymb, mathtools}
\usepackage{geometry}
\usepackage{indentfirst}
\usepackage{titleps}
\usepackage{soulutf8}
\usepackage{multicol}
\usepackage{tabularx}
\usepackage{pgfplots}
\usepackage{cancel}
\usepackage{import}
\usepackage{xifthen}
\usepackage{pdfpages}
\usepackage{transparent}
\usepackage{setspace}
\usepackage{graphicx}
\usepackage{float}
\usepackage{wrapfig}
\usepackage{contour}
\usepackage{mathrsfs}

\onehalfspacing

\contourlength{1pt}

\pgfplotsset{compat=1.18, width=7cm}

\newpagestyle{main}{
    %\setheadrule{0.4pt}
    \sethead{}{}{}
    %\setfootrule{0.4pt}
    \setfoot{}{\thepage}{}
}

\pagestyle{main}

\theoremstyle{plain}
\newtheorem{theorem}{Теорема}
\newtheorem{corollary}{Следствие}
\newtheorem{lemma}{Лемма}[]
\newtheorem*{lemma*}{Лемма}
\newtheorem*{definition}{Определение}
\newtheorem*{remark}{Замечание}
\newtheorem{example}{Пример}
\newtheorem*{proposition}{Предложение}
\newtheorem*{theorem*}{Теорема}
\newtheorem*{example*}{Пример}
\newtheorem*{corollary*}{Следствие}

\geometry{top=25mm}
\geometry{bottom=30mm}
\geometry{left=20mm}
\geometry{left=20mm}

\newcommand{\incfig}[1]{%
    \def\svgwidth{\columnwidth}
    \import{./figures/}{#1.pdf_tex}
}

\graphicspath{ {./figures/} }

\DeclareMathOperator{\Kerr}{Ker}
\DeclareMathOperator{\Imm}{Im}
\DeclareMathOperator{\Int}{Int}
\DeclareMathOperator{\Mat}{Mat}
\DeclareMathOperator{\End}{End}
\DeclareMathOperator{\sign}{sign}
\DeclareMathOperator{\dist}{dist}
\DeclareMathOperator{\rank}{rank}
\DeclareMathOperator{\diam}{diam}
\DeclareMathOperator{\diag}{diag}
\DeclareMathOperator{\supp}{supp}
\DeclareMathOperator{\grad}{grad}
\DeclareMathOperator{\rot}{rot}
\DeclareMathOperator{\divv}{div}
\DeclareMathOperator{\Ext}{Ext}
\DeclareMathOperator{\Id}{id}
\DeclareMathOperator{\Char}{char}
%\DeclareMathOperator{\dist}{dist}
\DeclareMathOperator*{\id}{id}
\renewcommand{\phi}{\varphi}
\renewcommand{\theta}{\vartheta}
\renewcommand{\epsilon}{\varepsilon}
\newcommand{\R}{\mathbb{R}}
\renewcommand{\C}{\mathbb{C}}
\newcommand{\Q}{\mathbb{Q}}
\newcommand{\N}{\mathbb{N}}
\setcounter{lemma}{11} % вот тут пофиксить
\newcommand{\lrhimani}[1]{\underset{#1}{\underline{\int}}}
\newcommand{\urhimani}[1]{\underset{#1}{\overline{\int}}}
\newcommand{\rhimani}[1]{\underset{#1}{\int}}
\newcommand{\mycontour}[1]{\contour{red}{#1}}
\newcommand{\charf}[1]{\chi_{#1}(x)}
\newcommand{\pfrac}[2]{\frac{\partial #1}{\partial #2}}

\DeclareMathOperator{\Kerr}{Ker}
\DeclareMathOperator{\Imm}{Im}
\DeclareMathOperator{\Int}{Int}
\DeclareMathOperator{\Mat}{Mat}
\DeclareMathOperator{\rank}{rank}
\DeclareMathOperator{\diam}{diam}
\DeclareMathOperator*{\id}{id}
\renewcommand{\phi}{\varphi}
\renewcommand{\theta}{\vartheta}
\renewcommand{\epsilon}{\varepsilon}
\newcommand{\R}{\mathbb{R}}
\renewcommand{\C}{\mathbb{C}}
\newcommand{\Q}{\mathbb{Q}}
\newcommand{\N}{\mathbb{N}}
\setcounter{lemma}{3}
\DeclareMathOperator{\lrhimani}{\underset{\Pi}{\underline{\int}}}
\DeclareMathOperator{\urhimani}{\underset{\Pi}{\overline{\int}}}
\DeclareMathOperator{\rhimani}{\underset{\Pi}{\int}}

\begin{document}

    \title{Математический анализ}
    \date{26 сентября 2022}
    \maketitle

    \pagebreak

    \subsection*{Определение интеграла Римана через интегральные суммы}

    $\Pi \subset \R^n, \ f : \Pi \rightarrow \R$ огр.
    \par $p$ -- разбиение $\Pi$, $= \{\pi_i, \ i = 1, \dots, N\}$
    \par $\Xi = \{\xi_i \in \pi_i, \ | \ i = 1, \dots, N\}$
    \par $\sum(f, p, \Xi) := \sum_{i=1}^N f(\xi) v(\pi_i)$ -- интегральная сумма Римана

    \begin{definition}
        Если $\exists I \in \R : \forall \{p_k\}_{k=1}^\infty : d(p_k) \xrightarrow[k \rightarrow]{} 0 \ \forall \{\Xi\}_{k=1}^\infty$
        \[
            \sum(f, p_k, \Xi_k) \xrightarrow[k \rightarrow \infty]{} I, \ \text{ то } f \text{ интегрируема по Риману и } I = \rhimani f    
        \]
    \end{definition}

    \begin{theorem}
        $ $
        \par $\exists I \ \forall \{p_k\} : d(p_k) \xrightarrow[k \rightarrow \infty]{} 0 \ \forall \{\Xi\} \ \sum(f, p_k, \Xi_k) \xrightarrow[k \rightarrow \infty]{} I \Leftrightarrow \lrhimani f = \urhimani f = \rhimani f$
    \end{theorem}
    
    \begin{proof}
        $ $
        \begin{enumerate}
            \item[$\boxed{\Rightarrow}$] $\epsilon, p_k : d()p_k \xrightarrow[k \rightarrow \infty]{} 0 \ \forall \pi \in p_k \ \exists \xi \in \pi:$
                \[
                    f(\xi) - \inf_\pi f < \epsilon
                \]
                Получим $\Xi_k \ \sum(f, p_K, \Xi_k) - L(f, p_k) = \sum_{\pi \in p_k}(f(\xi(\pi)) - \inf_\pi f) \cdot v(\pi) \le$
                \par \quad $\le \epsilon \cdot \sum_{\pi \in p} v(\pi) = \epsilon \cdot v(\pi)$
                \par По \textit{Л. 3} $L(f, p_k) \xrightarrow[k \rightarrow 0]{} \lrhimani f \Rightarrow 0 \le I - \lrhimani f \le \epsilon \cdot v(\pi)$
                \[
                    \begin{rcases}
                        \forall \epsilon \Rightarrow \lrhimani f = I \\
                        \text{Аналогично } \urhimani f = I
                    \end{rcases} \Rightarrow \exists \rhimani f = i
                \]
            \item[$\boxed{\Leftarrow}$] Пусть $\lrhimani f = \urhimani f = \rhimani f$. Возьмем произвольные
                \[
                    \{p_k\}, \ d(p_k) \rightarrow 0, \ \{\Xi_k\} \quad _{(*)}:    
                \]
                \[
                    L(f, p_k) \le \sum(f, p_k, \Xi_k) = \sum_{\pi \in p_k}\underbrace{f(\xi(\pi))}_{(*)} v(\pi) \le U(f, p_k)    
                \]
                \[
                    _{(*)} \quad \inf_\pi \le \dots \le \sup_\pi f    
                \]
                \[
                    L(f, p_k) \xrightarrow[\text{Л. 3, } k \rightarrow \infty]{} \lrhimani f = \urhimani f \xleftarrow[\text{Л. 3, } k \rightarrow \infty]{} U(f, p_k)    
                \]
                \[
                    \Rightarrow \sum(f, p_k, \Xi_k) \xrightarrow[k \rightarrow \infty]{} I
                \]
        \end{enumerate}
    \end{proof}

    \subsection*{Множество меры ноль}

    \begin{definition}
        $E \subset \R^n$ имеет меру ноль, если $\forall \epsilon > 0 \ \exists$ покрытие $E \subset \bigcup_{k = 1}^\infty C_k$, где $C_k$ -- открытые кубы
        \[
            \sum_{k=1}^\infty v(C_k) \le \epsilon \quad \quad \mu(E) = 0 \ \text{ -- мера}   
        \]
    \end{definition}

    \begin{remark}
        Открытые кубы $\Leftrightarrow$ замкнутые
    \end{remark}
    \begin{remark}
        $E_1 \subset E, \ \mu(E) = 0 \Rightarrow \mu(E_1) = 0$
    \end{remark}

    \begin{lemma}
        $\mu(E_k) = 0 \Rightarrow \forall k \in \N \Rightarrow \mu(\bigcup_{k=1}^\infty E_k) = 0$
    \end{lemma}

    \begin{proof}
        $\forall k \ \exists$ покрытие кубами с $\sum$ объемов $ < \frac \epsilon 2^k$
        \par Тогда $E = \bigcup_{k=1}^\infty E_k$ будут покрыты и $\sum$ объемов $< \epsilon \cdot \sum_{k=1}^\infty \frac 1 2^k = \epsilon$
    \end{proof}

    \begin{definition}
        $E \subset \R^n$ имеет объем ноль, если $\forall \epsilon \ \exists$ конечное покрытие
        \par $E = \bigcup_{k=1}^\infty C_k$, где $c_k$ -- открытый куб
        \[
            \sum_{k=1}^N v(C_k) < \epsilon \quad \quad v(E) = 0   
        \]
    \end{definition}

    \begin{remark}
        $ $
        \begin{enumerate}
            \item открытые $\Leftrightarrow$ замкнутые кубы
            \item $v(E) = 0 \Rightarrow \mu(E) = 0$
        \end{enumerate}
    \end{remark}

    \begin{theorem}
        \par $[a, b] \subset \R$ не может иметь объем $0$
    \end{theorem}
    \begin{proof}
        докажем, что если $[a, b] \subset \bigcup_{k=1}^N C_k$,
        \[
            C_k \text{ -- отрезки, то } \sum_{k=1}^N v(C_k) \ge b - a
        \]
        база : $N = 1$
        \par \quad $[a, b] \subset C_1 \Rightarrow v(C_1) \ge b - a$
        \par переход : $N + 1$
        \par \quad $a \in U_{k=1}^{N+1} C_k \Rightarrow \exists k : a \in C_k$ \text{ перенумеруем } $C_k$ так, чтобы $a \in C_1 = [\alpha, \beta]$
        \[
            \alpha \le a \le \beta \le b % ВОТ ТУТ БЫ НАРИСОВАТЬ ОТРЕЗОК... НО Я ЧЕТА НЕ МОГУ НАЙТИ КАК ;(    
        \]
        Если $b \in [\alpha, \beta]$, то $[a, b] \subset [\alpha, \beta]$,
        \[
            \sum_{k=1}^{N+1} v(C_k) > v(C_1) = \beta - \alpha \ge b - a    
        \]
        Если $b \not \in [\alpha, \beta], \ b > \beta$ % ТУТ ТОЖЕ РИСУНОК ПЖПЖПЖ
        \[
            (\beta, b] \subset \bigcup_{k=2}^{N+1} C_k    
        \]
        \[
            \Rightarrow [\beta, b] \subset \bigcup_{k=2}^{N+1}  C_k \xRightarrow[\text{инд. п.}]{} \sum_{k=2}^{N+1} v(C_k) \ge b - \beta
        \]
        \[
            v(C_1) \ge \beta - a    
        \]
        \[
            \Rightarrow \sum_{k=1}^{N+1} v(C_k) \ge b - a    
        \]
    \end{proof}

\end{document}