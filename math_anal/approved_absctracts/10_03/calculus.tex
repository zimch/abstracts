\documentclass[a4paper, 12pt]{article}

\usepackage[english, russian]{babel}
\usepackage[T2A]{fontenc}
\usepackage[utf8]{inputenc}
\usepackage{amsthm, amsmath, amsfonts, amssymb, mathtools}
\usepackage{geometry}
\usepackage{indentfirst}
\usepackage{titleps}
\usepackage{soulutf8}
\usepackage{multicol}
\usepackage{tabularx}
\usepackage{pgfplots}
\usepackage{cancel}
\usepackage{import}
\usepackage{xifthen}
\usepackage{pdfpages}
\usepackage{transparent}
\usepackage{setspace}
\usepackage{graphicx}
\usepackage{float}
\usepackage{wrapfig}
\usepackage{contour}
\usepackage{mathrsfs}

\onehalfspacing

\contourlength{1pt}

\pgfplotsset{compat=1.18, width=7cm}

\newpagestyle{main}{
    %\setheadrule{0.4pt}
    \sethead{}{}{}
    %\setfootrule{0.4pt}
    \setfoot{}{\thepage}{}
}

\pagestyle{main}

\theoremstyle{plain}
\newtheorem{theorem}{Теорема}
\newtheorem{corollary}{Следствие}
\newtheorem{lemma}{Лемма}[]
\newtheorem*{lemma*}{Лемма}
\newtheorem*{definition}{Определение}
\newtheorem*{remark}{Замечание}
\newtheorem{example}{Пример}
\newtheorem*{proposition}{Предложение}
\newtheorem*{theorem*}{Теорема}
\newtheorem*{example*}{Пример}
\newtheorem*{corollary*}{Следствие}

\geometry{top=25mm}
\geometry{bottom=30mm}
\geometry{left=20mm}
\geometry{left=20mm}

\newcommand{\incfig}[1]{%
    \def\svgwidth{\columnwidth}
    \import{./figures/}{#1.pdf_tex}
}

\graphicspath{ {./figures/} }

\DeclareMathOperator{\Kerr}{Ker}
\DeclareMathOperator{\Imm}{Im}
\DeclareMathOperator{\Int}{Int}
\DeclareMathOperator{\Mat}{Mat}
\DeclareMathOperator{\End}{End}
\DeclareMathOperator{\sign}{sign}
\DeclareMathOperator{\dist}{dist}
\DeclareMathOperator{\rank}{rank}
\DeclareMathOperator{\diam}{diam}
\DeclareMathOperator{\diag}{diag}
\DeclareMathOperator{\supp}{supp}
\DeclareMathOperator{\grad}{grad}
\DeclareMathOperator{\rot}{rot}
\DeclareMathOperator{\divv}{div}
\DeclareMathOperator{\Ext}{Ext}
\DeclareMathOperator{\Id}{id}
\DeclareMathOperator{\Char}{char}
%\DeclareMathOperator{\dist}{dist}
\DeclareMathOperator*{\id}{id}
\renewcommand{\phi}{\varphi}
\renewcommand{\theta}{\vartheta}
\renewcommand{\epsilon}{\varepsilon}
\newcommand{\R}{\mathbb{R}}
\renewcommand{\C}{\mathbb{C}}
\newcommand{\Q}{\mathbb{Q}}
\newcommand{\N}{\mathbb{N}}
\setcounter{lemma}{11} % вот тут пофиксить
\newcommand{\lrhimani}[1]{\underset{#1}{\underline{\int}}}
\newcommand{\urhimani}[1]{\underset{#1}{\overline{\int}}}
\newcommand{\rhimani}[1]{\underset{#1}{\int}}
\newcommand{\mycontour}[1]{\contour{red}{#1}}
\newcommand{\charf}[1]{\chi_{#1}(x)}
\newcommand{\pfrac}[2]{\frac{\partial #1}{\partial #2}}


\begin{document}

    \title{Математический анализ}
    \date{3 октября 2022}
    \maketitle

    \pagebreak

    \subsection*{Характеристическая функция множества}

    \begin{definition}
        Характеристическая функция множества 
        \begin{equation*}
            \chi_{E}(x) = 
             \begin{cases}
               1 & x \in E \\
               0 & x \not \in E
             \end{cases}
            \end{equation*}
    \end{definition}


    \begin{lemma}
        $\{\text{Точки разрыва} \ \chi_{E}(x)\} = \delta E$ 
    \end{lemma}

    \begin{proof}
       \par $x \in \Int E \cup \Ext E $
       \par На внутренних точках $\lim \chi_{E}(x) = 1 = \chi_{E}(x)$
       \par На внешних точках $\lim \chi_{E}(x) = 0 = \chi_{E}(x)$
       \bigskip
       \par На границе разрывна - очев

    \end{proof}

    \begin{definition}
        \par $E \in \Pi$, $\Pi$ - п/п, $f: E \rightarrow \R$ - ограничена
        \bigskip
        \par $\rhimani E f = \rhimani \Pi f\cdot \chi_{E}$
        \bigskip
        \par $\tilde f = f \cdot \charf{E}$

    \end{definition}

    \begin{lemma}
        \par $\mu(\delta E) = 0$, $f: E \rightarrow \R$ - почти везде непрерывна на $E$
        \bigskip
        \par Тогда $\exists \rhimani E f$
    \end{lemma}

    \begin{proof}
        \par Необходимо доказать, что $\exists \rhimani \Pi \tilde{f}$
        \par $\{\text{Точки разрыва} \ \tilde{f} \ \text{на} \ \Pi \} = \{\text{Точки разрыва} \ \tilde{f} \ \text{на} \ \Int E \} \cup \{\text{Точки разрыва} \ \tilde{f} \ \text{на} \ \delta E \} \cup \underbrace{\{\text{Точки разрыва} \ \tilde{f} \ \text{на} \ \Ext E \}}_{\emptyset}$
        \par $\{\text{Точки разрыва} \ \tilde{f} \ \text{на} \ \Pi \} \subset \{\text{Точки разрыва} \ f \ \text{на} \ \Int E \} \cup \delta E$
        \par $\mu(\{\text{Точки разрыва} \ \tilde{f} \ \text{на} \ \Pi \}) = 0$ - по критерию Лебега существует интеграл 
    \end{proof}

    \begin{definition}
        \par $E$ - ограничено, $\mu(\delta E) = 0$
        \par $\exists \Pi \in \R^n$ - п/п: $E \in \Pi$
        \par $\upsilon(E) = \rhimani E 1 = \rhimani \Pi \charf{E}$ - Жорданов объём

    \end{definition}

\end{document}