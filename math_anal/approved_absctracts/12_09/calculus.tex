\documentclass[a4paper, 12pt]{article}

\usepackage[english, russian]{babel}
\usepackage[T2A]{fontenc}
\usepackage[utf8]{inputenc}
\usepackage{amsthm, amsmath, amsfonts, amssymb, mathtools}
\usepackage{geometry}
\usepackage{indentfirst}
\usepackage{titleps}
\usepackage{soulutf8}
\usepackage{multicol}
\usepackage{tabularx}
\usepackage{pgfplots}
\usepackage{cancel}
\usepackage{import}
\usepackage{xifthen}
\usepackage{pdfpages}
\usepackage{transparent}
\usepackage{wrapfig}
\usepackage{setspace}

\onehalfspacing

\pgfplotsset{compat=1.18, width=7cm}

\newpagestyle{main}{
    %\setheadrule{0.4pt}
    \sethead{}{}{}
    %\setfootrule{0.4pt}
    \setfoot{}{\thepage}{}
}

\pagestyle{main}

\theoremstyle{plain}
\newtheorem{theorem}{Теорема}
\newtheorem{corollary}{Следствие}
\newtheorem{lemma}{Лемма}[]
\newtheorem*{definition}{Определение}
\newtheorem*{remark}{Замечание}
\newtheorem{illustration}{Пример}
\newtheorem*{proposition}{Предложение}

\geometry{top=25mm}
\geometry{bottom=30mm}
\geometry{left=20mm}
\geometry{left=20mm}

\newcommand{\incfig}[1]{%
    \def\svgwidth{\columnwidth}
    \import{./figures/}{#1.pdf_tex}
}

\DeclareMathOperator{\Kerr}{Ker}
\DeclareMathOperator{\Imm}{Im}
\DeclareMathOperator{\Int}{Int}
\DeclareMathOperator{\Mat}{Mat}
\DeclareMathOperator{\rank}{rank}
\DeclareMathOperator{\diam}{diam}
\DeclareMathOperator*{\id}{id}
\renewcommand{\phi}{\varphi}
\renewcommand{\theta}{\vartheta}
\renewcommand{\epsilon}{\varepsilon}
\newcommand{\R}{\mathbb{R}}
\renewcommand{\C}{\mathbb{C}}
\newcommand{\Q}{\mathbb{Q}}
\newcommand{\N}{\mathbb{N}}
\setcounter{lemma}{9} % вот тут пофиксить
\DeclareMathOperator{\lrhimani}{\underset{\Pi}{\underline{\int}}}
\DeclareMathOperator{\urhimani}{\underset{\Pi}{\overline{\int}}}
\DeclareMathOperator{\rhimani}{\underset{\Pi}{\int}}

\DeclareMathOperator{\Kerr}{Ker}
\DeclareMathOperator{\Imm}{Im}
\DeclareMathOperator{\Int}{Int}
\DeclareMathOperator{\Mat}{Mat}
\DeclareMathOperator{\rank}{rank}
\DeclareMathOperator*{\id}{id}
\renewcommand{\phi}{\varphi}
\renewcommand{\theta}{\vartheta}
\renewcommand{\epsilon}{\varepsilon}
\newcommand{\R}{\mathbb{R}}
\renewcommand{\C}{\mathbb{C}}

\begin{document}
    \title{Математический анализ}
    \date{12 сентября 2022}
    \maketitle{}

    \pagebreak

    \begin{corollary}
        (th. о локальном обращении)
        \par $D \subset \R^n$,  открыто, $f \in C^1(D, \R^n), \ f'(x)$ обратима при $\forall x \in D$
        \par Тогда для $\forall$ открытого $G \subset D \quad f(G)$ открыто
    \end{corollary}
    \begin{proof}
        Докажем сначала для $G = D$
        \[ \forall y \in f(D) \quad f^{-1}(y) := x \quad f(x) \text{ обр.,} \]
        \[ \exists U \text{ крестность } x : f(U) \text{ открыто} \]
        \[ y \in f(U) \subset f(D) \]
        \[ \Rightarrow f(U) \text{ - окр-ть } y \]
        \[\text{т. о. } f(D) \text{ открыто}\]
        \par Пусть $G \subset D$, открыто. Рассмотрим $f\big|_G \Rightarrow$
        \par $\quad \Rightarrow$ принимая доказанное $\Rightarrow$
        \par $\quad f\big|_G(G) = f(G)$ -- открыто
    \end{proof}

    $ $
    \linebreak
    \par $f$ -- биекция
    \par образ $\forall$ открытого множества открыт $\quad f$ -- окрытое отображение
    \par прообраз $\forall$ открытого множества открыт, $\quad f$ -- непрерывное отображение

    \begin{definition}
        Если и то, и другое, то $f$ -- гомеоформизм
    \end{definition}

    $ $
    \par $f : U \rightarrow V$
    \par $f^{-1} \in C(V, U)$
    \par $f \in C(U, V)$

    \begin{definition}
        Если $f : U \rightarrow V$ -- биекция, $f \in C^r(U, V)$,
        \par $\quad f^{-1} \in C^r(V, U)$, то $f$ -- диффеоморфизм гладкости $r \in [0, \infty]$
    \end{definition}

\end{document}